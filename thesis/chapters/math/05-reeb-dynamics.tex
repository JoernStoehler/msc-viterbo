% !TeX root = ../../main.tex
\section{Reeb dynamics and closed characteristics}
We next consider the Reeb dynamics in the polytope case. Again we assume familiarity with contact topology in the smooth case. The 1-form we are interested in is \(\alpha = \lambda|_{\partial K}\). We for now do not assume that \(\partial K\) is of contact type, i.e. we allow that \(\alpha|_{F_i \cap F_j} = 0\) on some 2-faces \(F_i \cap F_j\) (Lagrangian 2-faces). We are slightly more general here than \cite{CH2021}, who assume no Lagrangian 2-faces.

\paragraph{Regular case.} At \(x\in\partial K\) with outward unit normal \(n_x\), any \(v\in T_x\partial K\) satisfies \(\inner{J v}{J n_x}=\inner{v}{n_x}=0\). Normalizing to \(\alpha(R)=1\) gives
\[
  R(x) = \frac{2}{\inner{x}{n_x}}\, J n_x.
\]
The Reeb flow \(\dot x = R(x)\) preserves \(\alpha\) and \(d\alpha\). Periodic solutions are Reeb orbits. Closed characteristics are loops \(\gamma\) with \(\dot\gamma(t) \in \R_+ R(\gamma(t))\) for all \(t\).

\paragraph{Polytope case.} On facet interiors the Reeb vector field matches the facet velocity. Generalized closed characteristics are loops \(\gamma \in W^{1,2}(\T,\partial K)\) with
\[
  \dot\gamma(t) \in \R_+ \mathrm{conv}\{p_i : \gamma(t) \in F_i\} \quad \text{a.e.}
\]
Thus the concepts of generalized Reeb orbits, closed characteristics, and Hamiltonian orbits coincide up to time reparametrization that preserve orientation. We will in the following often talk about "orbits", and mean "Reeb orbits".

\begin{fact}\label{fact:closed-characteristic}
Any closed characteristic \(\gamma\) is parametrized uniquely as \(\gamma\in W^{1,2}(\T,\partial K)\) with period \(T>0\) such that \(\dot\gamma(t) \in \operatorname{conv}\{p_i : \gamma(t) \in F_i\}\) almost everywhere.
\end{fact}

\begin{lemma}[Flow on facets]\label{lem-facet-flow}
If an orbit \(\gamma\) meets the interior of a facet \(F_i\), it does so along a closed linear segment with velocity \(p_i\), entering and exiting at the boundary of \(F_i\) after finite time.
\end{lemma}
\begin{proof}
Immediate from the definition of \(p_i\) on facet interiors and the Reeb/Hamiltonian inclusion.
\end{proof}

At lower-dimensional faces the behavior depends on geometry.

\begin{lemma}[Flow on Lagrangian 2-faces]\label{lem-lagrangian-2face}
If a 2-face \(F_{ij}=F_i\cap F_j\) is Lagrangian, then \(p_i\) and \(p_j\) lie in its tangent plane. Locally the orbit may slide along \(F_{ij}\) with any \(W^{1,2}\) velocity in \(\operatorname{conv}\{p_i,p_j\}\). It enters and exits \(F_{ij}\) after finite time through its boundary (a 1- or 0-face).
\end{lemma}
\begin{proof}
Since \(p_i,p_j\) are tangent, the orbit cannot enter from facet interiors; it passes through the boundary. Also \(0\notin \operatorname{conv}\{p_i,p_j\}\), so \(\alpha\) integrates to a potential on \(F_{ij}\) and the orbit cannot stay forever.
\end{proof}

\begin{lemma}[Flow through non-Lagrangian 2-faces]\label{lem-nonlagrangian-2face}
If \(F_{ij}\) is non-Lagrangian, then the orbit crosses \(F_{ij}\) from one facet to the other at isolated times. The direction is determined by \(\omega(n_i,n_j)\): if \(\omega(n_i,n_j) > 0\) the orbit crosses from \(F_i\) to \(F_j\); if \(\omega(n_i,n_j)<0\) it crosses from \(F_j\) to \(F_i\).
\end{lemma}
\begin{proof}
We use \(\inner{J n_i}{n_j} = - \inner{J n_j}{n_i} = \omega(n_i,n_j)\). Its sign determines whether \(p_i\) points into or out of the half-space defined by \(F_j\). Locally the orbit consists of linear segments with velocities \(p_i\) and \(p_j\); touching times are isolated and the crossing direction follows the sign.
\end{proof}

\begin{lemma}[Flow on 1-faces]\label{lem-1face-flow}
If \(\gamma\) meets the interior of a 1-face \(F = \bigcap_{k=1}^m F_{i_k}\) (with \(m\ge3\)), there is a unique direction of flow along \(F\). Velocities may vary within the convex cone of incident \(p_{i_k}\). The orbit may enter or exit \(F\) through adjacent 0-, 2-, or 3-faces; the touching time is finite.
\end{lemma}
\begin{proof}
Convexity shows the normals do not convexly combine to zero; neither do the \(p_{i_k}\). Their convex cone lies on a unique half-line, giving the direction. Examples show trajectories can enter/exit as stated.
\end{proof}

\begin{lemma}[Flow on 0-faces]\label{lem-0face-flow}
If \(\gamma\) meets a 0-face where facets \(F_{i_1},\ldots,F_{i_m}\) meet, the orbit crosses through instantaneously. The orbit may enter from or exit to any of the incident facets, 2-faces, or 1-faces.
\end{lemma}
\begin{proof}
  Again we can see from convexity of \(K\) that the \(p_{i_k}\) do not convexly combine to zero, so the orbit cannot stay at the 0-face. Examples show all stated transitions can occur.
\end{proof}

\subsection{Generic behavior}
Some of the above behavior is known to be or conjectured to be non-generic. We briefly state our knowledge here, but won't use it in the rest of the thesis. It justifies why \cite{CH2021} focused on the non-Lagrangian generic case.
We will use the term "generic" formally, i.e. dense and open subset of configuration space.

\begin{lemma}[Non-Lagrangianness is generic]\label{lem-nonlagrangian-generic}
For a fixed number of facets (or vertices), polytopes with no Lagrangian 2-faces form an open dense set in the parameter space of facet normals/heights (respectively vertices).
\end{lemma}
\begin{proof}
Lagrangianness of \(F_{ij}\) is the single equation \(\omega(n_i,n_j)=0\), a codimension-one condition. A finite union of such subsets has complement open dense.
\end{proof}

\begin{conjecture}\label{conj-generic-0faces}
For a generic polytope, no closed characteristic passes through a 0-face.
\end{conjecture}
\begin{remark}
This conjecture appears in \cite{CH2021} (Conjecture~1.26): “We expect that Type 2 combinatorial Reeb orbits do not exist for generic polytopes.”
\end{remark}

\subsection{Homotopies of orbits}
Finally, we want to discuss how we can go from \(W^{1,2}\) orbits to simpler combinatorial representatives, using homotopies through orbits. For our algorithm we want to use such combinatorial representatives, instead of working with general \(W^{1,2}\) orbits. Intuitively that makes sense, since \(W^{1,2}\) behavior on flat polytope faces is not that interesting.
We will use the "action" of orbits, which is defined only one section later, all that matters is that the action is preserved along the homotopies we find.

\begin{definition}[Polygonal orbit]\label{def-piecewise-constant-velocity}
A Hamiltonian/Reeb orbit is \emph{polygonal} if time can be partitioned into finitely many open intervals such that during each interval the incident facet set is constant and the velocity is a single incident \(p_i\). Breakpoints need not coincide with face changes.
\end{definition}

\begin{theorem}[Homotopy to polygonal orbit]\label{thm-homotopy-pl}
Any Hamiltonian/Reeb orbit \(\gamma\) is homotopic through Hamiltonian/Reeb orbits of the same action to a polygonal orbit \(\gamma'\).
\end{theorem}
\begin{proof}
The face incidence changes only at isolated times, yielding a finite partition. On 3-faces the velocity is constant. On Lagrangian 2-faces, and on 1-faces, the velocity may vary as an arbitrary \(W^{1,2}\) path with velocities in \(\operatorname{conv}\{p_i: F_i \text{ incident}\}\). The start and end point on the face define a convex set of possible paths. The codomain of the paths is a convex subset of the face. Thus, polygonal paths lie dense in the set of possible paths, and so at least one lies in the set, which we can homotope to. We cannot directly define a decomposition into segments with one segment for each velocity here since the path may leave the face in that case; but we can find at least some finite polygonal approximation that lies within the face, and that we can linearly homotope to. The action of the path depends only on the endpoints since the codomain is flat, so the action is preserved along the homotopy.
\end{proof}

\begin{definition}[Simple orbit]\label{def-simple-orbit-2}
A polygonal Hamiltonian/Reeb orbit is \emph{simple} if each facet velocity \(p_i\) is used at most once.
\end{definition}

\begin{theorem}[Homotopy to simple orbit]\label{thm-min-action-simple}
Any Hamiltonian/Reeb orbit is homotopic through Hamiltonian/Reeb orbits to a simple orbit whose action is non-increasing along the homotopy.
\end{theorem}
\begin{proof}
The argument in \cite{HK2017} uses Clarke's dual principle, so we defer to the next chapter for our recounting of the proof.
\end{proof}

\begin{corollary}[Simple minimum-action orbit]\label{cor-simple-min-action}
There exists a minimum-action Hamiltonian/Reeb orbit that is simple.
\end{corollary}

\begin{example}[Viterbo counterexample degeneracy]\label{ex-viterbo-degeneracy}
The counterexample from \cite{HK2024} has multiple distinct minimum-action closed characteristics that are all homotopic.
\end{example}
