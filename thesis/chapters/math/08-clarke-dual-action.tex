% !TeX root = ../../main.tex
\section{Clarke's dual action principle}
We will below always consider the cases of a bounded strictly convex body with smooth boundary, or a polytope. We often skip explicitly tracking the inclusion relations for polytope dynamics, and instead focus on equality on strictly convex smooth bodies; the polytope case then works algebraically analogously.

\begin{theorem}[Fenchel duality for gauge and support functions]\label{thm-fenchel-duality}
Recall the gauge and support functions:
\[
  \gauge_K(x) = \min\{r>0: x \in rK\},\qquad
  \support_K(y) = \max_{x\in K} \inner{x}{y}.
\]
They are dual norms on \(\R^4\):
\[
  \gauge_K^2(x) = \sup_{y \in \R^4} \big( \inner{x}{y} - \tfrac14 \support_K(y)^2 \big),\qquad
  \tfrac14 \support_K^2(y) = \sup_{x \in \R^4} \big( \inner{x}{y} - \gauge_K(x)^2 \big).
\]
Thus
\[
  \gauge_K^2(x) + \tfrac14 \support_K^2(y) \ge \inner{x}{y},
\]
with equality iff \(y\in\partial \gauge_K^2(x)\) iff \(x\in\partial(\tfrac14 \support_K^2)(y)\).
In the polytope case this identifies supporting facets and contact points by Legendre duality:
\[
  \partial \gauge_K^2(x \in F) = \operatorname{conv}\Big\{ \frac{2}{h_i} n_i : x \in F_i \Big\},
\]
\[
  \partial (\tfrac14 \support_K^2)(y \in \dots) = F
\]
\end{theorem}
\begin{proof}
  Calculation.
\end{proof}

\paragraph{Clarke's dual action principle.}

To set up Clarke's dual action principle, we work in the Sobolev space \(W^{1,2}([0,T],\R^4)\) of absolutely continuous curves with square-integrable weak derivative and period \(T>0\).
We consider the following functionals
\begin{align*}
  A(z) &= \tfrac12 \int_0^T \inner{-J\dot z(t)}{z(t)}\,dt, \\
  I_g(z) &= \frac{1}{T} \int_0^T \gauge_K^2(-J\dot z(t))\,dt, \\
  I_h(z) &= T \int_0^T \tfrac14 \support_K^2(-J\dot z(t))\,dt, \\
\end{align*}
The functionals are all quadratic in spatial scaling, constant in time scaling. The functionals \(A,I_h\) are constant under spatial translation.

An interesting observation is that the Hamiltonian orbits on \(\R^4\) now correspond to the critical points of \(A - T I_g\). This can also be understood as \(T\) serving as a Lagrange multiplier, yielding critical points of \(A\) constrained to level sets of \(\gauge_K\). Intuitively, this again tells us that Hamiltonian orbits and Reeb orbits correspond.

One problem we have with this formulation is that \(I_g\) does not give us much freedom to homotope orbits. On the other hand \(I_h\) only depends on how much time is spent on each velocity, so even full rearrangement of velocities is possible without changing \(I_h\). In this sense, \(I_h\) is a useful framework to find new homotopies of orbits, that \(I_g\) does not show us.

\begin{theorem}[Clarke's dual action principle]\label{thm-clarke-dual}
Critical points of \(A - T I_g\) correspond to Hamiltonian orbits in \(\R^4\). The lagrange multiplier \(T\) can be used to fix the level set. Critical points of \(A - T I_h\) correspond to Hamiltonian orbits as well, but with a potential time and spatial scaling. The lagrange multiplier \(T\) can be used to fix the level set.
\end{theorem}


\subsection{Primal and dual optimization problems}

The primal problem (finding minimum-action closed characteristics) and the dual problem differ in their constraints:

\begin{center}
\begin{tabular}{@{}p{0.15\textwidth}p{0.38\textwidth}p{0.38\textwidth}@{}}
  \toprule
  & \textbf{Primal Problem} & \textbf{Dual Problem} \\
  \midrule
  Minimize & \(A(\gamma) = \int_\gamma \lambda\) & \(I_K(z) = \tfrac14 \int_0^T \support_K^2(-J\dot z)\,dt\) \\
  Space & \(\gamma \in W^{1,2}([0,T],\R^4)\) & \(z \in W^{1,2}([0,T],\R^4)\) \\
  \midrule
  Constraints & \(\int_0^T \dot\gamma = 0\) & \(\int_0^T \dot z = 0\) \\
              & \(\gauge_K^2(\gamma) \equiv 1\) & \(\int_0^T \inner{-J\dot z}{z}\,dt = 2T\) \\
              & \(\dot\gamma \in J \partial \gauge_K^2(\gamma)\) a.e. & (no differential inclusion) \\
              & & \(\int_0^T z\,dt = 0\) \\
  \bottomrule
\end{tabular}
\end{center}

The dual problem is easier for several reasons:
\begin{itemize}
  \item The minimization target \(I_K\) depends only on the velocity \(\dot z\), not on position.
  \item The function space has no direct position constraint (no \(\gamma(t)\in\partial K\)).
  \item The function space has no differential inclusion constraint.
\end{itemize}
These properties allow for homotopies that rearrange velocity segments, which is the key to proving existence of simple minimizers.

\begin{proof}[Proof of \cref{thm-clarke-dual}]
We show that critical points of the primal and dual problems correspond.

From variational calculus, critical points of \(A\) subject to \(\gauge_K^2(\gamma)\equiv1\) satisfy
\[
  -J\dot\gamma(t) \in \partial \gauge_K^2(\gamma(t)) \text{ a.e.}
\]
Critical points of \(I_K\) under the dual constraints satisfy
\[
  z(t) + c \in \partial (\tfrac14 \support_K^2)(-J\dot z(t)) \text{ a.e.}
\]
for some constant \(c\in\R^4\).

By \cref{thm-fenchel-duality}, these differential inclusions are equivalent under \(z = \gamma - \bar\gamma\), where \(\bar\gamma = \frac{1}{T}\int_0^T \gamma\,dt\) is the center of mass. The constraint correspondence follows.

Finally, integrating the Fenchel equality over time:
\[
  T + I_K(z) = 2 A(\gamma) = 2T \implies I_K(z) = T = A(\gamma). \qedhere
\]
\end{proof}

Clarke's dual action principle (specialized to convex bodies) considers
\[
  E = \Big\{ z\in W^{1,2}([0,1],\R^4): \int_0^1 \dot z = 0,\ \int_0^1 \inner{-J\dot z}{z}\,dt = 1 \Big\},\qquad
  I_K(z)=\tfrac14\int_0^1 \support_K^2(-J\dot z(t))\,dt.
\]
Critical points of \(I_K\) correspond to generalized characteristics and
\[
  \cEHZ(K) = \inf_{z\in E} I_K(z).
\]
For polytopes \(\support_K\) is piecewise linear, so \(I_K\) is piecewise quadratic in \(\dot z\); minimizers can be taken piecewise affine, leading to the combinatorial model below \cite{HK2017}.

\subsection{Existence of simple minimizers}

We now prove \cref{thm-min-action-simple} from the previous section: any orbit is homotopic to a simple orbit with non-increasing action.

\begin{theorem}[Haim--Kislev 2019]\label{thm-simple-minimizer-existence}
Let \(K\subset\R^{2n}\) be a convex polytope. There exists a generalized closed characteristic \(\gamma^*\) with minimal action \(\cEHZ(K)\) such that:
\begin{enumerate}
  \item \(\gamma^*\) is piecewise affine (breakpoints need not be on facet intersections).
  \item Velocities \(\dot\gamma^*(t)\) are pure facet Reeb vectors, not convex combinations.
  \item Each facet Reeb vector \(p_i\) appears at most once.
\end{enumerate}
\end{theorem}

\begin{proof}[Proof outline]
We work in the dual formulation and perform a sequence of transformations on any minimizer \(z\), each preserving or improving the action.

\textbf{Step 1 (Approximate).} Replace \(z\) by piecewise affine \(z_N\) in \(W^{1,2}\). Ensure \(\dot z_N(t)\) stays in the allowed cone. By continuity, \(A(z_N)\to A(z)\) and \(I_K(z_N)\to I_K(z)\).

\textbf{Step 2 (Split).} Replace mixed velocities \(\dot z_N(t)\in\operatorname{conv}\{p_{i_1},\ldots,p_{i_k}\}\) by concatenations of segments with pure velocities. Choose the time order so that the action does not decrease:
\[
  A(z_N') \ge A(z_N).
\]
This is possible because reversing the order reverses the sign of the area change. For pure paths, \(I_K(z_N') = T\).

\textbf{Step 3 (Rearrange).} If some \(p_i\) appears in disjoint time intervals, merge them by a ``grow+shrink'' rearrangement. Choose the direction so that \(A(z_N'')\ge A(z_N')\). The functional \(I_K\) is unchanged since it depends only on velocity magnitudes and times.

\textbf{Step 4 (Renormalize).} Scale time to restore the action constraint. If \(A(z_N'') > A(z_N)\), scale by factor \(\beta = T/A(z_N'')\) so that \(A(z_N''') = T\) and \(I_K(z_N''') = \beta T\).

\textbf{Step 5 (Compactness).} A simple loop is encoded by finite data: a permutation \(\sigma\) (facet order) and segment lengths \((|I_i|)\). The space of such encodings satisfying the constraints is compact. Taking a subsequential limit yields a simple minimizer with \(I_K(z^*) = T = A(\gamma^*) = \cEHZ(K)\).
\end{proof}

\begin{remark}
Each modification step (approximation, splitting, grow+shrink, rescaling) can be done as a homotopy through minimizers of the dual problem. Thus every minimizer is homotopic to a simple minimizer.
\end{remark}
