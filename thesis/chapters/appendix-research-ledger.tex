% !TeX root = ../main.tex

% Research Ledger Appendix
% Reader-facing historical record of research directions explored.
% Agent tracking now lives in experiments.md and the experiment-workflow skill.

\section*{Purpose}

This appendix documents the evolution of research questions explored during this thesis.
It provides transparency about what was investigated, what succeeded, and what was abandoned.

\section{Completed Research}

\subsection{EHZ Capacity of the Unit Cross-Polytope}
\label{ledger:cross-polytope-capacity}

\textbf{Research question:} What is the EHZ capacity of the 4-dimensional unit cross-polytope (16-cell)?

\textbf{Approach:} Apply the tube algorithm from Section~\ref{sec:ch2021-algorithm} to compute $\cEHZ(K)$ for the unit cross-polytope $K = \{z \in \R^4 : |z_1| + |z_2| + |z_3| + |z_4| \le 1\}$.

\textbf{Result:} The algorithm finds $\cEHZ(\text{cross-polytope}) = 1.0$ (to numerical precision $\epsilon \approx 10^{-2}$).

\textbf{Significance:} This value was previously unknown in the literature. Combined with the known $\cEHZ(\text{tesseract}) = 4$, this verifies the Mahler-type bound $\cEHZ(K) \cdot \cEHZ(K^\circ) = 4$ for the dual pair (cross-polytope, tesseract).

\textbf{Status:} Completed. See \texttt{packages/rust\_viterbo/tube/} for implementation.

\section{Planned Research}

\subsection{Billiard Algorithm on HK--O Counterexample}
\label{ledger:billiard-hko-orbit}

\textbf{Research question:} Can we use the billiard algorithm to compute the systolic ratio for the HK--O 2024 counterexample and visualize a minimum-action orbit?

\textbf{Approach:} Apply the billiard algorithm from~\cite{HK2024} to compute $\cEHZ(K)$. Plot one minimum-action orbit by projecting onto either factor of the Lagrangian product $P_5 \times_L R(90^\circ) P_5$.

\textbf{Status:} Ideation.

\section{Abandoned Research}

% Document research directions that were explored but did not yield useful results.
% This saves future researchers time and provides transparency.

(None yet.)

% [proposed]
