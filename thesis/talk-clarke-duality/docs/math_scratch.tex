% math_scratch.tex
% -----------------------------------------------------------------------------
% PURPOSE
% - Scratch / notes-only LaTeX file for the Clarke duality talk.
% - Not intended to be compiled or rendered.
% - Collects the exact math objects we need: definitions, constants, theorem statements,
%   and a proof skeleton with "no mistakes" conventions.
%
% SOURCES CONSULTED (local repo cache):
% - HK2017 (arXiv:1712.03494v3): /workspaces/worktrees/shared/arxiv-store/1712.03494v3/EHZ-polytopes.tex
% - Thesis draft: packages/latex_viterbo/chapters/math/06-action-capacity-systolic.tex
%               packages/latex_viterbo/chapters/math/08-clarke-dual-action.tex
%               packages/latex_viterbo/chapters/math/09-hk-ch-formula.tex
%
% Every place where I am uncertain is marked with:
%   %TODO(Joern): ...
% -----------------------------------------------------------------------------
\documentclass[12pt]{article}

\usepackage{amsmath,amssymb,amsthm,mathtools}

% Minimal macros (explicit, no “semantic magic”)
\newcommand{\R}{\mathbb{R}}
\newcommand{\T}{\mathbb{T}}
\newcommand{\inner}[2]{\langle #1,#2\rangle}
\newcommand{\abs}[1]{\lvert #1\rvert}
\newcommand{\conv}{\operatorname{conv}}

\newcommand{\cEHZ}{c_{\mathrm{EHZ}}}
\newcommand{\support}{h} % support function
\newcommand{\gauge}{g}   % gauge / Minkowski functional
\newcommand{\polar}{\circ}

\newtheorem{definition}{Definition}
\newtheorem{theorem}{Theorem}
\newtheorem{lemma}{Lemma}
\newtheorem{proposition}{Proposition}
\newtheorem{remark}{Remark}

\begin{document}

\section*{0. Setup: standard symplectic data}

We work in \(\R^{2n}\) with the standard Euclidean inner product \(\inner{\cdot}{\cdot}\).
Let \(J\) be the standard complex structure (matrix) and define the standard symplectic form
\[
  \omega(u,v) := \inner{Ju}{v}.
\]
The standard Liouville 1-form is
\[
  \lambda := \tfrac12 \inner{Jx}{dx}.
\]
For a (piecewise \(W^{1,2}\)) loop \(\gamma:[0,T]\to\R^{2n}\), define the action
\[
  A(\gamma) := \int_\gamma \lambda
  = \tfrac12 \int_0^T \inner{J\gamma(t)}{\dot\gamma(t)}\,dt
  = \tfrac12 \int_0^T \inner{-J\dot\gamma(t)}{\gamma(t)}\,dt.
\]

%TODO(Joern): Confirm we keep exactly this \lambda and action normalization on slides.
%             (This matches HK2017 Eqn in prelims: A(\gamma)=1/2 \int <J\gamma,\dot\gamma>.)

\section*{1. Convex-analytic dictionary (support, gauge, polar)}

Let \(K\subset\R^{2n}\) be a convex body (compact, convex, non-empty interior).

\begin{definition}[Support and gauge]
The \emph{support function} of \(K\) is
\[
  \support_K(y) := \sup_{x\in K} \inner{x}{y}.
\]
The \emph{gauge} (Minkowski functional) is
\[
  \gauge_K(x) := \inf\{r>0:\ x\in rK\}.
\]
\end{definition}

The \emph{polar body} is
\[
  K^{\polar} := \{y\in\R^{2n}:\ \support_K(y)\le 1\}.
\]

\begin{theorem}[Fenchel duality: \(\gauge_K^2\) vs \(\tfrac14\support_K^2\) (thesis)]
One has
\[
  \gauge_K^2(x) = \sup_{y\in\R^{2n}}\big(\inner{x}{y} - \tfrac14 \support_K(y)^2\big),
  \qquad
  \tfrac14 \support_K^2(y) = \sup_{x\in\R^{2n}}\big(\inner{x}{y} - \gauge_K(x)^2\big),
\]
and therefore the pointwise inequality
\[
  \gauge_K^2(x) + \tfrac14 \support_K^2(y) \ge \inner{x}{y},
\]
with equality iff \(y\in\partial \gauge_K^2(x)\) iff \(x\in\partial(\tfrac14 \support_K^2)(y)\).
\end{theorem}

%TODO(Joern): In the talk, do you want to include the full duality identities (sup formulas),
%             or only the inequality + equality characterization?
%             (Your questionnaire suggests: definitions + geometric meaning, with subgradients as “technical tool”.)

\section*{2. Closed characteristics and the EHZ capacity}

\subsection*{2.1 Smooth boundary: geometric definition (for intuition)}

If \(\Sigma\subset\R^{2n}\) is a smooth domain, the characteristic line field on \(\partial\Sigma\)
is \(\ker(\omega|_{T\partial\Sigma})\). A \emph{closed characteristic} is a closed embedded curve
\(\gamma\subset \partial\Sigma\) tangent to that line field.

Equivalently (in the smooth strictly convex case), closed characteristics are Reeb orbits for the
contact form \(\alpha=\lambda|_{\partial\Sigma}\), up to \emph{positive} reparametrization.

\begin{definition}[Reeb orbits (talk-primary object)]
Given a contact form \(\alpha\) on \(\partial\Sigma\), the Reeb vector field \(R\) is defined by
\[
  \iota_R d\alpha = 0,
  \qquad
  \alpha(R)=1.
\]
A \emph{Reeb orbit} is a \(T\)-periodic solution \(\gamma:[0,T]\to\partial\Sigma\) of
\(\dot\gamma(t)=R(\gamma(t))\).
\end{definition}

\begin{remark}[Period \(=\) action for Reeb parameterization]
For a Reeb orbit \(\gamma\) with \(\dot\gamma=R\), we have \(\lambda(\dot\gamma)=\alpha(R)=1\) and therefore
\[
  A(\gamma)=\int_0^T \alpha(\dot\gamma)\,dt = \int_0^T 1\,dt = T.
\]
\end{remark}

%TODO(Joern): The talk will include a slide comparing:
%   - closed characteristics (unparameterized),
%   - Reeb orbits (parameterized, period = action),
%   - Hamiltonian orbits (parameterized, for H = g_K^2, with inclusions on polytopes).
% Please confirm the exact phrasing you want on that comparison slide.

\subsection*{2.2 Polytopes: generalized closed characteristics (HK2017)}

Assume \(K\subset\R^{2n}\) is a convex polytope with non-empty interior.
Let its \((2n-1)\)-dimensional facets be \(\{F_i\}_{i=1}^{F}\), with \emph{unit outward} normals \(n_i\).
Define the outward normal cone at \(x\in\partial K\):
\[
  N_K(x) := \R_{+}\,\conv\{n_i:\ x\in F_i\}.
\]

\begin{definition}[Closed characteristic on a polytope (HK2017 Def.\ 2.1)]
A \emph{closed characteristic} on \(\partial K\) is a closed loop
\(\gamma\in W^{1,2}([0,T],\R^{2n})\) (for some \(T>0\)) such that \(\mathrm{Im}(\gamma)\subset\partial K\) and
\[
  \dot\gamma(t)\in J\,N_K(\gamma(t))
  \quad\text{for a.e. }t\in[0,T].
\]
\end{definition}

\begin{remark}
The inclusion \(\dot\gamma(t)\in JN_K(\gamma(t))\) is invariant under positive reparametrization
of time (``unparameterized''). Scaling the generators of the cone by positive constants does not
change the cone.
\end{remark}

\begin{definition}[EHZ capacity (HK2017, smooth case; extended by continuity)]
For a smooth convex body \(K\subset\R^{2n}\),
\[
  \cEHZ(K) := \min\{A(\gamma): \gamma \text{ closed characteristic on }\partial K\}.
\]
For non-smooth convex bodies (including polytopes) this is extended by continuity; it can still be
described as a minimum action over generalized closed characteristics.
\end{definition}

%TODO(Joern): Insert the exact theorem/reference you want for:
%   - “EHZ extends continuously under Hausdorff convergence”
%   - “generalized closed characteristics = limits of smooth ones (e.g. smoothings K_\varepsilon)”
% I can currently cite HK2017’s statement “see singular_capacity / singular_survey_2”, but I don’t
% have the exact theorem name/number in this repo cache.

\section*{3. Clarke dual action principle (talk conventions + HK2017 reference)}

\subsection*{3.1 Dual functional (fixed) and constraint sets (two equivalent normalizations)}

We follow the HK2017 setup, but we will \emph{present} it with a free period \(T\) and an explicit centering condition,
because that makes “period/action bookkeeping” easier in a live talk.

\paragraph{Dual functional (any period \(T\)).}
For a loop \(z\in W^{1,2}([0,T],\R^{2n})\), define
\[
  I_K(z) := \frac14 \int_0^T \support_K^2\!\bigl(-J\dot z(t)\bigr)\,dt.
\]
It depends only on the velocity \(\dot z\), and is invariant under translation \(z\mapsto z+b\).

\paragraph{Talk normalization (preferred).}
For \(T>0\), define
\[
  \mathcal{E}_T^{\mathrm{talk}}
  :=
  \Big\{
    z\in W^{1,2}([0,T],\R^{2n}):
    z(0)=z(T),\
    \int_0^T z(t)\,dt=0,\
    A(z)=T
  \Big\}.
\]
Here \(A(z)=\tfrac12\int_0^T \inner{-J\dot z(t)}{z(t)}\,dt\) is the action.

Intuition: the constraint \(A(z)=T\) fixes a \emph{Reeb-like} parameterization (period \(=\) action),
and \(\int z=0\) fixes the translation ambiguity of the Euler--Lagrange equation.

%TODO(Joern): You wrote “centering avoids trouble with uniqueness of z given \dot z, and helps variational arguments”.
%             Please confirm: do you want centering as part of the official statement, or as a “we can assume wlog” remark?

\paragraph{HK2017 normalization (reference).}
HK2017 instead fixes the time interval \([0,1]\) and uses the constraint
\[
  \mathcal{E}^{\mathrm{HK}}
  :=
  \Big\{
    z\in W^{1,2}([0,1],\R^{2n}):
    \int_0^1 \dot z(t)\,dt=0,\
    \int_0^1 \inner{-J\dot z(t)}{z(t)}\,dt = 1
  \Big\}.
\]
No centering constraint is imposed there (translation invariance is handled implicitly).

%TODO(Joern): In the talk you said you “dislike the domain=[0,1] normalization”.
%             We can still cite HK2017 in this form, but present (and compute with) the \mathcal{E}^{talk}_T variant.
%             Confirm this split is acceptable.

\subsection*{3.2 Weak critical points and correspondence}

HK2017 defines the set of weak critical points
\[
  \mathcal{E}^{\dagger}
  :=
  \Big\{
    z\in\mathcal{E}^{\mathrm{HK}}:\ \exists \alpha\in\R^{2n}\ \text{s.t.}\ 8 I_K(z)\,z + \alpha \in \partial \support_K^2(-J\dot z)
  \Big\}.
\]

\begin{lemma}[Dual bijection (HK2017, Lemma “dual\_bijection”; wording)]
There is a correspondence between closed characteristics \(\gamma\) on \(\partial K\) and
elements \(z\in\mathcal{E}^{\dagger}\). Under this correspondence there exist \(\lambda>0\) and \(b\in\R^{2n}\) such that
\[
  z = \lambda \gamma + b,
  \qquad
  A(\gamma) = 2\,I_K(z).
\]
In particular, minimizers of \(I_K\) on \(\mathcal{E}^{\mathrm{HK}}\) correspond to minimum-action closed characteristics.
\end{lemma}

\begin{theorem}[Capacity as a dual minimum (HK2017)]
\[
  \cEHZ(K) = \min_{z\in\mathcal{E}^{\mathrm{HK}}} 2\,I_K(z).
\]
\end{theorem}

%TODO(Joern): IMPORTANT consistency check with the current slide draft:
%   - In packages/latex_talk_clarke_duality/main.tex, the “Clarke dual action principle” slide currently says
%       c_EHZ(K) = min_{z\in E} I_K(z)
%     but later slides use A(\gamma)=2 I_K(z).
%   - HK2017 uses c_EHZ(K) = min_{z\in \mathcal{E}^{HK}} 2 I_K(z).
% Your follow-up questionnaire says you want talk-specific conventions (free period T, no [0,1] normalization),
% but still contradiction-free. We should update slides to match one convention set.

\subsection*{3.2.1 Talk-friendly normalization: put the minimizer on \([0,T]\) with \(A=I_K=T\)}

Your follow-up preference is:
\begin{itemize}
\item avoid the fixed \([0,1]\) domain on slides,
\item avoid HK-style “extra constant \(c\) everywhere” bookkeeping,
\item keep a single symbol \(T\) that is both “the number we minimize” and “a period/action scale”.
\end{itemize}

One clean way to achieve this while still relying on HK2017 is:

\begin{enumerate}
\item Start with a minimizer \(z_0\in\mathcal{E}^{\mathrm{HK}}\) of \(I_K\) on \([0,1]\).
\item Define \(T := 2\,I_K(z_0)\). Then \(T=\cEHZ(K)\) by the theorem above.
\item Define a rescaled loop \(z_T:[0,T]\to\R^{2n}\) by
  \[
    z_T(t) := \sqrt{2T}\,z_0\!\left(\frac{t}{T}\right).
  \]
  (Optionally, translate to enforce centering: replace \(z_T\) by \(z_T-\frac1T\int_0^T z_T\).)
\end{enumerate}

Then the two key identities are:
\[
  A(z_T)=T,
  \qquad
  I_K(z_T)=T.
\]

Sketch check:
\begin{itemize}
\item The “area constraint” integral is reparametrization invariant and scales quadratically in space:
  \[
    2A(z_T)=\int_0^T \inner{-J\dot z_T}{z_T}\,dt
    = (2T)\int_0^1 \inner{-J\dot z_0}{z_0}\,ds
    = 2T.
  \]
\item For \(I_K\), use 2-homogeneity of \(\support_K^2\) and the derivative scaling \(\dot z_T=(\sqrt{2T}/T)\dot z_0\):
  \[
    I_K(z_T)
    = \frac14\int_0^T \support_K^2\!\Bigl(-J\frac{\sqrt{2T}}{T}\dot z_0\Bigr)\,dt
    = \frac14\cdot\frac{2T}{T}\cdot T\int_0^1 \support_K^2(-J\dot z_0)\,ds
    = 2I_K(z_0)
    = T.
  \]
\end{itemize}

%TODO(Joern): This normalization is my proposed “talk-specific convention set”.
%             Please sanity-check the scaling formulas above (especially the I_K scaling) and confirm you like this.

\subsection*{3.3 (Optional) Abbondandolo-style formulation via the polar body (HK2017 Remark)}

HK2017 also quotes an alternative expression (attributed to Abbondandolo, using Clarke duality):
\[
  \cEHZ(K)
  =
  \frac12\left[
    \sup_{z\in\widetilde{\mathcal{E}}} \int_0^1 \inner{-J\dot z(t)}{z(t)}\,dt
  \right]^{-1},
\]
where
\[
  \widetilde{\mathcal{E}}
  :=
  \Big\{
    z\in W^{1,2}([0,1],\R^{2n}):
    \int_0^1 \dot z(t)\,dt = 0,\ 
    \dot z(t)\in K^{\polar}\ \text{a.e.}
  \Big\},
  \qquad
  K^{\polar}=\{y:\ \inner{x}{y}\le 1\ \forall x\in K\}.
\]

%TODO(Joern): Decide if we want to mention this \widetilde{\mathcal{E}} formulation at all.
%             It is a nice intermediate viewpoint (“polar constraint on velocities”), but it may be a distraction.

\subsection*{3.4 Proof skeleton of Clarke dual action principle (high level)}

This is \emph{not} a complete proof; it is the minimum set of inequalities/equalities we need to not get lost
when presenting the “gears level” idea.

\begin{itemize}
\item Start from Fenchel inequality (pointwise in \(\R^{2n}\)):
  \[
    \gauge_K^2(x) + \tfrac14 \support_K^2(y) \ge \inner{x}{y}.
  \]
\item Plug \(x=z(t)\), \(y=-J\dot z(t)\) and integrate:
  \[
    \int_0^T \gauge_K^2(z(t))\,dt + \frac14\int_0^T \support_K^2(-J\dot z(t))\,dt
    \ \ge\
    \int_0^T \inner{z(t)}{-J\dot z(t)}\,dt.
  \]
  The RHS is \(2A(z)\) (for loops, modulo constant conventions).
\item In HK2017 one fixes the scale by the constraint \(\int_0^1 \inner{-J\dot z}{z}=1\), i.e. \(z\in\mathcal{E}^{\mathrm{HK}}\).
\item In the talk-preferred normalization, we instead reparameterize/scale so that \(A(z)=T\) and (for the minimizer) also \(I_K(z)=T\)
  as in Section 3.2.1.
\end{itemize}

%TODO(Joern): This proof skeleton is intentionally vague. If you really want a 10–15min proof on slides (M5.5),
%             we should lock down:
%   - which functional is minimized on which constraint set (and with which scaling invariances),
%   - where exactly the Euler–Lagrange/subgradient inclusion appears,
%   - and the precise statement we are proving (HK2017 uses the correspondence via \mathcal{E}^\dagger).

\section*{4. Polytope specializations (facet normals, heights, velocities)}

Assume \(K=\bigcap_{i=1}^{F}\{x:\ \inner{x}{n_i}\le h_i\}\) is an irredundant half-space description,
with \(\abs{n_i}=1\) and \(h_i=\support_K(n_i)>0\).

\begin{definition}[Facet velocities (HK2017)]
Assume \(0\in K\) (translation invariance allows this).
For each facet \(F_i\), define
\[
  p_i := J\,\partial \gauge_K^2|_{F_i} = \frac{2}{h_i}\,J n_i.
\]
\end{definition}

%TODO(Joern): On slides, do we call p_i a “facet Reeb vector”, “facet velocity”, or “Hamiltonian velocity”?
%             It is a canonical generator for the cone, but depends on the chosen Hamiltonian normalization.

\begin{proposition}[Constant-speed property of \(I_K\) on faces (HK2017 Prop.\ 4.1)]
Let \(c>0\) and \(z\in\mathcal{E}^{\mathrm{HK}}\).
Assume that for almost every \(t\in[0,1]\) there exists a non-empty face
\(F_{j_1}\cap\cdots\cap F_{j_\ell}\neq\emptyset\) such that
\[
  \dot z(t) \in c\cdot \conv\{p_{j_1},\ldots,p_{j_\ell}\}.
\]
Then
\[
  I_K(z)=c^2.
\]
\end{proposition}

Proof idea (HK2017): compute \(h_K(-J\dot z(t))\) explicitly:
\[
  \dot z = c\sum a_i p_{j_i}
  \ \Rightarrow\
  -J\dot z = 2c \sum \frac{a_i}{h_{j_i}}\,n_{j_i}
  \ \Rightarrow\
  h_K(-J\dot z)=2c.
\]

\section*{5. The “simple minimizer / facet-once” theorem (HK2017 main structural result)}

\begin{theorem}[Existence of a simple minimum-action orbit (HK2017 Thm.\ 1.2; talk form)]
For every convex polytope \(K\subset\R^{2n}\), there exists a closed characteristic
\(\gamma:[0,T]\to\partial K\) with minimal action (for some \(T>0\)) such that \(\dot\gamma\) is piecewise constant:
there exist vectors \((w_1,\ldots,w_m)\) and \(0=\tau_0<\cdots<\tau_m=T\) with
\[
  \dot\gamma(t)=w_i\quad\text{for }\tau_{i-1}<t<\tau_i.
\]
Moreover:
\begin{itemize}
\item For each \(j\) there exists a facet index \(i\) with \(w_j=C_j J n_i\) for some \(C_j>0\).
\item For each facet \(i\), the set \(\{t:\ \exists C>0,\ \dot\gamma(t)=C J n_i\}\) is connected.
\end{itemize}
Hence \(\gamma\) visits the interior of each facet at most once (and \(\dot\gamma\) has \(\le F\) discontinuities).
\end{theorem}

\begin{remark}[About the choice of \(T\)]
The differential inclusion defining closed characteristics is invariant under constant time rescaling.
So one may state the theorem on \([0,1]\) as in HK2017, or on \([0,T]\) for any \(T>0\).
In particular, in the smooth Reeb setting one can choose the Reeb parameterization where \(T=A(\gamma)=\cEHZ(K)\).
\end{remark}

%TODO(Joern): If you want to emphasize “period = action” as a theme, we can consistently denote
%             \(T:=\cEHZ(K)\) after we have introduced the minimizer, and then work on \([0,T]\) thereafter.

\begin{remark}[Multiplicity/degeneracy]
HK2017 notes examples where there exist minimizing closed characteristics that \emph{do not} have this property
(e.g. polytopes with Lagrangian faces), but there still exists at least one minimizer with the facet-once property.
\end{remark}

%TODO(Joern): You used “homotopic to” language in the questionnaire.
%             HK2017 theorem statement is an existence statement.
%             Decide what strength we claim on slides: existence vs “one can simplify a minimizer”.

\section*{6. Combinatorial formula for \(\cEHZ\) (HK2017)}

Let \(K\subset\R^{2n}\) be a convex polytope with facets \(F_i\) and unit outward normals \(n_i\).
Let \(h_i := \support_K(n_i)\).
Let \(S_F\) be the symmetric group.
Define
\[
  M(K) := \Big\{(\beta_i)_{i=1}^{F}:\ \beta_i\ge 0,\ \sum_i \beta_i h_i = 1,\ \sum_i \beta_i n_i = 0\Big\}.
\]

\begin{theorem}[HK2017 combinatorial formula (HK2017 Thm.\ 1.1)]
\[
  \cEHZ(K)
  =
  \frac12\left[
    \max_{\sigma\in S_F,\ \beta\in M(K)}
      \sum_{1\le j<i\le F} \beta_{\sigma(i)}\beta_{\sigma(j)}\,\omega(n_{\sigma(i)},n_{\sigma(j)})
  \right]^{-1}.
\]
\end{theorem}

\begin{remark}[Where the \(\beta\)’s come from (HK2017 proof sketch)]
For a simple loop, one has segment lengths \(T_i=|I_i|\) and \(p_{\sigma(i)}=\frac{2}{h_{\sigma(i)}}Jn_{\sigma(i)}\).
Setting \(\beta_{\sigma(i)}:=\frac{T_i}{h_{\sigma(i)}}\) converts the “\(p_i\) / times” constraints into the formula above.
\end{remark}

%TODO(Joern): For the 4D talk, do you want to present the full formula, or only the “finite combinatorial search space”
%             interpretation (facet order + segment lengths + constraints)?

\section*{7. Proof ingredients (for talk-proof sketch)}

The HK2017 proof of the simple-loop theorem is built from:

\subsection*{7.1 Piecewise affine approximation}

\begin{lemma}[Approximation inside a convex hull (HK2017 Lem.\ 4.2)]
Fix vectors \(v_1,\ldots,v_k\in\R^{2n}\). If \(z\in W^{1,2}([0,1],\R^{2n})\) satisfies
\(\dot z(t)\in\conv\{v_1,\ldots,v_k\}\) a.e., then for every \(\varepsilon>0\) there exists a
piecewise affine \(\zeta\) with \(\|z-\zeta\|_{1,2}<\varepsilon\) and \(\dot\zeta(t)\in\conv\{v_1,\ldots,v_k\}\) a.e.
\end{lemma}

\subsection*{7.2 Action constraint for piecewise constant velocities}

If \(z\) is a closed loop with piecewise constant velocity
\[
  \dot z(t)=\sum_{i=1}^m \mathbf{1}_{I_i}(t)\,w_i
  \qquad
  (I_i=(\tau_{i-1},\tau_i)),
\]
then HK2017 proves the identity
\[
  \int_0^1 \inner{-J\dot z(t)}{z(t)}\,dt
  =
  \sum_{1\le j<i\le m} |I_i|\,|I_j|\,\omega(w_i,w_j).
\]

\subsection*{7.3 Two simplification operations}

1) \emph{No linear combinations} (split a convex combination velocity into pure velocities)
without decreasing \(\int\inner{-J\dot z}{z}\).

2) \emph{One-speed / connectivity} (merge disjoint intervals with the same velocity into one block),
choosing the order so that \(\int\inner{-J\dot z}{z}\) does not decrease.

\subsection*{7.4 Rescaling back into \(\mathcal{E}^{\mathrm{HK}}\) (HK2017 proof step)}

After rearrangements, the constraint \(\int_0^1 \inner{-J\dot z}{z}=1\) may become \(\ge 1\).
Then divide \(z\) by \(A:=\sqrt{\int_0^1 \inner{-J\dot z}{z}}\) to return to \(\mathcal{E}^{\mathrm{HK}}\).
By the constant-speed property, \(I_K(z)\) scales in the expected way, and minimality forces equality in the limit.

Remark: centering \(\int z=0\) can be enforced at any time by translating \(z\mapsto z-\frac1T\int z\),
and this does not change \(A\) or \(I_K\) for closed loops (because \(\int \dot z=0\)).

\section*{8. Minimal “punchline package” (for the final slide)}

\[
  \text{Primal: } \cEHZ(K)=\min_{\gamma\subset\partial K,\ \dot\gamma\in JN_K} A(\gamma)
  \quad\Longleftrightarrow\quad
  \text{Dual (HK2017): } \cEHZ(K)=\min_{z\in\mathcal{E}^{\mathrm{HK}}} 2 I_K(z).
\]
On a polytope, \(I_K\) only sees the \emph{set and amount} of facet velocities, not their order.
This enables rearrangement \(\Rightarrow\) existence of a minimum-action orbit visiting each facet at most once
\(\Rightarrow\) finite/combinatorial optimization.

%TODO(Joern): You answered F1 with “pipeline slide” (and a bibliography slide before it).
%             Confirm: we should \emph{not} restate the full theorem on the last slide, only the pipeline.

\end{document}
