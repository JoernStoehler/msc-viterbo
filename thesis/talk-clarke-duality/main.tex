% Clarke's duality principle — seminar talk slides (ephemeral)
%
% IMPORTANT (lifecycle):
% - This package is meant for fast iteration during talk prep.
% - The final "as presented" version should be an annotated git tag:
%     talk-clarke-duality-YYYY-MM-DD
% - After migrating content into the thesis, this package may be deleted from `main`.
%   In that case, use the tag/commit to retrieve the exact sources/assets.
% - Post-talk workflow: see CHECKLIST_TEARDOWN.md
%
% Build:
%   ./scripts/build.sh
%
\documentclass[10pt,aspectratio=169]{beamer}

% Theme (nice defaults, no custom tikz required in authoring)
\usetheme[progressbar=frametitle]{metropolis}

% Fonts + basic typesetting
\usepackage[T1]{fontenc}
\usepackage{lmodern}
\usepackage{microtype}

% Math
\usepackage{amsmath,amssymb,mathtools}

% Tables / layout helpers
\usepackage{booktabs}

% Figures
\usepackage{graphicx}

% Drafting helpers:
% - Red text marks items that need J\"orn to verify: \NeedsCheck{...}
% - Slide variants are separate frames, tagged in code with: %variant: ...
%   and visible in the PDF as a frame subtitle like "(Variant 5-B)".
% - Hand-drawn image placeholders use \DraftImage{<label>}{<height>}
%   (or \DraftImage[<width>]{<label>}{<height>} if a slide needs a forced width)
%   which renders a big empty box (J\"orn draws directly into the PDF).

% Links
% (beamer loads hyperref; we just tune it)
\hypersetup{
  colorlinks=true,
  linkcolor=blue,
  urlcolor=blue
}

% Local macros + theorem setup
% Local preamble for a short-lived talk (keep this small).

% Common math shorthands
\newcommand{\R}{\mathbb{R}}
\newcommand{\C}{\mathbb{C}}
\newcommand{\Z}{\mathbb{Z}}
\newcommand{\Q}{\mathbb{Q}}
\newcommand{\N}{\mathbb{N}}
\newcommand{\T}{\mathbb{T}}

\newcommand{\abs}[1]{\lvert #1\rvert}
\newcommand{\norm}[1]{\lVert #1\rVert}
\newcommand{\inner}[2]{\langle #1, #2 \rangle}
\newcommand{\conv}{\operatorname{conv}}

% Symplectic shorthand (match thesis notation)
\newcommand{\cEHZ}{c_{\mathrm{EHZ}}}
\newcommand{\gauge}{g}
\newcommand{\support}{h}
\newcommand{\polar}{\circ}

% Beamer theorem blocks (numbered)
\setbeamertemplate{theorems}[numbered]


% Build metadata (generated by scripts/gen-version.sh; gitignored)
\providecommand{\TalkGitCommit}{unknown}
\providecommand{\TalkBuildDate}{\today}
\IfFileExists{version.tex}{\input{version}}{}

%\setbeamertemplate{frame footer}{\scriptsize source: \texttt{\TalkGitCommit}}

\title{Clarke's Dual Action Principle}
\author{J\"orn St\"ohler}
\institute{Symplectic Geometry Seminar}
\date{\TalkBuildDate}%\quad{\scriptsize source: \texttt{\TalkGitCommit}}}

\begin{document}

\maketitle

\begin{frame}{Outline}
  \begin{enumerate}
    \item Symplectic Geometry Refresh
    \item Viterbo's Conjecture
    \item Generalized Closed Characteristics on Polytopes
    \item Break
    \item Clarke's Dual Action Principle
    \item Existence of Simple Minimizers
    \item Outlook
  \end{enumerate}
\end{frame}

\section{Symplectic Geometry Refresh}

\begin{frame}{Standard Setting}
  \begin{itemize}
    \item 4-dimensional symplectic vector space \(\R^4\), coordinates \(x=(q_1,q_2,p_1,p_2)\).
    \item Symplectic form
      \[
        \omega_0 = \sum_{i=1}^2 dq_i\wedge dp_i ,\quad
        \omega_0(u,v) = \inner{Ju}{v},
      \]
    \item Almost complex structure 
      \[
        J = \begin{pmatrix}
          0 & -I_2 \\
          I_2 & 0
        \end{pmatrix} ,\quad
        J^2 = -I_4 ,\quad
        J((q,p)) = (-p,q).
      \]
    \item Liouville 1-form
      \[
        \lambda_0 = \tfrac12\inner{Jx}{dx} = \tfrac12\sum_{i=1}^2 (p_i\,dq_i - q_i\,dp_i) ,\quad
        d\lambda_0 = \omega_0.
      \]
    \item Higher dimensions \(\R^{2n}\) work analogously.
  \end{itemize}
\end{frame}

\begin{frame}{Loops, Action, Hamiltonians}
  \begin{itemize}
    \item Hamiltonian function \(H:\R^4\to\R\).
    \item Hamiltonian vector field \(X_H = J\nabla H\).
    \item Hamiltonian orbit: periodic solution \(\gamma:[0,T]\to\R^4\) of
      \[
        \dot\gamma(t) = X_H(\gamma(t)) = J\nabla H(\gamma(t)).
      \]
    \item Preserves energy levels: \(H(\gamma(t))\equiv \mathrm{const}\).
    \item Symplectic action of a loop \(\gamma:[0,T]\to\R^4\):
      \[
        A(\gamma) = \int_\gamma \lambda_0 = \int_0^T \lambda_0(\dot\gamma(t))\,dt = \tfrac12\int_0^T \inner{-J\dot\gamma(t)}{\gamma(t)}\,dt.
      \]
    \item Per Stokes' theorem, if \(\gamma\) bounds a disk \(u: D^2\to\R^4\), then
      \[
        A(\gamma) = \int_{D^2} u^* \omega_0 = \int_{D^2} \omega_0(\partial_x u, \partial_y u)\,dx\,dy.
      \]
  \end{itemize}
\end{frame}

\begin{frame}{Reeb flow}
  \begin{itemize}
    \item Smooth star-shaped hypersurface \(\Sigma\subset\R^4\). E.g., energy surface \(\Sigma = \{H=1\}\) for certain \(H\).
    \item Contact form: \(\alpha = \lambda_0|_{\Sigma}\).
    \item Reeb vector field \(R\) defined by
      \[
        \iota_R d\alpha = 0,
        \qquad
        \alpha(R) = 1.
      \]
    \item A Reeb orbit is a periodic solution \(\gamma:[0,T]\to\Sigma\) of
      \[
        \dot\gamma = R(\gamma).
      \]
    \item Useful fact: \(A(\gamma) = \int_0^T \alpha(R(\gamma(t)))\,dt = T\).
  \end{itemize}
\end{frame}

\begin{frame}{Closed Characteristics}
  \small
  \begin{center}
    \begin{tabular}{@{}p{0.30\textwidth}p{0.30\textwidth}p{0.30\textwidth}@{}}
      \toprule
      \textbf{Hamiltonian orbit} & \textbf{Reeb orbit} & \textbf{Closed characteristic} \\
      \midrule
      \(\gamma: [0,T]\to\R^4\) & \(\gamma: [0,T]\to\Sigma\) & \(\gamma: [0,1]\to\Sigma\) \\
      \(\dot\gamma(t) = J\nabla H(\gamma(t))\) & \(\dot\gamma(t) = R(\gamma(t))\) & \(\dot\gamma(t) \in \R_+ R(\gamma(t))\) \\
      \(H(\gamma) \equiv \mathrm{const}\) & \(\gamma(t)\in\Sigma\) & \(\gamma(t)\in\Sigma\) \\
      \(A(\gamma) = T\) & \(A(\gamma) = T\) & - \\
      \bottomrule
    \end{tabular}
  \end{center}
  \begin{itemize}
    \item We exclude constant loops with \(T=0, A(\gamma)=0\) from all three definitions.
  \end{itemize}
\end{frame}

\begin{frame}
  % TODO: handdrawn image that shows a smooth hypersurface \Sigma in R^3, with two curves \gamma,\tilde\gamma on it with arrows for directions.
  % Purpose: show that a closed characteristic is an unparameterized orbit (reparametrization freedom).
  % Suggestion: draw Σ as a surface, and two parameterizations of the same geometric curve (same image, different arrows/param ticks).
  \centering
  \DraftImage{HANDDRAWN: closed-characteristics-on-surface}{0.78\textheight}
\end{frame}

\section{Viterbo's Conjecture}

\begin{frame}{EHZ capacity / Minimum action}
  \begin{itemize}
    \item Fix a convex body \(K\subset\R^4\) (compact, convex, \(0 \in \mathrm{int}(K)\)).
    \item So far: smooth case, but we will generalize later.
    \item Define the Ekeland--Hofer--Zehnder capacity as
      \[
        \cEHZ(K) = A_{\min}(K) := \min\{A(\gamma) \colon \gamma \text{ closed characteristic on } \partial K\}.
      \]
    \item Note: we do not take the minimum over all loops, only over closed characteristics.
    \item This minimum is attained.
    \item Note: \(\cEHZ(K)>0\), since we exclude constant loops, and if \(\gamma\) is a closed characteristic, then \(A(\gamma)=T>0\).
  \end{itemize}
\end{frame}

\begin{frame}{EHZ capacity / Minimum action}
  \begin{block}{Symplectic Capacity}
    The map \(\cEHZ: K\mapsto \R_+\) is a symplectic capacity, i.e.:
    \begin{enumerate}
      \item (Monotonicity) If there exists a symplectic embedding \(K_1\to K_2\), then \(\cEHZ(K_1)\le \cEHZ(K_2)\).
      \item (Conformality) For \(\alpha\neq 0\), \(\cEHZ(\alpha K) = \abs{\alpha}^2 \cEHZ(K)\).
      \item (Normalization) \(\cEHZ(B^{4}(1)) = \cEHZ(Z^{4}(1)) = \pi\), where
        \[
          B^{4}(1) = \{x\in\R^4:\ \abs{x}\le 1\},
          \qquad
          Z^{4}(1) = \{(q_1,q_2,p_1,p_2):\ q_1^2 + p_1^2 \le 1\}.
        \]
    \end{enumerate}
  \end{block}
\end{frame}

\begin{frame}{Viterbo's Conjecture}
  \begin{block}{Viterbo's Conjecture}
    For any convex body \(K\subset\R^{2n}\):
    \[
      \sys(K) := \frac{\cEHZ(K)^n}{n!\,\mathrm{vol}(K)} \le 1.
    \]
  \end{block}
  \begin{block}{Strong Viterbo Conjecture}
    All normalized symplectic capacities agree on convex domains. In particular, the EHZ capacity, the Gromov width, and the cylindrical capacity coincide.
    \[
      c_{\mathrm{Gromov}}(K) := \sup\{\pi r^2:\ B^{2n}(r) \xhookrightarrow{\text{symp.}} K\}
    \]
    \[
      c_{\mathrm{cylindrical}}(K) := \inf\{\pi r^2:\ K \xhookrightarrow{\text{symp.}} Z^{2n}(r)\}
    \]
    \[
      \cEHZ(K) := \min \{A(\gamma):\ \gamma \text{ closed characteristic on } \partial K\}
    \]
  \end{block}
\end{frame}

\begin{frame}{Viterbo's Conjecture -- Counterexample}
  \begin{itemize}
    \item Viterbo's Conjecture has been open since 2000.
    \item Recently (2024), a counterexample in \(\R^4\) was found
  \end{itemize}
  %TODO
  \begin{block}{Theorem (Haim--Kislev 2024)}
    For the lagrangian product polytope \(K = P_5 \times_L 90^\circ \, P_5\) where \(P_5\) is the regular pentagon, and \(90^\circ \, P_5\) is its rotation by \(90^\circ\), we have 
    \[
      \sys(K) = \frac{\cEHZ(K)^2}{2!\,\mathrm{vol}(K)} = \frac{\sqrt{5}+3}{5} > 1.
    \]
  \end{block}
\end{frame}

\begin{frame}
  % TODO: handdrawn image
  % shows the two factors (P_5, 90° P_5) since we cannot draw a 4d polytope directly
  % shows a Billiard trajectory in red:
  %   arrowed lines, thick dots, both are numbered so that we can match the left side with the right
  %   each left line (q-space) corresponds to a right dot (p-space) and vice versa
  \centering
  \vfill
  \DraftImage[\textwidth]{HANDDRAWN: hk2024-billiard-trajectory-on-P5xP5}{0.90\textheight}
  \vfill
\end{frame}

\begin{frame}{Thesis Topic}
  \begin{itemize}
    \item Surprisingly simple counterexample to a longstanding conjecture!
    \item Previous searches for counterexamples stopped too early.
    \item What more can we learn from computational approaches?
    \item Thesis topic: probe Viterbo's conjecture \textbf{computationally}, make observations and conjectures.
  \end{itemize}
\end{frame}

\begin{frame}{Computing the EHZ capacity}
  \begin{itemize}
    \item Milestone: compute \(\cEHZ(K)\) for convex polytopes \(K\subset\R^4\).
    \item Why polytopes? 
      \begin{itemize}
        \item Dense in convex bodies (approximation).
        \item Finite combinatorial structure (facets, normals).
        \item Definitions can be generalized from the smooth setting (generalized closed characteristics).
      \end{itemize}
  \end{itemize}
\end{frame}

\section{Generalized Closed Characteristics on Polytopes}

\begin{frame}{Polytopes}
  \begin{columns}[T,onlytextwidth]
    \column{0.58\textwidth}
      \begin{itemize}
        \item A polytope is a compact intersection of finitely many half-spaces.
          \[
            K = \bigcap_{i=1}^N \{x\in\R^4:\ \inner{x}{n_i} \le h_i\}.
          \]
        \item Facet normals \(n_i\) are outward unit normals.
        \item Facet heights/supports \(h_i\) are positive, so that \(0\in\mathrm{int}(K)\).
        \item The boundary \(\partial K\) is \textbf{not smooth}
        \begin{itemize}
          \item Flat facets \(F_i = \{x:\ \inner{x}{n_i} = h_i\} \cap \partial K\).
          \item Facets intersect into 0,1,2-dimensional faces.
        \end{itemize}
      \end{itemize}
  
    \column{0.40\textwidth}
      % a 3d polytope, with one half-space shown as a transparent plane, labeled with one F_i, n_i, h_i
      \DraftImage{assets/manual/fig-polytope-normals-reeb.png}{0.62\textheight}
  \end{columns}
\end{frame}

\begin{frame}{Polytopes}
  \begin{itemize}
    \item Question: Polytopes are limits of smooth convex bodies. What are the limits of Hamiltonian orbits / Reeb orbits / closed characteristics?
      \[
        \dot\gamma(t) = R(\gamma(t)) ,\quad \iota_R d\alpha = 0 ,\quad \alpha(R) = 1
      \]
    \item Answer: Generalized Hamiltonian orbits / Reeb orbits / closed characteristics.
      \[
        \dot\gamma(t) \in \mathrm{conv} \{ R_i : \gamma(t) \in F_i \}
      \]
      where the \(R_i\) are the constant Reeb vectors on each facet \(F_i\).
      \[
        R_i = \frac{2}{h_i} J n_i
      \]
  \end{itemize}
\end{frame}

\begin{frame}
  % TODO: handdrawn image
  % shows a partial view of a 3d polytope
  % labeled facet F_i, normal n_i, Reeb vector R_i for i=1,2,3
  % obeys the combinatorial constraint: no two Reeb vectors point to/from each other at an 1-face
  % generalized closed characteristic \gamma in the image, straight lines on facets, and we "fake" slide along a 1-face (this doesn't really make sense i.e. is not a valid convex combination of R_1,R_2, but we cannot illustrate in 3d otherwise)
  \DraftImage{assets/manual/fig-polytope-reeb-vectors.png}{0.55\textheight}
\end{frame}

\begin{frame}{Generalized Closed Characteristics}
  \begin{center}
    \begin{tabular}{@{}p{0.30\textwidth}p{0.30\textwidth}p{0.30\textwidth}@{}}
      \toprule
      \textbf{Hamiltonian orbit} & \textbf{Reeb orbit} & \textbf{Closed characteristic} \\
      \midrule
      \(\gamma \in W^{1,2}([0,T],\R^4)\) & \(\gamma \in W^{1,2}([0,T],\partial K)\) & \(\gamma \in W^{1,2}([0,1],\partial K)\) \\
      \(\int_0^T \dot\gamma(t)\,dt = 0\) & \(\int_0^T \dot\gamma(t)\,dt = 0\) & \(\int_0^1 \dot\gamma(t)\,dt = 0\) \\
      \(\dot\gamma(t) \in J \partial H(\gamma(t))\) a.e. & \(\dot\gamma(t) \in \mathrm{conv}\{R_i:\ \gamma(t)\in F_i\}\) a.e. & \(\dot\gamma(t) \in J N_+(\gamma(t))\) a.e. \\
      \(A(\gamma) = T\) & \(A(\gamma) = T\) & - \\
      \bottomrule
    \end{tabular}
  \end{center}
  with
  \begin{itemize}
    \item Subdifferential \(\partial H(x) = \{y: H(z) \ge H(x) + \inner{y}{z-x}\ \forall z\}\), for a convex function \(H\).
    \item Normal cone \(N_+(x) = \R_+ \mathrm{conv}\{n_i:\ x\in F_i\}\).
  \end{itemize}
\end{frame}

\begin{frame}{EHZ capacity for polytopes}
  \begin{block}{Definition}
    The Ekeland--Hofer--Zehnder capacity of a convex polytope \(K\subset\R^{2n}\) is
    \[
      \cEHZ(K) = \min\{A(\gamma):\ \gamma \text{ generalized closed characteristic on } \partial K\}.
    \]
  \end{block}
  \begin{itemize}
    \item This minimum is attained.
    \item \(\cEHZ(K)\) is continuous with respect to the Hausdorff metric on convex bodies.
  \end{itemize}
\end{frame}

\begin{frame}{The Primal Optimization Problem}
  \begin{center}
    \begin{tabular}{@{}p{0.2\textwidth}p{0.7\textwidth}@{}}
      \toprule
      & \textbf{Primal Problem (Closed Characteristics)} \\
      \midrule
      Minimize & \(A(\gamma) = \int_\gamma \lambda_0\) \\
      \midrule
      Function Space & \(\gamma \in W^{1,2}([0,1],\R^4)\) \\
      \midrule
      Constraints & \(\int_0^1 \dot\gamma(t)\,dt = 0\) \\
                  & \(\gamma(t) \in \partial K\) for all \(t\) \\
                  & \(\dot\gamma(t) \in J N_+(\gamma(t))\) a.e. \\
      \bottomrule
    \end{tabular}
  \end{center}
  \begin{itemize}
    \item Infinite-dimensional search space.
    \item Constraints are hard to handle (especially \(\gamma(t)\in\partial K\)).
  \end{itemize}
\end{frame}


\begin{frame}{The Primal Optimization Problem}
  \[
    H(x) = g_K^2(x) ,\quad
    g_K(x) = \inf\{r>0:\ x\in rK\} \text{ (gauge function)}.
  \]
  \[
    \nabla H(x) = \frac{2}{h_i} n_i \text{ for } x\in \mathrm{int}(F_i) ,\quad
    J \partial H(x) = \mathrm{conv}\Big\{\frac{2}{h_i} J n_i:\ x\in F_i\Big\} = \mathrm{conv}\{R_i:\ x\in F_i\}.
  \]
  \begin{center}
    \begin{tabular}{@{}p{0.2\textwidth}p{0.7\textwidth}@{}}
      \toprule
      & \textbf{Primal Problem (Hamiltonian)} \\
      \midrule
      Minimize & \(A(\gamma) = \int_\gamma \lambda_0\) \\
      \midrule
      Function Space & \(\gamma \in W^{1,2}([0,T],\R^4)\) \\
      \midrule
      Constraints & \(\int_0^T \dot\gamma(t)\,dt = 0\) \\
                  & \(H(\gamma) \equiv 1\) \\
                  & \(\dot\gamma(t) \in J \partial H(\gamma(t))\) a.e. \\
      \bottomrule
    \end{tabular}
  \end{center}
\end{frame}

\section{Break}

\section{Clarke dual action principle}

\begin{frame}{Support and Gauge}
  \begin{itemize}
    \item Convex body \(K\subset\R^4\) with \(0\in\mathrm{int}(K)\).
    \item Support function:
      \[
        h_K(y) = \sup_{x\in K} \inner{x}{y}.
      \]
    \item Gauge (Minkowski functional):
      \[
        g_K(x) = \inf\{r>0:\ x\in rK\}.
      \]
    \item Note: \(g_K(x) \equiv 1\) on \(\partial K\).
    \item Note: \(h_K(n_i) = h_i\) for facet normals \(n_i\).
  \end{itemize}
\end{frame}

\begin{frame}{Legendre-Fenchel}
  \begin{block}{Lemma (Legendre-Fenchel duality)}
    For a convex body \(K\subset\R^4\) with \(0\in\mathrm{int}(K)\):
    \[
      g_K^2(x) = \sup_{y\in\R^4} \Bigl( \inner{x}{y} - \tfrac14 h_K^2(y) \Bigr),
    \]
    \[
      \tfrac14 h_K^2(y) = \sup_{x\in\R^4} \Bigl( \inner{x}{y} - g_K^2(x) \Bigr).
    \]
    We get an inequality:
    \[
      g_K^2(x) + \tfrac14 h_K^2(y) \ge \inner{x}{y}.
    \]
    Equality holds iff 
    \[
      g_K^2(x) + \tfrac14 h_K^2(y) = \inner{x}{y}
      \iff
      y \in \partial g_K^2(x)
      \iff x \in \partial (\tfrac14 h_K^2)(y).
    \]
  \end{block}
\end{frame}

\begin{frame}{Dual Problem}
  \begin{itemize}
    \item For a Hamiltonian orbit \(- J \dot\gamma \in \partial H(\gamma)\), the equality condition holds with $x=\gamma$, $y=-J\dot\gamma$:
      \[
        g_K^2(\gamma) + \tfrac14 h_K^2(-J\dot\gamma) = \inner{\gamma}{-J\dot\gamma}.
      \]
    \item Idea: integrate the Fenchel inequality over time, to switch from the minimization target \(A(\gamma)\) to some other functional \(I_K\) that depends on \(-J \dot\gamma\) only.
      \[
        I_K(z) = \tfrac14 \int_0^T h_K^2(-J\dot z(t))\,dt,
      \]
    \item For a Hamiltonian orbit with \(g_K^2(\gamma) \equiv 1\) we get:
      \[
        T + I_K(\gamma) = 2 A(\gamma) = 2 T \implies I_K(\gamma) = T = A(\gamma).
      \]
    \item Idea: show that this gives us a dual optimization problem in a variable \(z\), such that the critical points of \(I_K\) correspond 1:1 to Hamiltonian orbits.
  \end{itemize}
\end{frame}

\begin{frame}{Primal and Dual Problems}
  \begin{center}
    \begin{tabular}{@{}p{0.2\textwidth}p{0.35\textwidth}p{0.35\textwidth}@{}}
      \toprule
      & \textbf{Primal Problem} & \textbf{Dual Problem} \\
      \midrule
      Minimize & \(A(\gamma) = \int_\gamma \lambda_0\) & \(I_K(z) = \tfrac14 \int_0^T h_K^2(-J\dot z(t))\,dt\) \\
      \midrule
      Function Space & \(\gamma \in W^{1,2}([0,T],\R^4)\) & \(z \in W^{1,2}([0,T],\R^4)\) \\
      \midrule
      Constraints & \(\int_0^T \dot\gamma(t)\,dt = 0\) & \(\int_0^T \dot z(t)\,dt = 0\) \\
                  & \(H(\gamma) \equiv 1\) & \(\int_0^T \inner{-J\dot z(t)}{z(t)}\,dt = 2T\) \\
                  & \(\dot\gamma(t) \in J \partial H(\gamma(t))\) a.e. & - \\
                  & - & \(\int_0^T z(t)\,dt = 0\) \\
      \bottomrule
    \end{tabular}
  \end{center}
  \begin{block}{Theorem (Clarke dual action principle)}
    The minimizers of the primal and dual problems correspond 1:1, with 
    \[
      z = \gamma - \mathrm{center}(\gamma) \,,\quad
      I_K(z) = T = A(\gamma).
    \]
  \end{block}
\end{frame}

\begin{frame}{Proof: Clarke Dual Action Principle}
  \begin{itemize}
    \item From variational calculus, critical points of \(A\) are given by
      \[
        -J\dot\gamma(t) \in \partial g_K^2(\gamma(t)) \text{ a.e.} \,,\quad g_K^2(\gamma(t)) \equiv 1.
      \]
    \item From variational calculus, critical points of \(I_K\) under the constraints are given by
      \[
        z(t) + \mathrm{const} \in \partial (\tfrac14 h_K^2)(-J\dot z(t)) \text{ a.e.} \,,\quad \int_0^T \inner{-J\dot z(t)}{z(t)}\,dt = 2T.
      \]
    \item By the equality condition of Fenchel inequality, the first two differential inclusions are equivalent under the transformation \(z = \gamma - \mathrm{center}(\gamma)\).
    \item The second two constraints then are also equivalent.
    \item Finally we get by integrating the Fenchel equality over time that
      \[
        T + I_K(z) = 2 A(\gamma) \implies I_K(z) = T = A(\gamma).
      \]
  \end{itemize}
\end{frame}

\begin{frame}{Dual Problem Is Nicer}
  \begin{center}
    \begin{tabular}{@{}p{0.2\textwidth}p{0.35\textwidth}p{0.35\textwidth}@{}}
      \toprule
      & \textbf{Primal Problem} & \textbf{Dual Problem} \\
      \midrule
      Minimize & \(A(\gamma) = \int_\gamma \lambda_0\) & \(I_K(z) = \tfrac14 \int_0^T h_K^2(-J\dot z(t))\,dt\) \\
      \midrule
      Function Space & \(\gamma \in W^{1,2}([0,T],\R^4)\) & \(z \in W^{1,2}([0,T],\R^4)\) \\
      \midrule
      Constraints & \(\int_0^T \dot\gamma(t)\,dt = 0\) & \(\int_0^T \dot z(t)\,dt = 0\) \\
                  & \(H(\gamma) \equiv 1\) & \(\int_0^T \inner{-J\dot z(t)}{z(t)}\,dt = 2T\) \\
                  & \(\dot\gamma(t) \in J \partial H(\gamma(t))\) a.e. & - \\
                  & - & \(\int_0^T z(t)\,dt = 0\) \\
      \bottomrule
    \end{tabular}
  \end{center}
  \begin{itemize}
    \item Minimization target \(I_K\) depends only on velocity \(\dot z\).
    \item Function space has an action constraint, but no direct position constraint.
    \item Function space has no differential inclusion constraint.
  \end{itemize}
\end{frame}

\begin{frame}{Existence of Simple Minimizers}
  \begin{block}{Theorem (Haim-Kislev 2019)}
    Let \(K\subset\R^{2n}\) be a convex polytope.
    Then there exists a generalized closed characteristic / Reeb orbit / Hamiltonian orbit \(\gamma^* \in W^{1,2}([0,T],\partial K)\) with minimal action \(\cEHZ(K)\), such that
    \begin{itemize}
      \item \(\gamma^*\) is piecewise affine (breakpoints need not be on facet intersections).
      \item Velocities \(\dot\gamma^*(t)\) are pure facet Reeb vectors, not convex combinations.
      \item Each facet Reeb vector appears at most once, i.e. 
        \[
          \{t: \dot\gamma^*(t) = R_i\} \text{ is an interval or empty, for each } i.
        \]
    \end{itemize}
  \end{block}
\end{frame}

\begin{frame}
  %TODO: handdrawn image
  % Purpose: visualize the "split + rearrange" operations on a piecewise affine loop.
  % Suggestion: time axis with colored blocks (velocities), and show before/after for Step 2 and Step 3.
  \centering
  \DraftImage{HANDDRAWN: proof-operations-time-axis}{0.72\textheight}
\end{frame}

\begin{frame}{Proof outline: existence of a simple minimizer}
  \begin{enumerate}
    \item \textbf{Approximate:} replace \(z\) by piecewise affine \(z_N\).
    \item \textbf{Split:} convex-combination velocities \(\to\) pure facet velocities.
    \item \textbf{Rearrange:} group equal velocities (``grow+shrink'').
    \item \textbf{Renormalize:} restore the action constraint by rescaling.
    \item \textbf{Compactness:} take a limit in a finite-dimensional parameter space.
  \end{enumerate}

  % TODO(Joern): say one sentence per bullet, then move on fast.
  % TODO(Joern): emphasize: Step 2+3 use the "area/action identity" and the fact that I_K depends only on the velocity scale.
\end{frame}

\begin{frame}{Step 1: Approximation by piecewise affine loops}
  \begin{itemize}
    \item Start with a minimizer \(z\) of the dual problem.
    \item Approximate \(z\) in \(W^{1,2}\) by piecewise affine loops \(z_N\).
    \item Ensure \(\dot z_N(t)\) stays in the same allowed cone:
      \[
        \dot z_N(t)\in \mathrm{conv}\{R_i\}\quad\text{a.e.}
      \]
    \item Then \(A(z_N)\to A(z)\) and \(I_K(z_N)\to I_K(z)\) by continuity of the integrals.
  \end{itemize}

  % TODO(Joern): blackboard: state the approximation lemma you want to assume (HK2017-style).
\end{frame}

\begin{frame}{Step 2: Split convex combinations into pure facet velocities}
  \begin{itemize}
    \item We may have mixed velocities (note that in the dual problem we do not track what facets \(z\) lies on).
      \[
        \dot z_N(t)\in \mathrm{conv}\{R_{1},\ldots,R_{F}\}.
      \]
    \item Replace each mixed segment by a concatenation of segments with \emph{pure} velocities \(R_{i_j}\).
    \item Choose the time order so that the action/area constraint does not decrease:
      \[
        A(z_N') \ge A(z_N).
      \]
    \item Possible since reversing the time order reverses the change in area.
    \item For pure paths we have
      \[
        I_K(z_N') = T
      \]
      So the functional changed only slightly.
  \end{itemize}

  % TODO(Joern): blackboard: show the 2-segment "swap" identity (order matters for area).
\end{frame}

\begin{frame}{Step 3: Rearrange equal velocities (``grow+shrink'')}
  \begin{itemize}
    \item After splitting, \(\dot z\) takes values in a finite set \(\{R_1,\ldots,R_F\}\).
    \item If some \(R_i\) appears in disjoint time intervals, we merge them by a rearrangement step.
    \item One chooses the rearrangement direction so the action constraint does not decrease:
      \[
        A(z_N'') \ge A(z_N').
      \]
    \item Again possible since reversing the pair reverses the change in area.
    \item The functional \(I_K\) is unchanged by this rearrangement (it depends only on the velocity / time spent).
  \end{itemize}

  % TODO(Joern): blackboard: draw the "time blocks" picture; explain why one direction increases the symplectic area.
\end{frame}

\begin{frame}{Step 4: Renormalize to restore the action constraint}
  \begin{itemize}
    \item After Steps 2--3, we may have increased the action:
      \(A(z_N'') \ge A(z_N)\)
    \item Scale the time spent on each segment by factor $\beta$ to restore the constraint:
      \[
        A(z_N''') = T_N''' = T \cdot \beta = A(z_N'') \beta^2
      \]
      \[
        \beta = T / A(z_N'')
      \]
    \item Then \(A(z_N''')=T'''\), and \(I_K(z_N''') = \beta I_K(z_N'') = \beta T = T_N''' \)
  \end{itemize}

  % TODO(Joern): blackboard: scaling laws: A(\lambda z)=\lambda^2 A(z), I_K(\lambda z)=\lambda^2 I_K(z).
\end{frame}

\begin{frame}{Step 5: Compactness and taking a limit}
  \begin{itemize}
    \item A simple loop is encoded by finite data:
      \[
        (\text{facet order } \sigma,\ \text{segment lengths } (|I_i|)).
      \]
    \item The order set is finite, and the set of segment lengths that fulfills our constraints is compact.
    \item Take an approximation sequence, take their simple loops and extract a convergent subsequence.
    \item We get by \(A(z_N)\to A(z)\) and \(I_K(z_N) \to I_K(z)\) and minimality of \(I_K(z)\) and by \(I_K(z_N''') = T^2 / A(z_N'')\) that \(T^*=T\), \(I_K(z^*) = I_K(z)\), so we have found a simple minimum.
  \end{itemize}

  % TODO(Joern): say: "this is where functional-analytic details hide; we use standard compactness + l.s.c."
\end{frame}

\begin{frame}{Remark: the manipulations are homotopies}
  \begin{block}{Remark}
    Each modification step (approximation, splitting, grow+shrink, rescale) can be done as a homotopy through minimizers of the dual problem.
    Thus, every minimizer is homotopic to a simple minimizer, though we don't claim all minimizers are connected via such homotopies.
  \end{block}
\end{frame}

\begin{frame}{Dual Optimization Problem -- Finite Dimensional Reduction}
  \begin{center}
    \begin{tabular}{@{}p{0.2\textwidth}p{0.7\textwidth}@{}}
      \toprule
      & \textbf{Dual Problem (Simple Piecewise Affine)} \\
      \midrule
      Minimize & \(I_K = \sum |I_i|\) \\
      \midrule
      Function Space & Facet order \(\sigma \in S_N\) \\
                     & Segment times \(|I_i| \ge 0\) \\
      \midrule
      Constraints & \(\sum |I_i| R_{\sigma(i)} = 0\) \\
                  & \(\sum_{j<i} |I_i| |I_j| \omega(R_{\sigma(i)}, R_{\sigma(j)}) = 2 \sum |I_i|\) \\
      \bottomrule
    \end{tabular}
  \end{center}
  \begin{itemize}
    \item The problem can be further turned into a Quadratic Programming problem.
    \item There's additional structure to exploit, e.g. facet adjacency constraints on what changes of velocities are possible.
  \end{itemize}
\end{frame}

\begin{frame}{References}
  \footnotesize
  \begin{itemize}
    \item P.\ Haim-Kislev, \emph{On the Symplectic Size of Convex Polytopes},
      \emph{Geom.\ Funct.\ Anal.} 29(2):440--463 (2019).
      doi:\,10.1007/s00039-019-00486-4. (arXiv:\,1712.03494)
    \item P.\ Haim-Kislev, Y.\ Ostrover, \emph{A Counterexample to Viterbo's Conjecture},
      arXiv:\,2405.16513v3 (2025). doi:\,10.48550/arXiv.2405.16513.
    \item J.\ Chaidez, M.\ Hutchings, \emph{Computing Reeb dynamics on four-dimensional convex polytopes},
      \emph{J.\ Comput.\ Dyn.} 8(4):403--445 (2021). doi:\,10.3934/jcd.2021016. (arXiv:\,2008.10111)
  \end{itemize}
\end{frame}

\begin{frame}{Summary}
  \begin{enumerate}
    \item Viterbo's conjecture requires solving an optimization problem over closed characteristics.
    \item Clarke's Dual Action Principle switches to a dual problem over arbitrary loops.
    \item In the dual problem, we can show existence of minimizers that are piecewise affine and use only pure facet Reeb vectors.
    \item AND we can show that the existence of minimizers that use each facet Reeb vector at most once!
    \item This finally yields a finite-dimensional optimization problem.
  \end{enumerate}
\end{frame}

\end{document}
