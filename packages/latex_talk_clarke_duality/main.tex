% Clarke's duality principle — seminar talk slides (ephemeral)
%
% IMPORTANT (lifecycle):
% - This package is meant for fast iteration during talk prep.
% - The final "as presented" version should be an annotated git tag:
%     talk-clarke-duality-YYYY-MM-DD
% - After migrating content into the thesis, this package may be deleted from `main`.
%   In that case, use the tag/commit to retrieve the exact sources/assets.
% - Post-talk workflow: see CHECKLIST_TEARDOWN.md
%
% Build:
%   ./scripts/build.sh
%
\documentclass[10pt,aspectratio=169]{beamer}

% Theme (nice defaults, no custom tikz required in authoring)
\usetheme[progressbar=frametitle]{metropolis}

% Fonts + basic typesetting
\usepackage[T1]{fontenc}
\usepackage{lmodern}
\usepackage{microtype}

% Math
\usepackage{amsmath,amssymb,mathtools}

% Tables / layout helpers
\usepackage{booktabs}

% Figures
\usepackage{graphicx}

% Drafting helpers:
% - Red text marks items that need J\"orn to verify: \NeedsCheck{...}
% - Slide variants are separate frames, tagged in code with: %variant: ...
%   and visible in the PDF as a frame subtitle like "(Variant 5-B)".
% - Hand-drawn image placeholders use \DraftImage{assets/manual/<file>.png}{<height>}
%   which renders a big box if the file does not exist.

% Links
% (beamer loads hyperref; we just tune it)
\hypersetup{
  colorlinks=true,
  linkcolor=blue,
  urlcolor=blue
}

% Local macros + theorem setup
% Local preamble for a short-lived talk (keep this small).

% Common math shorthands
\newcommand{\R}{\mathbb{R}}
\newcommand{\C}{\mathbb{C}}
\newcommand{\Z}{\mathbb{Z}}
\newcommand{\Q}{\mathbb{Q}}
\newcommand{\N}{\mathbb{N}}
\newcommand{\T}{\mathbb{T}}

\newcommand{\abs}[1]{\lvert #1\rvert}
\newcommand{\norm}[1]{\lVert #1\rVert}
\newcommand{\inner}[2]{\langle #1, #2 \rangle}
\newcommand{\conv}{\operatorname{conv}}

% Symplectic shorthand (match thesis notation)
\newcommand{\cEHZ}{c_{\mathrm{EHZ}}}
\newcommand{\gauge}{g}
\newcommand{\support}{h}
\newcommand{\polar}{\circ}

% Beamer theorem blocks (numbered)
\setbeamertemplate{theorems}[numbered]


% Build metadata (generated by scripts/gen-version.sh; gitignored)
\providecommand{\TalkGitCommit}{unknown}
\providecommand{\TalkBuildDate}{\today}
\IfFileExists{version.tex}{\input{version}}{}

\setbeamertemplate{frame footer}{\scriptsize source: \texttt{\TalkGitCommit}}

\title{Clarke's Duality Principle}
\author{J\"orn St\"ohler}
\institute{Symplectic Geometry Seminar}
\date{\TalkBuildDate\quad{\scriptsize source: \texttt{\TalkGitCommit}}}

\begin{document}

\maketitle

%variant: 0-A
\begin{frame}{Big picture}{\VariantLabel{Variant 0-A}}
  \begin{itemize}
    \item Thesis context: Viterbo-type questions reduce to understanding \emph{minimum-action} closed characteristics.
    \item For polytopes, the dynamics is non-smooth and the ``minimize action'' problem has nasty constraints.
    \item Clarke's dual action principle reframes the problem as a constrained minimization on loops
      where the functional depends only on the \emph{velocity}.
    \item This exposes a rearrangement argument \(\Rightarrow\) a surprisingly \emph{simple} minimum-action orbit:
      polygonal, and each facet direction is used at most once.
  \end{itemize}
\end{frame}

\begin{frame}{Outline}
  \begin{enumerate}
    \item Notation refresh: Reeb vs Hamiltonian vs closed characteristics
    \item Thesis context: Viterbo conjecture, and the HK2024 counterexample
    \item Polytopes: generalized closed characteristics and why the primal problem is hard
    \item Clarke dual action principle (full proof sketch)
    \item Break
    \item HK2017: rearrangement \(\Rightarrow\) facet-once minimizer (detailed proof sketch)
    \item (Optional) computation takeaway
    \item References \(\rightarrow\) pipeline takeaway
  \end{enumerate}
\end{frame}

\section{Module 1: Symplectic dynamics (refresh)}

\begin{frame}{Standard data and action (fix notation)}
  \begin{itemize}
    \item Work in \((\R^4,\omega_0)\) with \(\omega_0(u,v)=\inner{Ju}{v}\) and \(\lambda_0=\tfrac12\inner{Jx}{dx}\).
    \item For a loop \(\gamma:[0,T]\to\R^4\), define the action
      \[
        A(\gamma)=\int_\gamma \lambda_0
        =\tfrac12\int_0^T \inner{J\gamma(t)}{\dot\gamma(t)}\,dt
        =\tfrac12\int_0^T \inner{-J\dot\gamma(t)}{\gamma(t)}\,dt.
      \]
    \item (Remark) Most statements work in \(\R^{2n}\) similarly; we keep \(n=2\) in the talk.
  \end{itemize}
\end{frame}

%variant: 1-A
\begin{frame}{Energy surfaces and the Reeb flow}{\VariantLabel{Variant 1-A}}
  \begin{columns}[T,onlytextwidth]
    \column{0.52\textwidth}
      \begin{itemize}
        \item Let \(\Sigma\subset\R^4\) be a smooth star-shaped hypersurface.
        \item Contact form: \(\alpha=\lambda_0|_{\Sigma}\).
        \item Reeb vector field \(R\) is defined by
          \[
            \iota_R d\alpha = 0,
            \qquad
            \alpha(R)=1.
          \]
        \item A Reeb orbit is \(\gamma:[0,T]\to\Sigma\) with \(\dot\gamma=R(\gamma)\), \(\gamma(0)=\gamma(T)\).
        \item Then \(A(\gamma)=\int_0^T \alpha(\dot\gamma)\,dt = T\).
      \end{itemize}

    \column{0.48\textwidth}
      % Purpose: remind the audience what “energy surface + flow line” means, and fix the Reeb normalization α(R)=1.
      % Image suggestion: draw a 3D energy surface Σ (e.g. ellipsoid), a tangent arrow field R, and one periodic orbit γ.
      % Annotate: α = λ|_Σ, α(R)=1, and a small “(3D mock)” tag.
      \DraftImage{assets/manual/fig-energy-surface-reeb.png}{0.55\textheight}
  \end{columns}
\end{frame}

%variant: 1-B (more explicit comparison table; choose 1-A or 1-B later)
\begin{frame}{Three equivalent viewpoints (memorize this slide)}{\VariantLabel{Variant 1-B}}
  \small
  \begin{center}
    \begin{tabular}{@{}p{0.30\textwidth}p{0.32\textwidth}p{0.32\textwidth}@{}}
      \toprule
      \textbf{Object} & \textbf{Equation (smooth case)} & \textbf{Notes} \\
      \midrule
      Closed characteristic
      & \(\dot\gamma(t)\in \ker(\omega|_{T\Sigma})\)
      & unparameterized (reparametrization-free) \\
      \addlinespace[0.4em]
      Reeb orbit
      & \(\dot\gamma(t)=R(\gamma(t))\), \(\alpha(R)=1\)
      & parameterized, period \(T\); \(A(\gamma)=T\) \\
      \addlinespace[0.4em]
      Hamiltonian orbit
      & \(\dot\gamma(t)=J\nabla H(\gamma(t))\)
      & on a level set \(H=\mathrm{const}\) \\
      \bottomrule
    \end{tabular}
  \end{center}

  \vspace{0.6em}
  \small
  On polytopes, \(\nabla\) becomes a subgradient \(\partial\) and equations become inclusions.
\end{frame}

\section{Module 1.5: Thesis context (Viterbo)}

\begin{frame}{EHZ capacity and Viterbo's conjecture (in 4D)}
  \begin{itemize}
    \item For a smooth convex body \(K\subset\R^4\), the EHZ capacity is the minimal action
      \[
        \cEHZ(K)=\min\{A(\gamma): \gamma \text{ closed characteristic on }\partial K\}.
      \]
    \item Define the systolic ratio (4D):
      \[
        \sys(K):=\frac{\cEHZ(K)^2}{2\,\mathrm{vol}(K)}.
      \]
    \item Viterbo's conjecture predicts \(\sys(K)\le 1\) for convex \(K\), with equality for the ball.
  \end{itemize}

  \small
  (General dimension: replace the exponent \(2\) by \(n\) and \(2\) by \(n!\).)
\end{frame}

\begin{frame}{Strong Viterbo (very informal)}
  \begin{columns}[T,onlytextwidth]
    \column{0.56\textwidth}
      \begin{itemize}
        \item There are many symplectic capacities \(c(\cdot)\).
        \item \textbf{Strong Viterbo (informal):} on convex bodies, all reasonable capacities agree.
        \item In particular, a capacity becomes computable if we can compute \(\cEHZ\).
      \end{itemize}

    \column{0.44\textwidth}
      % Purpose: give the audience a mental hook that “this is not random — it’s a central conjecture theme”.
      % Image suggestion: a “map” bubble diagram: Convex bodies → capacities → equal? → Viterbo/Mahler landscape.
      \DraftImage{assets/manual/fig-viterbo-landscape.png}{0.62\textheight}
  \end{columns}
\end{frame}

\begin{frame}{Why polytopes? (computationally accessible)}
  \begin{columns}[T,onlytextwidth]
    \column{0.52\textwidth}
      \begin{itemize}
        \item Polytopes approximate convex bodies arbitrarily well (Hausdorff metric).
        \item They have finite combinatorial data: facets, normals, adjacency.
        \item So: if we can describe minimizers combinatorially, we can compute with them.
      \end{itemize}

    \column{0.48\textwidth}
      % Purpose: visualize “smooth body → polytope → smoothing K_ε” and convince that the polytope setting is not artificial.
      % Image suggestion: 3 columns (or 2): smooth rounded body, a polytope approximation, and a slightly “inflated” smoothing K_ε.
      \DraftImage{assets/manual/fig-polytope-approximation-smoothing.png}{0.55\textheight}
  \end{columns}
\end{frame}

%variant: 1.5-A
\begin{frame}{Counterexample exists (HK2024)}{\VariantLabel{Variant 1.5-A}}
  \begin{columns}[T,onlytextwidth]
    \column{0.52\textwidth}
      \begin{itemize}
        \item HK2024 gives a counterexample polytope (in \(\R^4\)) with \(\sys(K)>1\).
        \item For this talk, this is mainly a \textbf{hook}: the minimizer structure matters.
        \item Side note: there can be multiple minimum-action generalized closed characteristics.
      \end{itemize}

    \column{0.48\textwidth}
      % Purpose: give one concrete object for the audience to “anchor” the discussion, without diving into details.
      % Image suggestion: schematic picture of the counterexample polytope (or product structure), plus a small label “HK2024”.
      \DraftImage{assets/manual/fig-hk2024-counterexample.png}{0.55\textheight}
  \end{columns}
\end{frame}

\begin{frame}{What this talk focuses on}
  \begin{itemize}
    \item We \textbf{do not} focus on the full counterexample construction.
    \item We focus on a reusable toolchain:
      \[
        \text{(primal) minimize action on }\partial K
        \ \xRightarrow{\ \text{Clarke duality}\ }\ 
        \text{(dual) minimize a convex velocity functional}
        \ \xRightarrow{\ \text{rearrangement}\ }\ 
        \text{simple minimizer}.
      \]
    \item This is exactly the kind of statement that turns geometry into computation.
  \end{itemize}
\end{frame}

\section{Module 2: Polytopes + generalized dynamics}

\begin{frame}{Polytopes: facets, normals, heights, facet velocities}
  \begin{columns}[T,onlytextwidth]
    \column{0.58\textwidth}
      \begin{itemize}
        \item Write \(K\subset\R^4\) (full-dimensional, \(0\in\mathrm{int}(K)\)) as an irredundant intersection of halfspaces:
          \[
            K=\bigcap_{i=1}^F\{x:\ \inner{x}{n_i}\le h_i\},
            \qquad
            \abs{n_i}=1,\quad h_i=\support_K(n_i)>0.
          \]
        \item Facet \(F_i = \{x\in K:\ \inner{x}{n_i}=h_i\}\).
        \item ``Facet Reeb velocity'' (HK2017):
          \[
            p_i := \frac{\NeedsCheck{2}}{h_i}\,J n_i.
          \]
      \end{itemize}

    \column{0.42\textwidth}
      % Purpose: familiarize the symbols F_i, n_i, h_i, p_i and their geometric meaning (normal vs tangent velocity).
      % Image suggestion: 3D mock polytope with 3 facets labeled F_1,F_2,F_3; outward unit normals n_i; tangent arrows p_i.
      % Add a [3D mock] warning; on the side write p_i = (2/h_i) J n_i and “p_i ⟂ n_i”.
      \DraftImage{assets/manual/fig-polytope-normals-reeb.png}{0.62\textheight}
  \end{columns}
\end{frame}

\begin{frame}{Generalized closed characteristics on a polytope}
  \begin{itemize}
    \item Normal cone at \(x\in\partial K\):
      \[
        N_K(x):=\R_+\,\conv\{n_i:\ x\in F_i\}.
      \]
    \item Allowed velocity cone at \(x\): \(J N_K(x)\).
    \item A generalized closed characteristic is \(\gamma\in W^{1,2}\) with \(\gamma(t)\in\partial K\) and
      \[
        \dot\gamma(t)\in JN_K(\gamma(t))\quad\text{for a.e. }t.
      \]
  \end{itemize}

  \begin{columns}[T,onlytextwidth]
    \column{0.52\textwidth}
      \small
      If \(\gamma(t)\in\mathrm{int}(F_i)\), then \(\dot\gamma(t)\) is a positive multiple of \(J n_i\) (equivalently \(p_i\)).
      On lower faces, \(\dot\gamma(t)\) can be a convex combination of neighboring facet directions.

    \column{0.48\textwidth}
      % Purpose: remind the audience that on lower-dimensional faces we have a *cone* of allowed directions (and the orbit can “slide”).
      % Image suggestion: mock polytope with two facets; draw the edge between them; at a point on the edge draw the convex cone of velocities.
      % Also draw a path γ that goes across facet 1, then “slides” near the edge, then across facet 2. Mark [3D mock].
      \DraftImage{assets/manual/fig-polytope-reeb-cone-and-orbit.png}{0.55\textheight}
  \end{columns}
\end{frame}

%variant: 2-A
\begin{frame}{Limits work out (not a focus today)}{\VariantLabel{Variant 2-A}}
  \begin{columns}[T,onlytextwidth]
    \column{0.52\textwidth}
      \begin{itemize}
        \item If \(K_\varepsilon\) is a smoothing of a polytope \(K\), closed characteristics on \(\partial K_\varepsilon\)
          converge (after subsequences) to generalized closed characteristics on \(\partial K\).
        \item Capacities (and volume) are continuous under Hausdorff limits.
        \item We will use this only as \textbf{sanity}: our generalized definitions are compatible with smooth geometry.
      \end{itemize}

    \column{0.48\textwidth}
      % Purpose: show “smoothings K_ε → polytope K” and that orbits limit to generalized ones.
      % Image suggestion: draw K, the offset body K_ε (rounded edges), and one orbit on K_ε converging to a broken orbit on K.
      \DraftImage{assets/manual/fig-smoothing-limit-orbits.png}{0.55\textheight}
  \end{columns}
\end{frame}

\section{Module 3: The primal minimization problem}

\begin{frame}{The primal problem (hard constraints)}
  \begin{block}{Primal minimization (geometric)}
    \[
      \cEHZ(K) = \min\{A(\gamma):\ \gamma\subset\partial K,\ \dot\gamma(t)\in JN_K(\gamma(t))\ \text{a.e.}\}.
    \]
  \end{block}

  \begin{itemize}
    \item Infinite-dimensional search space (\(W^{1,2}\)-loops).
    \item Hard state constraint \(\gamma(t)\in\partial K\).
    \item Differential inclusion + reparameterization freedom.
    \item Naive local optimization wants to collapse to a point (action \(0\)), but that is not allowed.
  \end{itemize}
\end{frame}

\section{Module 4: Legendre--Fenchel (the convex-analytic engine)}

\begin{frame}{Support vs gauge: the key inequality}
  \begin{columns}[T,onlytextwidth]
    \column{0.58\textwidth}
      \begin{itemize}
        \item Support function: \(\support_K(y)=\sup_{x\in K}\inner{x}{y}\).
        \item Gauge (Minkowski functional): \(\gauge_K(x)=\inf\{r>0:\ x\in rK\}\) (so \(\gauge_K\equiv 1\) on \(\partial K\)).
        \item Fenchel inequality (pointwise in \(\R^4\)):
          \[
            \gauge_K^2(x)+\NeedsCheck{\tfrac14}\,\support_K^2(y)\ \ge\ \inner{x}{y}.
          \]
        \item Equality condition:
          \(y\in\partial \gauge_K^2(x)\) iff \(x\in\partial(\NeedsCheck{\tfrac14}\support_K^2)(y)\).
      \end{itemize}

    \column{0.42\textwidth}
      % Purpose: refresh Legendre–Fenchel duality intuition quickly (why “support” and “gauge” are dual).
      % Image suggestion: 1D convex function picture: plot f and its Legendre transform f*, with a tangent line picture.
      \DraftImage{assets/manual/fig-fenchel-1d-intuition.png}{0.62\textheight}
  \end{columns}
\end{frame}

\section{Module 5: Clarke dual action principle}

%variant: 5-A
\begin{frame}{Clarke dual action principle: primal vs dual problems}{\VariantLabel{Variant 5-A}} %variant: 5-A
  \begin{columns}[T,onlytextwidth]
    \column{0.50\textwidth}
      \begin{block}{Primal problem (geometry on \(\partial K\))}
        \[
          \cEHZ(K)
          = \min\Bigl\{A(\gamma):\ \gamma\subset\partial K,\ \dot\gamma(t)\in JN_K(\gamma(t))\ \text{a.e.}\Bigr\}.
        \]
      \end{block}

      \begin{itemize}
        \item Variable: \(\gamma:[0,T]\to\partial K\) (generalized closed characteristic / Reeb orbit).
        \item Hard constraint: \(\gamma(t)\in\partial K\) for all \(t\).
        \item Objective: minimize the symplectic action \(A(\gamma)=\int_\gamma \lambda_0\).
      \end{itemize}

    \column{0.50\textwidth}
      \begin{block}{Dual problem (convex analysis on loops in \(\R^{2n}\))}
        \[
          I_K(z)=\NeedsCheck{\frac14}\int_0^T \support_K^2\!\bigl(-J\dot z(t)\bigr)\,dt,
          \qquad
          z\in\mathcal{E}^{T}.
        \]
        \[
          \mathcal{E}^{T}
          =
          \Bigl\{
            z\in W^{1,2}([0,T],\R^{2n}):
            z(0)=z(T),\
            \int_0^T z\,dt=0,\
            \int_0^T \inner{-J\dot z}{z}\,dt=1
          \Bigr\}.
        \]
      \end{block}

      \begin{itemize}
        \item Variable: \(z:[0,T]\to\R^{2n}\) (not constrained to \(\partial K\)).
        \item The functional depends only on the \emph{velocity} \(-J\dot z\).
        \item The constraint \(\int\inner{-J\dot z}{z}=1\) prevents collapse to \(0\).
      \end{itemize}
  \end{columns}
\end{frame}

%variant: 5-A2
\begin{frame}{Clarke dual action principle: critical points correspond}{\VariantLabel{Variant 5-A2}} %variant: 5-A2
  \begin{block}{Theorem (Clarke dual action principle; talk normalization on \([0,T]\))}
    Let \(K\subset\R^{2n}\) be a convex body.
    Then:
    \begin{enumerate}
      \item Critical points of \(I_K|_{\mathcal{E}^{T}}\) correspond \textbf{1:1} to closed characteristics on \(\partial K\),
        up to the natural symmetries (\NeedsCheck{time-shift/reparametrization?} and translation of \(z\)).
      \item Minimizers correspond to \textbf{minimum-action} closed characteristics, and
        \[
          \cEHZ(K)=\min_{z\in \mathcal{E}^{T}} \NeedsCheck{2}\,I_K(z).
        \]
    \end{enumerate}
  \end{block}

  \small
  \textbf{What to remember:} primal = hard constraint \(\gamma(t)\in\partial K\); dual = hard constraint is moved into
  \(\mathcal{E}^{T}\), while \(I_K\) becomes a velocity-only functional.
\end{frame}

%variant: 5-B (talk-friendly normalization; free period T + centering)
\begin{frame}{Talk normalization (free period \(T\) and centering)}{\VariantLabel{Variant 5-B}}
  \begin{itemize}
    \item Keep \(I_K\) as
      \[
        I_K(z)=\frac14\int_0^T \support_K^2\!\bigl(-J\dot z(t)\bigr)\,dt.
      \]
    \item We prefer to normalize so that
      \[
        A(z)=T,
        \qquad
        \int_0^T z(t)\,dt = 0
        \quad(\text{fix translation}).
      \]
    \item Then (for minimizers) we can arrange the bookkeeping so that
      \[
        \NeedsCheck{I_K(z_{\min})=T=\cEHZ(K)}.
      \]
  \end{itemize}

  \small
  This avoids the \([0,1]\) normalization and helps keep ``\(T\)'' consistent throughout the talk.
\end{frame}

%variant: 5-C (explicit rescaling from HK2017 normalization to “A=I=T” normalization)
\begin{frame}{Normalization check: rescale \(z_0\) on \([0,1]\) to \(z_T\) on \([0,T]\)}{\VariantLabel{Variant 5-C}}
  \small
  Start with a minimizer \(z_0\in\mathcal{E}^{HK}\) of \(I_K\) on \([0,1]\), and set
  \[
    T := \NeedsCheck{2\,I_K(z_0)} \quad (= \cEHZ(K)\ \text{in HK2017}).
  \]
  Define \(z_T:[0,T]\to\R^{2n}\) by
  \[
    z_T(t) := \NeedsCheck{\sqrt{2T}}\,z_0\!\Big(\frac{t}{T}\Big).
  \]
  Then one checks:
  \[
    A(z_T)=T,
    \qquad
    I_K(z_T)=T.
  \]

  \vspace{0.5em}
  \footnotesize
  % Purpose: make the scaling bookkeeping explicit so we can later freely talk on [0,T].
  % No hand-drawn image needed here.
  I marked the scaling constants in red so you can verify them once and then we can treat them as fixed.
\end{frame}

%variant: 5-D
\begin{frame}{Proof idea: why duality is true (smooth case)}{\VariantLabel{Variant 5-D}}
  \begin{itemize}
    \item Start from Fenchel inequality:
      \[
        \gauge_K^2(x)+\tfrac14\,\support_K^2(y)\ \ge\ \inner{x}{y}.
      \]
    \item Plug \(x=z(t)\), \(y=-J\dot z(t)\), integrate over \([0,T]\):
      \[
        \int_0^T \gauge_K^2(z)\,dt + I_K(z)\ \ge\ 2A(z).
      \]
    \item For a Reeb orbit on \(\partial K\), we have \(\gauge_K\equiv 1\) \(\Rightarrow \int_0^T \gauge_K^2 = T\)
      and (Reeb parameterization) \(A(z)=T\).
    \item So \(T + I_K(z)\ge 2T\), hence \(I_K(z)\ge T\).
    \item Equality \(\Leftrightarrow -J\dot z(t)\in \partial \gauge_K^2(z(t))\) a.e.\ \(\Leftrightarrow\) Hamiltonian/Reeb equation.
  \end{itemize}
\end{frame}

%variant: 5-E (where subgradients enter; polytope caveat)
\begin{frame}{Where non-smoothness enters (polytopes)}{\VariantLabel{Variant 5-E}}
  \begin{itemize}
    \item On polytopes, gradients do not exist everywhere.
    \item Replace \(\nabla\) by the convex subdifferential \(\partial\).
    \item Hamiltonian equation becomes the inclusion
      \[
        \dot\gamma(t)\in J\,\partial \gauge_K^2(\gamma(t))\quad\text{a.e.}
      \]
      and closed characteristics correspond to weak critical points of \(I_K\) on \(\mathcal{E}^{HK}\).
  \end{itemize}

  \small
  (We will not drown in technicalities today; the rearrangement argument is the main focus.)
\end{frame}

%variant: break
\begin{frame}{Break}{\VariantLabel{5--10 min}}
  \begin{center}
    \Large Break
  \end{center}
  \vspace{0.5em}
  \small
  After the break: HK2017 rearrangement proof \(\Rightarrow\) facet-once minimizer.
\end{frame}

\section{Module 6: Simple minimizer theorem (HK2017)}

\begin{frame}{Main theorem (HK2017): a simple minimum-action orbit}
  \begin{block}{Theorem (Haim--Kislev, arXiv:1712.03494)}
    Let \(K\subset\R^{2n}\) be a convex polytope with facets \(F_i\) and outward unit normals \(n_i\).
    Then there exists a minimum-action closed characteristic \(\gamma\) on \(\partial K\) such that:
    \begin{enumerate}
      \item \(\dot\gamma\) is piecewise constant (so \(\gamma\) is polygonal);
      \item for each facet \(F_i\), the set \(\{t:\ \dot\gamma(t)\ \text{is a positive multiple of } Jn_i\}\) is connected.
    \end{enumerate}
    In particular, \(\gamma\) visits the interior of each facet at most once.
  \end{block}

  \small
  This is an \textbf{existence} statement: there may be other minimizers that are not simple.
\end{frame}

%variant: 6-A (remark about homotopies; optional)
\begin{frame}{Remark: the manipulations are homotopies}{\VariantLabel{Variant 6-A}}
  \begin{itemize}
    \item In the proof we modify a dual minimizer \(z\) by:
      approximation \(\rightarrow\) splitting \(\rightarrow\) rearrangement \(\rightarrow\) rescaling.
    \item \NeedsCheck{Claim (informal):} each modification can be done continuously (as a homotopy through admissible loops),
      so we do not ``teleport'' between disconnected minimizer components.
  \end{itemize}

  \small
  If you want, we can keep this only as a spoken comment (or remove it entirely).
\end{frame}

%variant: 6-B
\begin{frame}{Key polytope input: \(I_K\) is constant on face-cones}{\VariantLabel{Variant 6-B}}
  \begin{block}{Proposition (HK2017)}
    If for a.e.\ \(t\) we have \(\dot z(t)\in c\,\conv\{p_{j_1},\ldots,p_{j_\ell}\}\) for some non-empty face
    \(F_{j_1}\cap\cdots\cap F_{j_\ell}\neq\emptyset\),
    then \(\support_K(-J\dot z(t))=\NeedsCheck{2}c\) a.e.\ and hence \(I_K(z)=c^2\).
  \end{block}

  \small
  This is the reason rearranging velocities is possible: \(I_K\) depends on \(\dot z\) only via the scale \(c\).
\end{frame}

%variant: 6-C
\begin{frame}{Proof roadmap (dual side)}{\VariantLabel{Variant 6-C}}
  \begin{enumerate}
    \item Start with a dual minimizer \(z\) (Clarke dual action principle).
    \item Approximate by piecewise affine loops (same velocity cone condition).
    \item Split convex combinations \(\Rightarrow\) only pure facet velocities remain.
    \item Rearrange equal velocities to be in one connected time block.
    \item Rescale to satisfy the constraint again; minimality forces we did not increase \(I_K\).
    \item Take a limit \(\Rightarrow\) obtain a minimizer with the ``facet-once'' property.
    \item Translate back to a minimum-action closed characteristic \(\gamma\) on \(\partial K\).
  \end{enumerate}
\end{frame}

\begin{frame}{Action of a piecewise-constant loop (order matters)}
  \begin{itemize}
    \item Suppose \(z\) is a closed loop and \(\dot z(t)=w_i\) on intervals \(I_i=(\tau_{i-1},\tau_i)\) partitioning \([0,1]\).
  \end{itemize}

  \begin{block}{Area / action identity (HK2017)}
    \[
      \int_0^1 \inner{-J\dot z(t)}{z(t)}\,dt
      = \sum_{1\le j<i\le m} |I_i|\,|I_j|\,\omega(w_i,w_j).
    \]
  \end{block}

  \small
  So: the constraint depends on the \emph{order} of segments, even though \(I_K\) does not.
\end{frame}

%variant: 6-D (uses existing HK2017 figure)
\begin{frame}{Step 1: split convex combinations}{\VariantLabel{Variant 6-D (uses HK figure)}}
  \begin{columns}[T,onlytextwidth]
    \column{0.58\textwidth}
      \begin{itemize}
        \item On lower-dimensional faces, velocities can be convex combinations of facet velocities.
        \item We replace a mixed segment by multiple pure segments.
      \end{itemize}

      \begin{block}{Lemma (HK2017)}
        There exists a piecewise affine \(z'\) with \(\dot z'(t)\in\{v_1,\ldots,v_k\}\) a.e.\ and
        \[
          \int_0^1 \inner{-J\dot z'}{z'}\,dt \ \ge\ \int_0^1 \inner{-J\dot z}{z}\,dt.
        \]
      \end{block}

    \column{0.42\textwidth}
      \includegraphics[width=\linewidth]{assets/manual/linearcombination.pdf}
  \end{columns}
\end{frame}

%variant: 6-E hand-drawn version of the same slide
\begin{frame}{Step 1: split convex combinations}{\VariantLabel{Variant 6-E (hand-drawn)}} %variant: 6-E
  \begin{columns}[T,onlytextwidth]
    \column{0.58\textwidth}
      \begin{itemize}
        \item Replace a ``mixed'' velocity segment (convex combination) by a sequence of pure segments.
        \item Choose the order to ensure the action constraint does not decrease.
      \end{itemize}

      \begin{block}{Lemma (HK2017, informal)}
        Splitting \(\dot z\in\conv\{v_i\}\) into pure \(v_i\)-segments can be done so that
        \(\int \inner{-J\dot z}{z}\) does not decrease.
      \end{block}

    \column{0.42\textwidth}
      % Purpose: visualize “convex combination velocity” → “sequence of pure velocities”, and that reversing order flips the sign of the extra term.
      % Image suggestion: time axis with 1 mixed block labeled “a v_1 + (1-a) v_2”, then split into two blocks v_1 then v_2;
      % show also the reversed order; annotate “order chosen to increase area constraint”.
      \DraftImage{assets/manual/fig-lemma-split-convex-combination.png}{0.52\textheight}
  \end{columns}
\end{frame}

%variant: 6-F (uses existing HK2017 reorder figure)
\begin{frame}{Step 2: group equal velocities}{\VariantLabel{Variant 6-F (uses HK figure)}}
  \begin{columns}[T,onlytextwidth]
    \column{0.58\textwidth}
      \begin{itemize}
        \item After splitting, velocities lie in a finite set \(\{v_1,\ldots,v_k\}\).
        \item If the same \(v_j\) appears in disjoint intervals, merge them by rearrangement.
      \end{itemize}

      \begin{block}{Lemma (HK2017)}
        There exists a rearranged loop \(z'\) so that \(\{t:\dot z'(t)=v_j\}\) is connected for each \(j\),
        and \(\int \inner{-J\dot z'}{z'} \ge \int \inner{-J\dot z}{z}\).
      \end{block}

    \column{0.42\textwidth}
      \includegraphics[width=\linewidth]{assets/manual/reorder.pdf}
  \end{columns}
\end{frame}

%variant: 6-G hand-drawn version of the same slide
\begin{frame}{Step 2: group equal velocities}{\VariantLabel{Variant 6-G (hand-drawn)}} %variant: 6-G
  \begin{columns}[T,onlytextwidth]
    \column{0.58\textwidth}
      \begin{itemize}
        \item Reorder segments so each facet direction appears in one connected time block.
        \item Choose the reorder direction that does not decrease the area constraint.
      \end{itemize}

    \column{0.42\textwidth}
      % Purpose: show “same velocity appears twice” → “merge intervals” and that one merge direction increases the area constraint.
      % Image suggestion: time axis with v_1, v_2, v_1 blocks; show two possible merges (remove left vs remove right) and mark “choose good sign”.
      \DraftImage{assets/manual/fig-lemma-group-equal-velocities.png}{0.52\textheight}
  \end{columns}
\end{frame}

%variant: 6-H
\begin{frame}{Rescaling trick: why this preserves minimality}{\VariantLabel{Variant 6-H}}
  \begin{itemize}
    \item Starting from a dual minimizer \(z\), the lemmas produce \(\tilde z\) with:
      \begin{itemize}
        \item same ``speed scale'' \(c\) \(\Rightarrow\) same \(I_K(\tilde z)\),
        \item \(\int_0^1 \inner{-J\dot{\tilde z}}{\tilde z}\,dt \ge 1\).
      \end{itemize}
    \item Renormalize:
      \[
        \alpha := \int_0^1 \inner{-J\dot{\tilde z}}{\tilde z}\,dt,
        \qquad
        z^* := \frac{\tilde z}{\sqrt{\alpha}}.
      \]
    \item Then \(z^*\in\mathcal{E}^{HK}\) and \(I_K(z^*)=I_K(\tilde z)/\alpha \le I_K(z)\).
  \end{itemize}

  \small
  So there exists a minimizer whose facet velocities appear at most once.
\end{frame}

%variant: 6-I
\begin{frame}{Compactness: extract a limiting simple minimizer}{\VariantLabel{Variant 6-I}}
  \begin{itemize}
    \item After simplification, each loop is encoded by:
      \[
        (\sigma,(T_1,\ldots,T_F))\in S_F\times\Delta^{F-1},
      \]
      where \(\sigma\) is the facet order and \(T_i\) are (normalized) segment lengths.
    \item \(S_F\) is finite and \(\Delta^{F-1}\) is compact \(\Rightarrow\) take a convergent subsequence.
    \item The limit loop is still in \(\mathcal{E}^{HK}\) and remains a minimizer by lower-semicontinuity/continuity.
  \end{itemize}
\end{frame}

%variant: 6-J
\begin{frame}{Back to geometry: facet-once minimum-action closed characteristic}{\VariantLabel{Variant 6-J}}
  \begin{itemize}
    \item Clarke correspondence: a dual minimizer \(z^*\) produces a minimum-action closed characteristic \(\gamma\subset\partial K\)
      (up to translation/scaling).
    \item The connected-time-block property for velocities transfers to:
      \[
        \gamma \text{ visits each facet interior at most once.}
      \]
  \end{itemize}

  \begin{alertblock}{Takeaway}
    Minimum-action on a polytope admits a \emph{combinatorial} minimizer: a polygonal orbit with at most one segment per facet.
  \end{alertblock}
\end{frame}

\section{Module 7: Computational takeaway (optional)}

%variant: 7-A (purely conceptual; recommended)
\begin{frame}{Computation takeaway}{\VariantLabel{Variant 7-A}}
  \begin{itemize}
    \item After HK2017, the search space for a minimizer can be parameterized by:
      \begin{itemize}
        \item a facet order \(\sigma\) (combinatorial),
        \item segment lengths / time weights \((T_i)\) (continuous, but finite-dimensional),
      \end{itemize}
      under constraints:
      \begin{itemize}
        \item closedness (sum of segment vectors \(=0\)),
        \item area/action constraint (a quadratic constraint coming from \(\int\inner{-J\dot z}{z}\)).
      \end{itemize}
    \item This is ``almost'' a quadratic program once \(\sigma\) is fixed.
  \end{itemize}
\end{frame}

%variant: 7-B more formula-heavy (only keep if you like it)
\begin{frame}{Computation takeaway (more explicit)}{\VariantLabel{Variant 7-B}} %variant: 7-B
  \small
  For a simple loop with velocities \(p_{\sigma(i)}=\frac{2}{h_{\sigma(i)}}Jn_{\sigma(i)}\) and weights \(\beta_i\ge 0\) with
  \[
    \sum_i \beta_i h_i = 1,
    \qquad
    \sum_i \beta_i n_i = 0,
  \]
  define
  \[
    Q(\sigma,\beta)=\sum_{1\le j<i\le F}\beta_{\sigma(i)}\beta_{\sigma(j)}\,\omega(n_{\sigma(i)},n_{\sigma(j)}).
  \]
  HK2017 gives
  \[
    \cEHZ(K)=\frac{1}{2}\Big[\max_{\sigma,\beta} Q(\sigma,\beta)\Big]^{-1}.
  \]

  \small
  \NeedsCheck{If you keep this variant, I’ll align all constants with your final Clarke-duality normalization.}
\end{frame}

\section{Meta}

\begin{frame}{References (arXiv-first)}
  \footnotesize
  \begin{itemize}
    \item P.\ Haim--Kislev, \emph{On the symplectic size of convex polytopes}, arXiv:\,1712.03494 (v3).
    \item P.\ Haim--Kislev, Y.\ Ostrover, \emph{A Counterexample to Viterbo's Conjecture}, arXiv:\,2405.16513.
    \item J.\ Chaidez, M.\ Hutchings, \emph{Computational methods for testing Viterbo's conjecture via combinatorial Reeb dynamics}, arXiv:\,2008.10111.
    \item \NeedsCheck{F.\,H.\ Clarke (1979): original dual action principle reference (add exact journal info or drop).}
  \end{itemize}
\end{frame}

%variant: Z-B
\begin{frame}{Pipeline takeaway (final slide)}{\VariantLabel{Variant Z-B}}
  \begin{enumerate}
    \item Viterbo-type problems \(\Rightarrow\) need minimum-action closed characteristics \(\Rightarrow\) need \(\cEHZ(K)\).
    \item Polytopes are computationally accessible, but the primal minimization is hard.
    \item Clarke dual action principle: rewrite as a dual minimization where the functional depends only on \(\dot z\).
    \item Polytope special: \(I_K\) is insensitive to mixing/reordering facet velocities.
    \item Rearrangement \(\Rightarrow\) \textbf{existence of a facet-once minimizer} \(\Rightarrow\) finite/combinatorial search space.
  \end{enumerate}
\end{frame}

\end{document}
