% Clarke's duality principle — seminar talk slides (ephemeral)
%
% IMPORTANT (lifecycle):
% - This package is meant for fast iteration during talk prep.
% - The final "as presented" version should be an annotated git tag:
%     talk-clarke-duality-YYYY-MM-DD
% - After migrating content into the thesis, this package may be deleted from `main`.
%   In that case, use the tag/commit to retrieve the exact sources/assets.
% - Post-talk workflow: see CHECKLIST_TEARDOWN.md
%
% Build:
%   ./scripts/build.sh
%
\documentclass[10pt,aspectratio=169]{beamer}

% Theme (nice defaults, no custom tikz required in authoring)
\usetheme[progressbar=frametitle]{metropolis}

% Fonts + basic typesetting
\usepackage[T1]{fontenc}
\usepackage{lmodern}
\usepackage{microtype}

% Math
\usepackage{amsmath,amssymb,mathtools}

% Figures
\usepackage{graphicx}
\graphicspath{{assets/manual/}}

% Links
% (beamer loads hyperref; we just tune it)
\hypersetup{
  colorlinks=true,
  linkcolor=blue,
  urlcolor=blue
}

% Local macros + theorem setup
% Local preamble for a short-lived talk (keep this small).

% Common math shorthands
\newcommand{\R}{\mathbb{R}}
\newcommand{\C}{\mathbb{C}}
\newcommand{\Z}{\mathbb{Z}}
\newcommand{\Q}{\mathbb{Q}}
\newcommand{\N}{\mathbb{N}}
\newcommand{\T}{\mathbb{T}}

\newcommand{\abs}[1]{\lvert #1\rvert}
\newcommand{\norm}[1]{\lVert #1\rVert}
\newcommand{\inner}[2]{\langle #1, #2 \rangle}
\newcommand{\conv}{\operatorname{conv}}

% Symplectic shorthand (match thesis notation)
\newcommand{\cEHZ}{c_{\mathrm{EHZ}}}
\newcommand{\gauge}{g}
\newcommand{\support}{h}
\newcommand{\polar}{\circ}

% Beamer theorem blocks (numbered)
\setbeamertemplate{theorems}[numbered]


% Build metadata (generated by scripts/gen-version.sh; gitignored)
\providecommand{\TalkGitCommit}{unknown}
\providecommand{\TalkBuildDate}{\today}
\IfFileExists{version.tex}{\input{version}}{}

\setbeamertemplate{frame footer}{\scriptsize source: \texttt{\TalkGitCommit}}

\title{Clarke's Duality Principle}
\author{J\"orn St\"ohler}
\institute{Symplectic Geometry Seminar}
\date{\TalkBuildDate\quad{\scriptsize source: \texttt{\TalkGitCommit}}}

\begin{document}

\maketitle

\begin{frame}{Plan}
  \begin{enumerate}
    \item Closed characteristics on polytopes and the action
    \item Clarke's dual action principle (variational reformulation)
    \item Why polytopes reduce to finitely many velocities
    \item Rearrangement argument: a \emph{simple} minimum-action orbit
    \item (Optional) combinatorial encoding
  \end{enumerate}
\end{frame}

\section{Motivation}

\begin{frame}{Motivation}
  \begin{itemize}
    \item For a smooth convex body \(K\subset(\R^{2n},\omega)\), the Ekeland--Hofer--Zehnder capacity
      \(\cEHZ(K)\) equals the \emph{minimal action} of a closed characteristic on \(\partial K\).
    \item For polytopes, closed characteristics are \emph{generalized} (velocity can jump / slide on lower faces),
      but \(\cEHZ(K)\) is still a minimal action.
    \item Clarke's dual action principle rewrites the minimum-action problem as a minimization of a convex functional
      on a Sobolev loop space.
    \item Computational goal: replace an infinite-dimensional minimization problem by a finite/combinatorial one.
    \item Key structural fact (HK): an action-minimizing orbit can be chosen to visit each facet at most once.
  \end{itemize}
\end{frame}

\begin{frame}{What Clarke duality buys us (one slide summary)}
  \begin{itemize}
    \item \textbf{Geometric problem:} minimize the action \(A(\gamma)\) over closed characteristics \(\gamma\subset\partial K\).
    \item \textbf{Variational problem:} minimize \(I_K(z)=\tfrac14\int \support_K^2(-J\dot z)\) over a constraint set \(E\subset W^{1,2}\).
    \item For polytopes, \(I_K\) becomes \emph{insensitive} to reordering/mixing facet velocities.
    \item So we can simplify a minimizer by rearranging its velocity, and then rescale back into the constraint \(E\).
  \end{itemize}
\end{frame}

\begin{frame}{Target statement (facet-once minimizer)}
  \begin{block}{Theorem (Haim--Kislev, 2017)}
    Let \(K\subset\R^{2n}\) be a convex polytope with facets \(\{F_i\}_{i=1}^F\) and outward unit normals \(n_i\).
    Then there exists a \emph{minimum-action} closed characteristic \(\gamma\) on \(\partial K\) such that:
    \begin{enumerate}
      \item \(\dot\gamma\) is piecewise constant, hence \(\gamma\) is polygonal;
      \item for each facet \(F_i\), the set \(\{t:\ \dot\gamma(t)\ \text{is a positive multiple of } Jn_i\}\) is connected.
    \end{enumerate}
    In particular, \(\gamma\) visits the interior of each facet at most once.
  \end{block}

  \vspace{0.5em}
  \small
  Here \(J\) is the standard complex structure and \(\omega(u,v)=\inner{Ju}{v}\).
\end{frame}

\section{Setup}

\begin{frame}{Setup and notation}
  \begin{itemize}
    \item Work in \((\R^{2n},\omega)\), \(\omega(u,v)=\inner{Ju}{v}\), \(\lambda=\tfrac12\inner{Jx}{dx}\).
    \item Polytope via irredundant half-spaces:
      \[
        K=\bigcap_{i=1}^F\{x:\ \inner{x}{n_i}\le h_i\},\qquad \abs{n_i}=1,\ h_i=\support_K(n_i)>0.
      \]
    \item Facets: \(F_i = K \cap \{ \inner{x}{n_i}=h_i\}\).
    \item Facet velocities (Reeb on facet interiors):
      \[
        p_i := \frac{2}{h_i}\,J n_i.
      \]
  \end{itemize}
\end{frame}

\begin{frame}{Convex-analytic dictionary (support, gauge, polar)}
  \begin{itemize}
    \item Support function: \(\support_K(y)=\sup_{x\in K}\inner{x}{y}\).
    \item Gauge (Minkowski functional): \(\gauge_K(x)=\inf\{r>0:\ x\in rK\}\), so \(\gauge_K\equiv 1\) on \(\partial K\).
    \item Polar body: \(K^{\polar}=\{y\in\R^{2n}:\support_K(y)\le 1\}\).
  \end{itemize}

  \begin{block}{Polytope specializations}
    If \(K=\bigcap_i\{\inner{x}{n_i}\le h_i\}\) is irredundant, then \(\support_K(n_i)=h_i\),
    and \(K^{\polar}\) is the polytope with vertices \(\{n_i/h_i\}\).
  \end{block}
\end{frame}

\begin{frame}{Closed characteristics on a polytope}
  \begin{itemize}
    \item A (generalized) closed characteristic is a loop \(\gamma\in W^{1,2}(\T,\partial K)\) with
      \[
        \dot\gamma(t)\in \R_{>0}\,\conv\{p_i:\ \gamma(t)\in F_i\}
        \qquad \text{for a.e. }t.
      \]
    \item If \(\gamma\) meets the interior of \(F_i\), then (for a.e. such \(t\)) we have \(\dot\gamma(t)=c\,p_i\) for some \(c>0\),
      so \(\gamma\) runs along a \emph{straight segment} on that facet.
    \item The action of a \(T\)-periodic loop \(\gamma\) is
      \[
        A(\gamma)=\int_\gamma\lambda=\frac12\int_0^T \inner{-J\dot\gamma(t)}{\gamma(t)}\,dt.
      \]
  \end{itemize}
\end{frame}

\section{Clarke duality}

\begin{frame}{Dual action functional}
  \begin{itemize}
    \item Clarke's principle packages the dynamics into a constrained minimization problem on loops.
    \item Define the constraint set
      \[
        E=\Big\{z\in W^{1,2}([0,1],\R^{2n}):\ \int_0^1 \dot z\,dt=0,\ \int_0^1 \inner{-J\dot z}{z}\,dt=1\Big\}.
      \]
    \item Define the \emph{dual action functional}
      \[
        I_K(z)=\frac14\int_0^1 \support_K^2\!\bigl(-J\dot z(t)\bigr)\,dt.
      \]
    \item \(I_K\) depends only on the velocity \(\dot z\) (not on \(z(0)\)), and is convex in \(\dot z\).
  \end{itemize}
\end{frame}

\begin{frame}{Fenchel/Legendre duality: gauge vs support}
  \begin{itemize}
    \item For a convex body \(K\subset\R^{2n}\), define the gauge and support functions
      \[
        \gauge_K(x)=\inf\{r>0:\ x\in rK\},\qquad \support_K(y)=\sup_{x\in K}\inner{x}{y}.
      \]
    \item Their squares are Fenchel dual (up to the factor \(\tfrac14\)):
      \[
        \gauge_K^2(x) + \tfrac14\,\support_K^2(y)\ \ge\ \inner{x}{y}.
      \]
    \item Equality holds iff \(y\in\partial \gauge_K^2(x)\) (equivalently \(x\in\partial(\tfrac14\,\support_K^2)(y)\)).
  \end{itemize}

  \small
  This is the pointwise convex-analytic input behind Clarke's dual action principle and the \(z \leftrightarrow \gamma\) correspondence.
\end{frame}

\begin{frame}{Clarke dual action principle (what we use)}
  \begin{block}{Theorem (Clarke duality, specialized to convex bodies)}
    For a convex body \(K\subset\R^{2n}\),
    \[
      \cEHZ(K)=\min_{z\in E} I_K(z).
    \]
    Any minimizer \(z\) corresponds to a minimum-action closed characteristic \(\gamma\) on \(\partial K\)
    (up to translation and scaling).
  \end{block}

  \vspace{0.5em}
  \small
  For polytopes, ``corresponds'' is understood using subdifferentials / Hamiltonian inclusions.
\end{frame}

\begin{frame}{From minimizers to orbits (what “corresponds” means)}
  \begin{itemize}
    \item For smooth \(K\), critical points of \(I_K\) on \(E\) satisfy an Euler--Lagrange equation which is a Hamiltonian equation.
    \item For polytopes, one uses the Hamiltonian \(\gauge_K^2\) and replaces \(\nabla\) by the convex subdifferential \(\partial\):
      \[
        \dot\gamma(t)\in J\,\partial \gauge_K^2(\gamma(t)) \quad \text{a.e.}
      \]
    \item (HK / Abbondandolo) There is a bijection between closed characteristics \(\gamma\) on \(\partial K\) and
      critical points \(z\) of \(I_K\) on \(E\): there exist \(\lambda>0\), \(b\in\R^{2n}\) with
      \[
        z=\lambda\,\gamma + b,
        \qquad
        A(\gamma)=2\,I_K(z).
      \]
  \end{itemize}

  \small
  So: proving a structural statement for minimizers of \(I_K\) gives the same structural statement for minimum-action orbits.
\end{frame}

\begin{frame}{Why polytopes are special for \(I_K\)}
  \begin{itemize}
    \item On a polytope, \(\support_K\) is piecewise linear, so \(\support_K^2\) is piecewise quadratic.
    \item Crucial constant-speed property (HK): if \(\dot z(t)\) stays in a velocity cone of a face, then \(I_K\) is \emph{constant}.
  \end{itemize}

  \begin{block}{Key observation (informal)}
    If for a.e.\ \(t\) we have \(\dot z(t)\in c\,\conv\{p_{j_1},\ldots,p_{j_\ell}\}\) for some face \(F_{j_1}\cap\cdots\cap F_{j_\ell}\neq\emptyset\),
    then \(\support_K(-J\dot z(t))=2c\) a.e., hence \(I_K(z)=c^2\).
  \end{block}

  \small
  So: \(I_K\) does \emph{not} care how we mix/reorder facet velocities, as long as we keep the same scale \(c\).
\end{frame}

\section{Proof idea (simple minimizer)}

\begin{frame}{Step 0: reduce to piecewise affine loops}
  \begin{itemize}
    \item Let \(z\in E\) minimize \(I_K\). By the correspondence, \(\dot z(t)\) lies in a face cone:
      \(\dot z(t)\in c\,\conv\{p_i:\ \text{incident facets at }t\}\) for some \(c>0\).
    \item Approximate \(z\) in \(W^{1,2}\) by piecewise affine loops \(\zeta_N\) with the \emph{same velocity constraint}
      \(\dot\zeta_N(t)\in c\,\conv\{p_1,\ldots,p_F\}\) a.e.
    \item Goal: from each \(\zeta_N\), build a new loop that uses each \(p_i\) at most once, without increasing \(I_K\).
  \end{itemize}

  \small
  This is where the rearrangement lemmas enter: they modify the velocity \(\dot z\) while keeping the average \(\int\dot z=0\).
\end{frame}

\begin{frame}{Action of a piecewise-constant loop}
  \begin{itemize}
    \item For the proof we approximate minimizers by loops with piecewise constant velocity.
    \item Suppose \(\dot z(t)=w_i\) on intervals \(I_i=(\tau_{i-1},\tau_i)\) (a partition of \([0,1]\)).
  \end{itemize}

  \begin{block}{Area / action formula}
    \[
      \int_0^1 \inner{-J\dot z(t)}{z(t)}\,dt
      = \sum_{1\le j<i\le m} |I_i|\,|I_j|\,\omega(w_i,w_j).
    \]
  \end{block}

  \small
  Consequence: the constraint \(\int\inner{-J\dot z}{z}=1\) \emph{does} depend on the \emph{order} of the velocities.
\end{frame}

\begin{frame}{Step 1: break convex combinations (``linear combination'' lemma)}
  \begin{columns}[T,onlytextwidth]
    \column{0.62\textwidth}
      \begin{itemize}
        \item Problem: on lower-dimensional faces, we may have \(\dot z(t)\) as a convex combination of facet velocities.
        \item Fix \(v_1,\ldots,v_k\in\R^{2n}\). If \(\dot z(t)\in\conv\{v_1,\ldots,v_k\}\), replace a mixed interval by several pure ones.
      \end{itemize}

      \begin{block}{Lemma (HK)}
        There exists a piecewise-affine \(z'\) with \(\dot z'(t)\in\{v_1,\ldots,v_k\}\) a.e.\ and
        \[
          \int_0^1 \inner{-J\dot z'}{z'}\,dt \ \ge\ \int_0^1 \inner{-J\dot z}{z}\,dt.
        \]
      \end{block}

      \small
      Sketch: splitting adds an ``extra'' term \(\sum_{r < s} a_r a_s\,\omega(v_s,v_r)\); reversing the order flips its sign, so choose the better order.

    \column{0.38\textwidth}
      \vspace{-0.25em}
      \begin{center}
        \includegraphics[width=\textwidth]{linearcombination.pdf}

        \scriptsize Break a convex combination into pure segments.
      \end{center}
  \end{columns}
\end{frame}

\begin{frame}{Step 2: group equal velocities (``one-speed'' lemma)}
  \begin{columns}[T,onlytextwidth]
    \column{0.62\textwidth}
      \begin{itemize}
        \item After Step 1, velocities lie in a \emph{finite} set \(\{v_1,\ldots,v_k\}\).
        \item If the same velocity \(v_j\) appears in multiple disjoint time intervals, make those intervals adjacent.
      \end{itemize}

      \begin{block}{Lemma (HK)}
        There exists a rearranged piecewise-affine loop \(z'\) such that for every \(j\),
        \(\{t:\dot z'(t)=v_j\}\) is connected, and
        \[
          \int_0^1 \inner{-J\dot z'}{z'}\,dt \ \ge\ \int_0^1 \inner{-J\dot z}{z}\,dt.
        \]
      \end{block}

      \small
      Sketch: merging two equal-velocity intervals can be done in two directions; the action changes by quantities of opposite sign, so one direction is non-decreasing.

    \column{0.38\textwidth}
      \vspace{-0.25em}
      \begin{center}
        \includegraphics[width=\textwidth]{reorder.pdf}

        \scriptsize Rearrange segments to group equal velocities.
      \end{center}
  \end{columns}
\end{frame}

\begin{frame}{Scaling trick: why this produces a minimizer}
  \begin{itemize}
    \item Start from a minimizer \(z\in E\) of \(I_K\).
    \item Apply Step 1 + Step 2 to get \(\tilde z\) with the \emph{same} ``speed scale'' \(c\), hence the same \(I_K(\tilde z)\),
      but with \(\int\inner{-J\dot{\tilde z}}{\tilde z}\) \emph{not smaller}.
  \end{itemize}

  \begin{block}{Renormalize back to \(E\)}
    Set \(A(\tilde z)=\int_0^1\inner{-J\dot{\tilde z}}{\tilde z}\,dt\) and \(z^*=\tilde z/\sqrt{A(\tilde z)}\).
    Then \(z^*\in E\) and
    \[
      I_K(z^*)=\frac{I_K(\tilde z)}{A(\tilde z)}\ \le\ I_K(z).
    \]
  \end{block}

  \small
  Therefore there exists an \(I_K\)-minimizer whose velocity uses each facet velocity at most once.
\end{frame}

\begin{frame}{Step 3: pass to a limiting simple minimizer}
  \begin{itemize}
    \item After Step 1+2, each approximant can be written as
      \[
        \dot z_N(t)=c_N\,p_{\sigma_N(i)} \quad \text{on disjoint intervals }I^N_i,
      \]
      where each facet velocity \(p_i\) appears at most once.
    \item Such a loop is encoded by \((\sigma_N,(|I^N_1|,\ldots,|I^N_F|))\in S_F\times\Delta^{F-1}\)
      (a permutation + a point in the simplex of time-lengths).
    \item Since \(S_F\) is finite and \(\Delta^{F-1}\) is compact, take a convergent subsequence:
      \(\sigma_N=\sigma\) eventually and \(|I^N_i|\to T_i\).
    \item The limit yields a loop \(z^*\in E\) with the same structural properties and \(I_K(z^*)=\min_E I_K\).
  \end{itemize}

  \small
  Finally, apply the correspondence \(z^* \leftrightarrow \gamma\) to obtain a minimum-action orbit that visits each facet at most once.
\end{frame}

\begin{frame}{(Optional) combinatorial encoding (HK formula)}
  \begin{itemize}
    \item A simple minimizer has velocities \(c\,p_{\sigma(i)}\) on intervals of length \(T_i\ge0\).
    \item Constraints become:
      \[
        \sum_{i=1}^F T_i\,p_{\sigma(i)}=0
        \quad\Longleftrightarrow\quad
        \sum_{i=1}^F \beta_i\,n_i=0
        \quad\text{with}\quad
        \beta_{\sigma(i)}:=\frac{T_i}{h_{\sigma(i)}}.
      \]
    \item Action normalization gives a quadratic form in \(\beta\):
      \[
        Q(\sigma,\beta)=\sum_{1\le j<i\le F}\beta_{\sigma(i)}\beta_{\sigma(j)}\,\omega(n_{\sigma(i)},n_{\sigma(j)}).
      \]
  \end{itemize}

  \begin{block}{Result}
    \[
      \cEHZ(K)=\frac{1}{2}\Big[\max_{\sigma,\beta} Q(\sigma,\beta)\Big]^{-1}
      \qquad
      (\beta_i\ge0,\ \sum\beta_i h_i=1,\ \sum\beta_i n_i=0).
    \]
  \end{block}
\end{frame}

\begin{frame}{Remarks / caveats}
  \begin{itemize}
    \item The theorem is an \emph{existence} statement: there may be other minimum-action orbits that revisit facets.
    \item Degeneracies are common when \(K\) has Lagrangian 2-faces (velocity can slide in a cone).
    \item Example: for the standard simplex in \(\R^4\), an action minimizer can switch between facet directions repeatedly on a Lagrangian face,
      but there still exists a minimizer with the “facet-once” property.
  \end{itemize}
\end{frame}

\begin{frame}{Conclusion: a simple minimum-action orbit}
  \begin{itemize}
    \item By Clarke's correspondence, a minimizer \(z^*\in E\) yields a minimum-action closed characteristic \(\gamma\) on \(\partial K\).
    \item For polytopes, \(\dot\gamma\) is piecewise constant with values \(c\,p_i\).
    \item The connectivity property for \(\{t:\dot z^*(t)=c\,p_i\}\) translates to:
      \(\gamma\) meets each facet interior at most once.
  \end{itemize}

  \begin{alertblock}{Main takeaway}
    The minimum-action problem admits a \emph{combinatorial} minimizer: a polygonal orbit with \(\le F\) segments, one per facet at most.
  \end{alertblock}
\end{frame}

\section{Meta}

\begin{frame}{References (minimal)}
  \footnotesize
  \begin{itemize}
    \item F.\,H.\ Clarke (1979): \emph{A classical variational principle for periodic Hamiltonian trajectories}.
    \item A.\ Abbondandolo, P.\ Majer (2015): \emph{A non-squeezing theorem for convex symplectic images of the Hilbert ball}.
    \item P.\ Haim--Kislev (2017): \emph{On the symplectic size of convex polytopes} (arXiv:1712.03494).
    \item S.\ Artstein-Avidan, Y.\ Ostrover (2014): \emph{Bounds for Minkowski billiard trajectories in convex bodies}.
  \end{itemize}
\end{frame}

\begin{frame}{Build info}
  \small
  \begin{itemize}
    \item Build date: \TalkBuildDate
    \item Source commit: \texttt{\TalkGitCommit}
  \end{itemize}
\end{frame}

\end{document}
