\chapter{Symplectic and convex-analytic background}
\label{chap:background}

This chapter fixes notation and conventions, and recalls the analytic and symplectic background used in the rest of the thesis.
We work exclusively in real dimension~$4$, i.e.\ complex dimension~$2$.
Proofs are mostly omitted or reduced to brief reminders; the primary goal is to make precise what objects we work with and which conventions we follow.

Throughout, all manifolds, maps and differential forms are smooth (i.e.\ of class~$C^\infty$) unless explicitly stated otherwise.

\section{Global conventions}
\label{sec:conventions}

\subsection{Linear algebra and symplectic structure on \texorpdfstring{$\R^4$}{R4}}

We identify $\R^4$ with $\C^2$ by
\begin{equation*}
  (x_1,x_2,y_1,y_2) \longleftrightarrow (z_1,z_2)
  = (x_1 + i y_1,\; x_2 + i y_2).
\end{equation*}
The standard symplectic form is
\begin{equation*}
  \omega_0 = dx_1 \wedge dy_1 + dx_2 \wedge dy_2,
\end{equation*}
and the standard Liouville form is
\begin{equation*}
  \lambda_0 = \tfrac12(x_1\,dy_1 - y_1\,dx_1 + x_2\,dy_2 - y_2\,dx_2).
\end{equation*}
The associated complex structure $J$ is given by
\begin{equation*}
  J\bigl(\partial_{x_j}\bigr) = \partial_{y_j},\qquad
  J\bigl(\partial_{y_j}\bigr) = -\,\partial_{x_j},
\end{equation*}
so that $\omega_0(v,w) = \langle Jv, w\rangle$, where $\langle\cdot,\cdot\rangle$ is the Euclidean scalar product on~$\R^4$.

We use the following notation:
\begin{itemize}
  \item $\R^4$ always carries the fixed structure $(\omega_0,\lambda_0,J,\langle\cdot,\cdot\rangle)$.
  \item $|v|$ denotes the Euclidean norm of $v\in\R^4$.
  \item For $r>0$, the closed ball and the standard cylinder are
  \begin{equation*}
    B(r) = \{z\in\C^2 \mid \pi|z|^2 \le r\},\qquad
    Z(r) = \{z\in\C^2 \mid \pi|z_1|^2 \le r\}.
  \end{equation*}
\end{itemize}

\subsection{Convex bodies, polytopes and faces}

\begin{definition}[Convex bodies and polytopes]\label{def:convex-body-polytope}
A \emph{convex body} $K\subset\R^4$ is a compact, convex, non-empty subset with non-empty interior.
A \emph{polytope} is a convex body that can be written as a finite intersection of closed half-spaces.
\end{definition}

We will always assume that our polytopes are full-dimensional (i.e.\ their affine span is all of $\R^4$).

\begin{definition}[Faces]\label{def:faces}
Let $K\subset\R^4$ be a polytope.
A non-empty subset $F\subset K$ is a \emph{face} of $K$ if there exists an affine linear functional $\ell:\R^4\to\R$ and a constant $c\in\R$ such that
\begin{equation*}
  K \subset \{\ell\le c\},\qquad
  F = K \cap \{\ell = c\}.
\end{equation*}
If $\dim F = k$, we call $F$ a \emph{$k$-face} of $K$.
A $3$-face is often called a \emph{facet} of $K$.
\end{definition}

\begin{itemize}
  \item We only call non-empty sets faces.
        This is important later when faces appear as outputs of algorithms and theorems; it keeps track of non-degeneracy conditions that would otherwise have to be added everywhere.
  \item Unless explicitly stated otherwise, ``face'' always means a face of the polytope $K$ we are currently working with.
\end{itemize}

\begin{definition}[Admissible polytopes]\label{def:admissible-polytope}
A polytope $K\subset\R^4$ is \emph{admissible} if
\begin{enumerate}
  \item $K$ is convex, compact, and full-dimensional,
  \item the origin lies in the interior: $0\in\mathrm{int}(K)$,
  \item $K$ is star-shaped with respect to the origin (automatic from convexity and $0\in\mathrm{int}(K)$, but we keep the terminology for comparison with the smooth literature),
  \item $\partial K$ is a topological $3$-sphere.
\end{enumerate}
\end{definition}

The last item holds for all full-dimensional convex polytopes in~$\R^4$ and is included only to clarify the intended setting.

\begin{definition}[Lagrangian and symplectic faces]\label{def:lagrangian-face}
Let $F$ be a $k$-face of a polytope $K$.
\begin{itemize}
  \item If $\dim F = 2$, we say that $F$ is a \emph{Lagrangian $2$-face} if $\omega_0|_{T F} = 0$, and a \emph{symplectic $2$-face} if $\omega_0|_{T F}$ is non-degenerate.
  \item For higher-dimensional faces we will not need a special terminology; only the $2$-dimensional case is subtle in dimension~$4$.
\end{itemize}
\end{definition}

\begin{definition}[Non-Lagrangian and general admissible polytopes]\label{def:non-lagrangian-admissible}
An admissible polytope $K\subset\R^4$ is \emph{non-Lagrangian} if none of its $2$-faces is Lagrangian.
We refer to such polytopes as \emph{non-Lagrangian admissible polytopes}.
\end{definition}

\begin{remark}[Comparison with \cite{ChaidezHutchings}]
What we call a non-Lagrangian admissible polytope is called a \emph{symplectic polytope} in \cite{ChaidezHutchings}.
We reserve the word ``admissible'' for the general class which may have Lagrangian $2$-faces; the restriction ``non-Lagrangian'' will be stated explicitly when needed.
\end{remark}

\subsection{Terminology for periodic orbits}\label{subsec:orbit-terminology}

There are several equivalent ways to describe the periodic dynamics on a contact type hypersurface in~$\R^4$.
We fix the following terminology.

\begin{definition}[Periodic objects]\label{def:periodic-objects}
Let $\Sigma\subset\R^4$ be a smooth hypersurface of contact type (see Definition~\ref{def:contact-type}).
\begin{enumerate}
  \item A \emph{parameterized closed characteristic} on $\Sigma$ is a smooth map
    \begin{equation*}
      \gamma:\R/T\Z\to\Sigma,\qquad T>0,
    \end{equation*}
    such that $\dot\gamma(t)$ spans the characteristic line distribution $\ker\bigl(\omega_0|_{T\Sigma}\bigr)$ for all~$t$.
    We always exclude the constant map and require $T>0$.
  \item A \emph{closed characteristic} is an equivalence class of parameterized closed characteristics modulo orientation-preserving reparametrization of~$\R/T\Z$.
  \item A \emph{Reeb orbit} for a contact form $\lambda$ on $\Sigma$ is a parameterized closed orbit of the Reeb vector field $R_\lambda$, see Definition~\ref{def:reeb}.
  \item A \emph{Hamiltonian orbit} on a regular level set $H^{-1}(c)$ of a Hamiltonian $H:\R^4\to\R$ is a parameterized periodic solution of the Hamiltonian ODE
    \begin{equation*}
      \dot x(t) = X_H(x(t)) := J\nabla H(x(t)).
    \end{equation*}
\end{enumerate}
\end{definition}

The following standard lemma (proof omitted) relates these notions.

\begin{lemma}[Equivalence of periodic descriptions]\label{lem:equivalence-periodic-objects}
Let $X\subset\R^4$ be a compact domain with smooth boundary $\Sigma=\partial X$ such that $\Sigma$ is star-shaped with respect to the origin.
Let $\lambda = \lambda_0|_\Sigma$.
\begin{enumerate}
  \item $\lambda$ is a contact form on $\Sigma$, and the Reeb vector field $R_\lambda$ is everywhere tangent to the characteristic line distribution $\ker(\omega_0|_{T\Sigma})$.
  \item Every non-constant Reeb orbit on $\Sigma$ is a parameterized closed characteristic.
  \item Conversely, every parameterized closed characteristic arises as a reparametrized Reeb orbit.
  \item If $H:\R^4\to\R$ is any smooth Hamiltonian for which $\Sigma$ is a regular level set and whose Hamiltonian vector field $X_H$ is nowhere tangent to the radial direction on~$\Sigma$, then the Hamiltonian orbits on~$\Sigma$ coincide with the Reeb orbits, up to a reparametrization of time.
\end{enumerate}
In particular, on such a hypersurface we may freely identify closed characteristics, Reeb orbits and Hamiltonian orbits, whenever the precise parametrization is irrelevant.
\end{lemma}

When we want to emphasize that the parametrization (and hence the period) matters, we speak of a \emph{parameterized} Reeb orbit or closed characteristic.
Otherwise we speak simply of a \emph{closed characteristic} or \emph{Reeb orbit}.

\section{Smooth star-shaped hypersurfaces and Reeb dynamics}
\label{sec:smooth-reeb-basics}

\subsection{Contact type hypersurfaces and Reeb vector field}

\begin{definition}[Star-shaped and contact type]\label{def:contact-type}
A smooth hypersurface $\Sigma\subset\R^4$ is \emph{star-shaped with respect to the origin} if the radial vector field
\begin{equation*}
  Y(x) = \tfrac12\,x
\end{equation*}
is everywhere transverse to $\Sigma$, i.e.\ $\langle x, n(x)\rangle\neq 0$ for all $x\in\Sigma$, where $n(x)$ is a normal vector.
Equivalently, $\Sigma$ is the boundary of some compact star-shaped domain $X\subset\R^4$ with $0\in\mathrm{int}(X)$.

The $1$-form $\lambda_0$ restricts to a $1$-form $\lambda := \lambda_0|_\Sigma$ on $\Sigma$.
We say that $\Sigma$ is of \emph{contact type} (for $\lambda_0$) if $\lambda\wedge d\lambda$ never vanishes on $\Sigma$.
In this case $(\Sigma,\lambda)$ is a contact manifold.
\end{definition}

\begin{definition}[Reeb vector field]\label{def:reeb}
Given a contact form $\lambda$ on a $3$-manifold $\Sigma$, the \emph{Reeb vector field} $R_\lambda$ is uniquely characterized by
\begin{equation*}
  \lambda(R_\lambda) = 1,\qquad
  d\lambda(R_\lambda,\cdot) = 0.
\end{equation*}
A \emph{Reeb orbit} is a periodic orbit of $R_\lambda$, i.e.\ a map
$\gamma:\R/T\Z\to\Sigma$ for some $T>0$ such that $\dot\gamma(t)=R_\lambda(\gamma(t))$.
\end{definition}

For hypersurfaces of contact type in $(\R^4,\omega_0,\lambda_0)$ one has the following basic facts (see e.g.\ \cite{HoferZehnder,HoferWysockiZehnder,ChaidezHutchings}):
\begin{itemize}
  \item Every star-shaped hypersurface $\Sigma$ is of contact type for~$\lambda_0$.
  \item Every such $\Sigma$ admits at least one Reeb orbit.
\end{itemize}

\subsection{Symplectic action and period}

\begin{definition}[Symplectic action]\label{def:action}
Let $\Sigma\subset\R^4$ be star-shaped with respect to the origin and let $\lambda=\lambda_0|_\Sigma$.
For a parameterized closed characteristic $\gamma:\R/T\Z\to\Sigma$ we define its \emph{symplectic action} by
\begin{equation}\label{eq:def-action}
  \mathcal{A}(\gamma)
  := \int_{\R/T\Z} \gamma^*\lambda_0
  = \int_0^T \lambda_0\bigl(\dot\gamma(t)\bigr)\,dt.
\end{equation}
\end{definition}

\begin{lemma}[Action equals period]\label{lem:action-equals-period}
Let $\Sigma\subset\R^4$ be star-shaped, $\lambda=\lambda_0|_\Sigma$ and $\gamma$ a Reeb orbit of $R_\lambda$ with period~$T$.
Then $\mathcal{A}(\gamma)=T$.
\end{lemma}

\begin{proof}[Proof sketch]
By definition of the Reeb vector field, $\lambda(R_\lambda)=1$.
Thus for a Reeb orbit $\dot\gamma(t)=R_\lambda(\gamma(t))$ we have
\begin{equation*}
  \mathcal{A}(\gamma)
  = \int_0^T \lambda(\dot\gamma(t))\,dt
  = \int_0^T 1\,dt
  = T.
\end{equation*}
\end{proof}

\begin{definition}[Minimal action]\label{def:minimal-action}
Let $X\subset\R^4$ be a compact domain with smooth star-shaped boundary $\Sigma=\partial X$.
The \emph{minimal action} of $X$ is
\begin{equation*}
  A_{\min}(X)
  := \inf\{\mathcal{A}(\gamma)\mid \gamma \text{ is a non-constant Reeb orbit on }\Sigma\}.
\end{equation*}
By Lemma~\ref{lem:action-equals-period}, this is the same as the infimum of periods of non-constant Reeb orbits.
\end{definition}

Later we will consider the minimal action for convex polytopes, using generalized closed characteristics; see Section~\ref{sec:generalized-closed-char}.

\subsection{Rotation number and Conley--Zehnder index in dimension four}\label{subsec:rotation-cz-smooth}

On a contact $3$-manifold $(\Sigma,\lambda)$, the contact structure $\xi=\ker\lambda$ is a rank-$2$ symplectic vector bundle, and the linearized Reeb flow acts by symplectic linear maps on the fibers of~$\xi$.
Using a symplectic trivialization of $\xi$ along a Reeb orbit $\gamma$ one can measure the rotation rate of the linearized flow and obtain the Conley--Zehnder index.

We recall only the bare minimum needed later; see for instance \cite{HoferZehnder,LongBook,ChaidezHutchings} for details.

\begin{definition}[Rotation number and Conley--Zehnder index]\label{def:rot-cz}
Let $\gamma:\R/T\Z\to\Sigma$ be a Reeb orbit on a $3$-dimensional contact manifold $(\Sigma,\lambda)$ and assume that $\xi=\ker\lambda$ is symplectically trivial along~$\gamma$.
\begin{enumerate}
  \item A \emph{symplectic trivialization} along $\gamma$ is a bundle isomorphism $\tau:\xi|_{\mathrm{im}(\gamma)}\to (\R/T\Z)\times\R^2$ such that each fiber map is a symplectic linear isomorphism.
  \item The linearized Reeb flow along $\gamma$ gives a path of symplectic matrices $\Phi_\tau:[0,T]\to\mathrm{Sp}(2)$ with $\Phi_\tau(0)=\mathbbm{1}$.
        Its lift to the universal cover $\widetilde{\mathrm{Sp}}(2)$ has a well-defined \emph{rotation number} $\rho(\gamma)\in\R$ (independent of $\tau$ in our setting).
  \item If the linearized return map $\Phi_\tau(T)$ has no eigenvalue equal to~$1$, the orbit is called \emph{non-degenerate}.
        In that case the Conley--Zehnder index of~$\gamma$ is defined by
        \begin{equation}\label{eq:cz-from-rotation}
          \mathrm{CZ}(\gamma) = \lfloor \rho(\gamma)\rfloor + \lceil \rho(\gamma)\rceil \in\Z.
        \end{equation}
\end{enumerate}
\end{definition}

For convex domains in $\R^4$ we will need the following classical result; see e.g.\ \cite{HoferWysockiZehnder,HuLong,AbbondandoloKangIrie,ChaidezHutchings}.

\begin{proposition}[Indices for convex energy surfaces in \texorpdfstring{$\R^4$}{R4}]\label{prop:convex-index-bounds}
Let $X\subset\R^4$ be a compact strictly convex domain with smooth boundary $\Sigma=\partial X$ and $0\in\mathrm{int}(X)$.
Then:
\begin{enumerate}
  \item Every Reeb orbit $\gamma$ on~$\Sigma$ has rotation number $\rho(\gamma)>1$.
        In particular, if $\gamma$ is non-degenerate then $\mathrm{CZ}(\gamma)\ge 3$.
  \item There exists a Reeb orbit $\gamma$ on~$\Sigma$ with $\mathcal{A}(\gamma)=A_{\min}(X)$ and $\rho(\gamma)\le 2$.
        If $\gamma$ is non-degenerate then $\rho(\gamma)<2$ and hence $\mathrm{CZ}(\gamma)=3$.
\end{enumerate}
\end{proposition}

\begin{remark}
The second item is the index-theoretic characterization of the Ekeland--Hofer--Zehnder capacity used in \cite{ArtsteinAvidanOstrover,ChaidezHutchings}.
The restriction to dimension four is crucial for the simple formula~\eqref{eq:cz-from-rotation}.
\end{remark}

\section{Symplectic capacities and Viterbo's conjecture}
\label{sec:capacities-viterbo}

\subsection{Symplectic capacities}

We recall the standard axioms of a symplectic capacity in the restricted setting we need.

\begin{definition}[Symplectic capacities on $\R^4$]\label{def:symplectic-capacity}
A \emph{symplectic capacity} on $\R^4$ is a map
\begin{equation*}
  c : \mathcal{D} \longrightarrow [0,\infty],
\end{equation*}
defined on some class $\mathcal{D}$ of subsets of $(\R^4,\omega_0)$, such that for all $(X,\omega_0),(X',\omega_0)\in\mathcal{D}$:
\begin{enumerate}
  \item \emph{Monotonicity:} if there exists a symplectic embedding
        $\varphi:(X,\omega_0)\hookrightarrow(X',\omega_0)$, then $c(X)\le c(X')$.
  \item \emph{Conformality:} for $r>0$ one has $c(rX) = r^2 c(X)$.
  \item \emph{Non-triviality:} for all $r>0$,
        \begin{equation*}
          0 < c\bigl(B(r)\bigr) < \infty.
        \end{equation*}
\end{enumerate}
A capacity $c$ is \emph{normalized} if it is defined on all compact convex domains in~$\R^4$ and satisfies
\begin{equation*}
  c\bigl(B(r)\bigr) = c\bigl(Z(r)\bigr) = r.
\end{equation*}
\end{definition}

Classical examples of normalized capacities are the Gromov width $c_{\mathrm{Gr}}$ and the Ekeland--Hofer--Zehnder capacity $c_{\mathrm{EHZ}}$, see e.g.\ \cite{Gromov,CieliebakHoferLatschevSchlenk,HoferZehnder,ArtsteinAvidanOstrover,ChaidezHutchings}.

\subsection{The Ekeland--Hofer--Zehnder capacity}\label{subsec:ehz-smooth}

For strictly convex star-shaped domains in $\R^4$ the Ekeland--Hofer--Zehnder capacity can be characterized dynamically.

\begin{definition}[EHZ capacity for smooth convex domains]\label{def:ehz-smooth}
Let $X\subset\R^4$ be a compact convex domain with smooth boundary $\Sigma=\partial X$ and $0\in\mathrm{int}(X)$.
The Ekeland--Hofer--Zehnder capacity of~$X$ is defined by
\begin{equation*}
  c_{\mathrm{EHZ}}(X) := A_{\min}(X).
\end{equation*}
\end{definition}

For smooth convex domains this agrees with the capacity defined via Hamiltonian dynamics by Ekeland, Hofer and Zehnder; see \cite[Thm.~2.2]{ArtsteinAvidanOstrover} and the original references therein.
The function $c_{\mathrm{EHZ}}$ extends continuously (with respect to the Hausdorff metric) from smooth convex domains to all convex bodies.
We will later recall explicit formulas for $c_{\mathrm{EHZ}}$ on polytopes due to Haim--Kislev \cite{HaimKislev}.

\subsection{Viterbo's conjecture and systolic ratio}\label{subsec:viterbo}

We recall the weak and strong forms of Viterbo's conjecture in dimension~$4$; see e.g.\ \cite{Viterbo,ChaidezHutchings}.

\begin{conjecture}[Weak Viterbo conjecture in dimension $4$]\label{conj:weak-viterbo}
Let $X\subset\R^4$ be a compact convex domain with smooth boundary and $0\in\mathrm{int}(X)$.
Then
\begin{equation}\label{eq:weak-viterbo}
  \mathrm{sys}(X) := \frac{c_{\mathrm{EHZ}}(X)^2}{2!\,\mathrm{vol}(X)} \;\le\; 1,
\end{equation}
with equality if and only if $X$ is a (possibly translated) Euclidean ball.
\end{conjecture}

The quantity $\mathrm{sys}(X)$ is the \emph{systolic ratio} of $X$.
It is invariant under translations and linear symplectomorphisms and is normalized so that $\mathrm{sys}(B(r))=1$.
The conjecture thus says that among all convex bodies of fixed volume, the ball maximizes the minimal action of a closed characteristic.

\begin{conjecture}[Strong Viterbo conjecture in dimension $4$]\label{conj:strong-viterbo}
All normalized symplectic capacities agree on convex bodies in~$\R^4$.
Equivalently, for every convex body $X$ and every normalized capacity $c$ we have
\begin{equation*}
  c(X) = c_{\mathrm{Gr}}(X) = c_{\mathrm{EHZ}}(X).
\end{equation*}
\end{conjecture}

The strong conjecture implies the weak one, because for convex $X$,
\begin{equation*}
  c_{\mathrm{EHZ}}(X)^2
  = c_{\mathrm{Gr}}(X)^2
  \le 2!\,\mathrm{vol}(X),
\end{equation*}
where the inequality follows from nonsqueezing and the volume of the embedded ball; see \cite{ChaidezHutchings}.

\begin{remark}[Connection to Mahler's conjecture]\label{rem:mahler}
As shown by Artstein-Avidan, Karasev and Ostrover \cite{ArtsteinAvidanKarasevOstrover}, the weak Viterbo conjecture implies the Mahler conjecture for centrally symmetric convex bodies.
For our purposes it suffices to remember that large systolic ratios are geometrically rigid and are conjecturally realized only by ellipsoids (and products of intervals in suitable senses).
\end{remark}

For convex polytopes $K\subset\R^4$ one can equivalently formulate the weak Viterbo conjecture with $c_{\mathrm{EHZ}}(K)$ defined via generalized closed characteristics; see Section~\ref{sec:generalized-closed-char}.

\section{Convex-analytic tools: gauge, support and polarity}
\label{sec:gauge-support}

We now recall the basic convex-analytic functions associated to a convex body $K$ that will be used in the dual action formulation of $c_{\mathrm{EHZ}}(K)$ and in the definition of our Hamiltonian.

\subsection{Gauge and support functions}

\begin{definition}[Gauge function]\label{def:gauge}
Let $K\subset\R^4$ be a convex body with $0\in\mathrm{int}(K)$.
The \emph{gauge function} (or Minkowski functional) of~$K$ is
\begin{equation*}
  g_K(x) := \inf\{\lambda>0 \mid x\in \lambda K\},\qquad x\in\R^4.
\end{equation*}
Equivalently, $g_K(x)$ is the unique $\lambda>0$ such that $x\in\lambda\partial K$ for $x\neq 0$, and $g_K(0)=0$.
\end{definition}

Basic properties (proofs omitted):
\begin{itemize}
  \item $g_K$ is convex and positively $1$-homogeneous: $g_K(\lambda x) = \lambda g_K(x)$ for $\lambda\ge 0$.
  \item $g_K(x)\le 1$ if and only if $x\in K$.
  \item $g_K(x)=1$ for all $x\in\partial K$.
\end{itemize}

\begin{definition}[Support function]\label{def:support}
For a convex body $K\subset\R^4$, the \emph{support function} $h_K:\R^4\to\R$ is
\begin{equation*}
  h_K(y) := \sup_{x\in K} \langle x, y\rangle,\qquad y\in\R^4.
\end{equation*}
\end{definition}

Again, $h_K$ is convex and positively $1$-homogeneous.
Geometrically, $h_K(y)$ is the signed distance of the supporting hyperplane with outer normal~$y$ from the origin.

\begin{definition}[Polar body]\label{def:polar-body}
The \emph{polar body} of $K$ is
\begin{equation*}
  K^\circ := \{y\in\R^4 \mid \langle x,y\rangle \le 1\;\text{for all }x\in K\}.
\end{equation*}
\end{definition}

One checks that $K^\circ$ is again a convex body with $0\in\mathrm{int}(K^\circ)$, and that
\begin{equation*}
  g_K(x) = h_{K^\circ}(x),\qquad h_K(y) = g_{K^\circ}(y).
\end{equation*}

\subsection{Fenchel duality between \texorpdfstring{$g_K^2$}{gK2} and \texorpdfstring{$h_K^2$}{hK2}}\label{subsec:fenchel}

Let $f:\R^4\to\R\cup\{+\infty\}$ be a proper convex function.
Its Legendre--Fenchel transform (convex conjugate) is
\begin{equation*}
  f^*(y) := \sup_{x\in\R^4} \bigl(\langle x,y\rangle - f(x)\bigr).
\end{equation*}

\begin{lemma}[Duality for gauge and support]\label{lem:fenchel-gauge-support}
Let $K\subset\R^4$ be a convex body with $0\in\mathrm{int}(K)$.
Define
\begin{equation*}
  F(x) := \tfrac14\,g_K(x)^2,\qquad
  G(y) := h_K(y)^2.
\end{equation*}
Then $F$ and $G$ are Fenchel conjugates of each other:
\begin{equation*}
  F^* = G,\qquad G^* = F.
\end{equation*}
\end{lemma}

\begin{proof}[Proof idea]
Both $F$ and $G$ are convex, even and $2$-homogeneous, and $F$ is the indicator transform of $K^\circ$ composed with a quadratic function.
The equality $F^*=G$ is a standard fact in convex geometry; see for instance \cite{ArtsteinAvidanOstrover,HaimKislev}.
The second identity follows from $F^{**}=F$.
\end{proof}

For smooth, strictly convex $K$ this duality identifies gradients:
\begin{equation*}
  y\in\partial F(x) \quad\Longleftrightarrow\quad x\in\partial G(y),
\end{equation*}
where $\partial$ denotes the convex subdifferential.
For polytopes $K$ the functions $g_K$ and $h_K$ are only piecewise differentiable, and the subdifferential is a non-trivial face (a whole cone) along directions corresponding to faces of~$K$.
Tracking this degeneracy will be important later when we discuss generalized Hamiltonian dynamics on~$\partial K$.

\section{Generalized closed characteristics on convex bodies and polytopes}
\label{sec:generalized-closed-char}

We now move from smooth star-shaped hypersurfaces to convex bodies with possibly non-smooth boundary, and in particular to polytopes.
We follow the generalized Reeb dynamics developed by K\"unzle, Artstein-Avidan--Ostrover and Haim--Kislev, specialized to our four-dimensional setting \cite{Kunzle,ArtsteinAvidanOstrover,HaimKislev}.

\subsection{Normal cones and generalized Reeb dynamics}

\begin{definition}[Normal cone of a convex body]\label{def:normal-cone}
Let $K\subset\R^4$ be a convex body and $x\in\partial K$.
The \emph{outer normal cone} of $K$ at~$x$ is
\begin{equation*}
  N_K(x)
  := \{v\in\R^4 \mid \langle y-x, v\rangle\le 0\text{ for all }y\in K\}.
\end{equation*}
Equivalently, $N_K(x)$ is the cone generated by all outer unit normals of supporting hyperplanes of $K$ at~$x$.
\end{definition}

\begin{remark}[Normal cone on a polytope]\label{rem:normal-cone-polytope}
If $K$ is a polytope and $x$ lies in the relative interior of a face $F$, then $N_K(x)$ depends only on $F$.
We denote it by $N^+_F K$ and call it the \emph{positive normal cone} of~$F$.
If $F$ is a facet with outer unit normal~$n_F$, then $N^+_F K = \R_{\ge 0}\,n_F$ is a ray.
If $F$ has codimension~$k$, then $N^+_F K$ is a polyhedral cone of dimension~$k$.
\end{remark}

\begin{definition}[Generalized closed characteristic]\label{def:generalized-closed-char}
Let $K\subset\R^4$ be a convex body.
A \emph{generalized closed characteristic} on $\partial K$ is a loop
\begin{equation*}
  \gamma:\R/T\Z\to\partial K
\end{equation*}
in the Sobolev space $W^{1,2}(\R/T\Z,\R^4)$ such that
\begin{equation}\label{eq:gen-closed-char-inclusion}
  \dot\gamma(t) \in J\,N_K\bigl(\gamma(t)\bigr)
  \quad\text{for almost every }t.
\end{equation}
The \emph{action} of $\gamma$ is still defined by
\begin{equation*}
  \mathcal{A}(\gamma) = \int_{\R/T\Z}\gamma^*\lambda_0,
\end{equation*}
which makes sense because $\gamma$ is absolutely continuous.
\end{definition}

For smooth convex $K$ the normal cone $N_K(x)$ is just the ray spanned by the outer normal vector, so~\eqref{eq:gen-closed-char-inclusion} reduces to the usual Reeb/Hamiltonian equation up to reparametrization.

\begin{definition}[Minimal action closed characteristics]\label{def:minimal-action-generalized}
For a convex body $K$, the \emph{minimal action} of generalized closed characteristics is
\begin{equation*}
  A_{\min}(K)
  := \inf\{\mathcal{A}(\gamma)\mid \gamma\text{ generalized closed characteristic on }\partial K,\,\gamma\text{ non-constant}\}.
\end{equation*}
\end{definition}

By work of K\"unzle, Artstein-Avidan--Ostrover and Haim--Kislev one has the following fundamental result.

\begin{theorem}[EHZ capacity via generalized closed characteristics]\label{thm:ehz-via-gen-char}
Let $K\subset\R^4$ be a convex body with $0\in\mathrm{int}(K)$.
Then
\begin{equation*}
  c_{\mathrm{EHZ}}(K) = A_{\min}(K).
\end{equation*}
Moreover, the infimum is achieved by at least one generalized closed characteristic.
\end{theorem}

We will rely on this theorem as a black box.
In particular, we regard generalized closed characteristics as the dynamical objects whose action realizes $c_{\mathrm{EHZ}}(K)$ for polytopes.

\subsection{Structure of action-minimizing orbits on polytopes in arbitrary dimension}

The next result is specific to polytopes and is due to Haim--Kislev \cite{HaimKislev}.
It gives a strong normal form for action-minimizing generalized closed characteristics.
We first set up notation.

Let $K\subset\R^{2n}$ be a convex polytope with facets $F_1,\dots,F_{F_K}$ and outward unit normals $n_1,\dots,n_{F_K}$.
Let $h_i = h_K(n_i)$ denote the \emph{oriented height} of the facet $F_i$.
As in \cite{HaimKislev}, define
\begin{equation}\label{eq:def-pi}
  p_i := \frac{2}{h_i}\,J n_i \in \R^{2n}.
\end{equation}

\begin{theorem}[Polyhedral normal form for minimizers]\label{thm:hk-polyhedral-normal-form}
Let $K\subset\R^{2n}$ be a convex polytope with $0\in\mathrm{int}(K)$.
Then there exists a generalized closed characteristic $\gamma:[0,1]\to\partial K$ with $\mathcal{A}(\gamma)=c_{\mathrm{EHZ}}(K)$ such that:
\begin{enumerate}
  \item $\gamma$ is piecewise affine with finitely many break points:
        there exists a partition
        \begin{equation*}
          0 = \tau_0 < \tau_1 < \dots < \tau_m = 1
        \end{equation*}
        such that on each open interval $(\tau_{j-1},\tau_j)$ the derivative $\dot\gamma$ is constant.
  \item For each $j$ there exists a facet index $i(j)\in\{1,\dots,F_K\}$ and a constant $c_j>0$ such that
        \begin{equation*}
          \dot\gamma(t) = c_j\,p_{i(j)}
          \quad\text{for all }t\in(\tau_{j-1},\tau_j).
        \end{equation*}
        In particular, $\gamma$ moves with constant velocity parallel to $J n_{i(j)}$ whenever it traverses (the interior of) the facet $F_{i(j)}$.
  \item For each facet index $i$, the set
        \begin{equation*}
          \{t\in[0,1] \mid \dot\gamma(t) \text{ is parallel to } J n_i\}
        \end{equation*}
        is connected.
        Equivalently, $\gamma$ visits the interior of the facet $F_i$ in at most one connected time interval.
\end{enumerate}
\end{theorem}

\begin{remark}\label{rem:hk-non-uniqueness}
The minimizer in Theorem~\ref{thm:hk-polyhedral-normal-form} need not be unique, and not every action-minimizing generalized closed characteristic satisfies these properties.
However, there always exists \emph{some} minimizer with the stated structure.
In dimension four we will only use the existence of such well-structured minimizers and not attempt to classify all minimizers.
\end{remark}

It is convenient to give a name to generalized closed characteristics with the properties above.

\begin{definition}[Polyhedral closed characteristics]\label{def:polyhedral-closed-char}
Let $K\subset\R^{2n}$ be a convex polytope.
A generalized closed characteristic $\gamma$ on~$\partial K$ is called a \emph{polyhedral closed characteristic} if it satisfies items~(1)--(3) of Theorem~\ref{thm:hk-polyhedral-normal-form}.
\end{definition}

Every convex polytope admits at least one polyhedral closed characteristic whose action equals $c_{\mathrm{EHZ}}(K)$.
In dimension four, we will later relate these to combinatorial Reeb orbits in the sense of \cite{ChaidezHutchings}.

\section{Combinatorial Reeb dynamics on non-Lagrangian admissible polytopes}
\label{sec:comb-reeb-nonlag}

We now specialize to non-Lagrangian admissible polytopes in~$\R^4$ and recall the combinatorial Reeb dynamics on their boundary, following \cite{ChaidezHutchings}.

\subsection{Tangent, normal and Reeb cones}

Let $K\subset\R^4$ be an admissible polytope.
For a point $x\in\partial K$, the \emph{tangent cone} $T_x^+ K$ is the closure of the set of vectors $v$ such that $x+\varepsilon v\in K$ for some $\varepsilon>0$.
The \emph{positive normal cone} $N_x^+ K$ is as in Definition~\ref{def:normal-cone}.
The \emph{Reeb cone} at~$x$ is
\begin{equation*}
  R_x^+ K := T_x^+ K \cap J\,N_x^+ K.
\end{equation*}

If $x$ lies in the relative interior of a face $F$, then $T_x^+ K$, $N_x^+ K$ and $R_x^+ K$ are independent of~$x$ and we write $T_F^+ K$, $N_F^+ K$ and $R_F^+ K$.
For a facet $E$ with outer unit normal $n_E$, one has
\begin{equation*}
  N_E^+ K = \R_{\ge 0}\,n_E,\qquad
  R_E^+ K = \R_{\ge 0}\,J n_E.
\end{equation*}

The following key fact is specific to non-Lagrangian admissible polytopes.

\begin{proposition}[One-dimensional Reeb cones for non-Lagrangian faces]\label{prop:reeb-cone-1d}
Let $K\subset\R^4$ be a non-Lagrangian admissible polytope and let $F$ be any face of~$K$.
Then the Reeb cone $R_F^+ K$ is one-dimensional.
\end{proposition}

\begin{proof}[Proof idea]
This is \cite[Prop.~1.5]{ChaidezHutchings} in different terminology.
The key points are:
\begin{itemize}
  \item For any face $F$, the cones satisfy $N_F^+ K = (T_F^+ K)^\circ$ and $T_F^+ K = (N_F^+ K)^\circ$, where $C^\circ$ denotes the polar cone.
  \item In dimension four, if a $2$-face $F$ is not Lagrangian, then the intersection $T_F^+ K \cap J N_F^+ K$ is one-dimensional.
  \item The one-dimensionality then propagates to all other faces using convexity and the fact that $K$ is star-shaped.
\end{itemize}
\end{proof}

\subsection{Good and bad edges, and types of combinatorial orbits}

Assume from now on that $K\subset\R^4$ is a non-Lagrangian admissible polytope.

\begin{definition}[Good and bad $1$-faces]\label{def:good-bad-edges}
Let $L$ be a $1$-face (edge) of~$K$.
\begin{itemize}
  \item $L$ is \emph{good} if the Reeb cone $R_L^+ K$ is not tangent to $L$, i.e.\ the unique generator of $R_L^+K$ points from $L$ into one of the adjacent facets.
  \item $L$ is \emph{bad} if $R_L^+ K$ is tangent to $L$.
\end{itemize}
\end{definition}

Bad edges cannot be avoided in general, but they play a limited role in the capacities we care about, due to index and rotation-number considerations; see \cite{ChaidezHutchings} for details.

We now describe the combinatorial Reeb flow on~$\partial K$.

\begin{definition}[Combinatorial Reeb orbit, non-Lagrangian case]\label{def:comb-reeb-orbit}
Let $K\subset\R^4$ be a non-Lagrangian admissible polytope.
A \emph{combinatorial Reeb orbit} is a finite cyclic list
\begin{equation*}
  \gamma = (\Gamma_1,\dots,\Gamma_k)
\end{equation*}
of oriented line segments in~$\partial K$ such that for each $i$:
\begin{enumerate}
  \item The final endpoint of $\Gamma_i$ equals the initial endpoint of $\Gamma_{i+1}$ (indices taken modulo~$k$).
  \item There exists a face $F_i$ of~$K$ with $\mathrm{int}(\Gamma_i)\subset F_i$, both endpoints of $\Gamma_i$ lying in the boundary of~$F_i$, and $\Gamma_i$ pointing in the direction of the Reeb cone $R_{F_i}^+ K$.
\end{enumerate}
Two lists are considered equivalent if they differ by a cyclic permutation.
\end{definition}

\begin{definition}[Combinatorial symplectic action]\label{def:comb-action}
The \emph{combinatorial symplectic action} of a combinatorial Reeb orbit $\gamma=(\Gamma_1,\dots,\Gamma_k)$ is
\begin{equation*}
  \mathcal{A}_{\mathrm{comb}}(\gamma)
  := \sum_{i=1}^k \int_{\Gamma_i}\lambda_0.
\end{equation*}
\end{definition}

Combinatorial Reeb orbits are classified into three types depending on how they meet the $1$-skeleton of~$K$.

\begin{definition}[Types of combinatorial Reeb orbits]\label{def:comb-orbit-types}
Let $K$ be a non-Lagrangian admissible polytope and $\gamma$ a combinatorial Reeb orbit.
\begin{itemize}
  \item $\gamma$ is of \emph{Type~1} if it does not intersect the $1$-skeleton of $K$.
  \item $\gamma$ is of \emph{Type~2} if it intersects the $1$-skeleton in finitely many points, all of which are endpoints of the segments $\Gamma_i$.
  \item $\gamma$ is of \emph{Type~3} if it contains (the interior of) a bad $1$-face.
\end{itemize}
\end{definition}

Type~1 orbits are the most important for our purposes; they are the combinatorial analogues of embedded Reeb orbits staying in the interiors of facets.
Type~2 orbits are expected to be non-generic, and Type~3 orbits correspond to wrapping around bad edges; see \cite{ChaidezHutchings} for quantitative results bounding their contribution to capacities.

\subsection{Symplectic flow graph and combinatorial dynamics}

For algorithmic purposes it is convenient to package the combinatorial Reeb dynamics into a symplectic flow graph, following \cite{ChaidezHutchings}.

\begin{definition}[Symplectic flow graph of a non-Lagrangian admissible polytope]\label{def:flow-graph}
Let $K\subset\R^4$ be a non-Lagrangian admissible polytope.
The \emph{symplectic flow graph} $G(K)$ is defined as follows.
\begin{itemize}
  \item The vertices of $G(K)$ are the $2$-faces $F$ of~$K$.
        To each vertex we associate the open linear domain $A_F := F^\circ$ (relative interior of~$F$) and the symplectic form $\omega_F := \omega_0|_{T F}$ on $T F$.
  \item There is a directed edge $e$ from a $2$-face $F_1$ to a $2$-face $F_2$ if there exists a facet $E$ adjacent to both $F_1$ and $F_2$ and a trajectory of the constant Reeb vector field on~$E$ from a point in $F_1$ to a point in~$F_2$.
        For such an edge $e$ we define:
        \begin{itemize}
          \item the domain $D_e\subset F_1$ consisting of points whose Reeb trajectory on~$E$ hits~$F_2$,
          \item the affine map $\phi_e:D_e\to F_2$ sending a point to the endpoint of its Reeb segment,
          \item the action function $f_e:D_e\to\R$ equal to the time-of-flight (equivalently, the $\lambda_0$-action) of that Reeb segment.
        \end{itemize}
\end{itemize}
A \emph{periodic orbit} of $G(K)$ is a cycle in this directed graph together with a fixed point of the corresponding composed affine map.
\end{definition}

\begin{proposition}[Type~1 orbits vs.\ flow graph]\label{prop:flow-graph-type1}
Let $K\subset\R^4$ be a non-Lagrangian admissible polytope.
There is a canonical bijection between:
\begin{itemize}
  \item periodic orbits of the flow graph $G(K)$, and
  \item combinatorial Reeb orbits on~$K$ of Type~1.
\end{itemize}
Under this bijection, the action of a periodic orbit (the sum of the edge action functions) equals the combinatorial symplectic action of the corresponding combinatorial Reeb orbit.
\end{proposition}

\begin{proof}[Proof sketch]
Given a periodic orbit in $G(K)$, glue the corresponding Reeb segments in the facets to obtain a closed piecewise linear loop in~$\partial K$ that stays away from the $1$-skeleton.
Conversely, a Type~1 combinatorial Reeb orbit determines an orbit in the flow graph by recording which $2$-face each segment starts in.
The equality of actions follows from additivity of the integral of~$\lambda_0$.
See \cite[Prop.~2.14]{ChaidezHutchings} for details.
\end{proof}

\section{Rotation numbers and indices in the combinatorial setting}
\label{sec:comb-rotation-cz}

To relate combinatorial Reeb orbits to genuine Reeb orbits on smoothings of~$K$, we need a notion of rotation number and Conley--Zehnder index for Type~1 combinatorial orbits.
We briefly recall the quaternionic trivialization from \cite{ChaidezHutchings}.

\subsection{Quaternionic trivialization and transition matrices}

Fix matrices $i,j,k\in\mathrm{SO}(4)$ representing the unit quaternions, with $i$ equal to the standard complex structure~$J$.
Thus left multiplication by $i,j,k$ preserves both $\langle\cdot,\cdot\rangle$ and~$\omega_0$.

Let $K\subset\R^4$ be a non-Lagrangian admissible polytope and let $F$ be a $2$-face.
There is a unique facet $E$ adjacent to $F$ such that the Reeb cone $R_F^+ K$ points from $F$ into~$E$.
Let $n_E$ be the outer unit normal of~$E$.

\begin{definition}[Quaternionic trivialization on a $2$-face]\label{def:quat-triv-face}
For a $2$-face $F$ as above, define a linear isomorphism
\begin{equation*}
  \tau_F : T F \longrightarrow \R^2
\end{equation*}
by
\begin{equation}\label{eq:def-tau-F}
  \tau_F(V) := \bigl(\langle V, j n_E\rangle,\; \langle V, k n_E\rangle\bigr).
\end{equation}
\end{definition}

One checks that $\tau_F$ is a symplectic isomorphism $(T F,\omega_0|_{T F})\to(\R^2,\omega_\mathrm{std})$.

Now let $e$ be an edge of the flow graph $G(K)$ from $F_1$ to $F_2$, with associated affine map $\phi_e:D_e\to F_2$.
The differential $T\phi_e:T F_1\to T F_2$ is a symplectic linear map.

\begin{definition}[Transition matrix]\label{def:transition-matrix}
The \emph{transition matrix} of an edge $e:F_1\to F_2$ with respect to the quaternionic trivialization is
\begin{equation*}
  A_e := \tau_{F_2}\circ T\phi_e \circ \tau_{F_1}^{-1} \in \mathrm{Sp}(2).
\end{equation*}
Among the lifts of $A_e$ to the universal cover $\widetilde{\mathrm{Sp}}(2)$ there is a unique one whose mod~$\Z$ rotation number lies in the interval $(0,\tfrac12)$; we denote this distinguished lift by~$\widetilde{A}_e$.
\end{definition}

\subsection{Combinatorial rotation number and Conley--Zehnder index}

Let $\gamma$ be a Type~1 combinatorial Reeb orbit.
Via the flow-graph description it corresponds to:
\begin{itemize}
  \item a cycle $p=e_1\cdots e_k$ in the directed graph $G(K)$, and
  \item a fixed point $x$ of the composed affine map $\phi_p := \phi_{e_k}\circ\cdots\circ\phi_{e_1}$.
\end{itemize}

\begin{definition}[Combinatorial rotation number and index]\label{def:comb-rotation-cz}
Let $\gamma$ be a Type~1 combinatorial Reeb orbit and let $p=e_1\cdots e_k$ be the corresponding cycle in $G(K)$.
\begin{enumerate}
  \item The \emph{combinatorial rotation element} of $\gamma$ is
        \begin{equation*}
          \widetilde{A}_p := \widetilde{A}_{e_k}\circ\cdots\circ\widetilde{A}_{e_1}
          \in\widetilde{\mathrm{Sp}}(2).
        \end{equation*}
  \item The \emph{combinatorial rotation number} of $\gamma$ is
        \begin{equation*}
          \rho_{\mathrm{comb}}(\gamma) := \rho(\widetilde{A}_p) \in \R,
        \end{equation*}
        where $\rho$ is the rotation number on $\widetilde{\mathrm{Sp}}(2)$ as in Definition~\ref{def:rot-cz}.
  \item We say that $\gamma$ is \emph{combinatorially non-degenerate} if $A_p$ has no eigenvalue equal to~$1$.
        In this case the \emph{combinatorial Conley--Zehnder index} of~$\gamma$ is
        \begin{equation*}
          \mathrm{CZ}_{\mathrm{comb}}(\gamma)
          := \lfloor \rho_{\mathrm{comb}}(\gamma)\rfloor
           + \lceil \rho_{\mathrm{comb}}(\gamma)\rceil \in\Z.
        \end{equation*}
\end{enumerate}
\end{definition}

\begin{remark}[Rotation number of a product]
Since each transition matrix $A_e$ has mod~$\Z$ rotation number in $(0,\tfrac12)$, one can efficiently compute $\rho_{\mathrm{comb}}(\gamma)$ from the list of $A_{e_i}$ using a monotonicity argument on $\widetilde{\mathrm{Sp}}(2)$; see \cite[Prop.~A.9]{ChaidezHutchings}.
This is important for the algorithmic aspects of this thesis.
\end{remark}

\subsection{Smooth--combinatorial correspondence for non-Lagrangian admissible polytopes}

For a convex polytope $K$, define its \emph{$\varepsilon$-smoothing} by
\begin{equation*}
  K_\varepsilon := \{z\in\R^4 \mid \mathrm{dist}(z,K)\le\varepsilon\}.
\end{equation*}
The boundary $\partial K_\varepsilon$ is a $C^1$ smooth convex hypersurface which is $C^\infty$ away from strata corresponding to the boundaries of faces of $K$.
For non-Lagrangian admissible polytopes, \cite{ChaidezHutchings} proves a precise correspondence between Type~1 combinatorial Reeb orbits on $K$ and Reeb orbits on $\partial K_\varepsilon$ for small~$\varepsilon$.

We record a simplified version adapted to our needs.

\begin{theorem}[From combinatorial to smooth]\label{thm:comb-to-smooth}
Let $K\subset\R^4$ be a non-Lagrangian admissible polytope and let $\gamma$ be a combinatorially non-degenerate Type~1 combinatorial Reeb orbit on~$K$.
Then there exists $\varepsilon_0>0$ and, for each $\varepsilon\in(0,\varepsilon_0)$, a distinguished Reeb orbit $\gamma_\varepsilon$ on~$\partial K_\varepsilon$ such that:
\begin{enumerate}
  \item $\gamma_\varepsilon$ converges to $\gamma$ in the $C^0$ topology as $\varepsilon\to 0$.
  \item $\displaystyle\lim_{\varepsilon\to 0} \mathcal{A}(\gamma_\varepsilon) = \mathcal{A}_{\mathrm{comb}}(\gamma)$.
  \item For all sufficiently small $\varepsilon$, the orbit $\gamma_\varepsilon$ is non-degenerate and
        \begin{equation*}
          \rho(\gamma_\varepsilon) = \rho_{\mathrm{comb}}(\gamma),\qquad
          \mathrm{CZ}(\gamma_\varepsilon) = \mathrm{CZ}_{\mathrm{comb}}(\gamma).
        \end{equation*}
\end{enumerate}
\end{theorem}

\begin{theorem}[From smooth to combinatorial]\label{thm:smooth-to-comb}
Let $K\subset\R^4$ be a non-Lagrangian admissible polytope.
Then there exist positive constants $c_F$ for each face $F$ of codimension at least~$1$ with the following property.

Let $\varepsilon_i\to 0^+$ and let $\gamma_i$ be Reeb orbits on $\partial K_{\varepsilon_i}$ with uniformly bounded rotation number $\rho(\gamma_i)\le R$ for some fixed~$R$.
Then, after passing to a subsequence, there is a combinatorial Reeb orbit $\gamma$ on~$K$ such that:
\begin{enumerate}
  \item $\gamma_i$ converges to $\gamma$ in $C^0$ as $i\to\infty$.
  \item $\displaystyle\lim_{i\to\infty}\mathcal{A}(\gamma_i) = \mathcal{A}_{\mathrm{comb}}(\gamma)$.
  \item The orbit $\gamma$ is of Type~1 or Type~2.
  \item If $\gamma$ is of Type~1 and combinatorially non-degenerate, then for all large~$i$ the orbits $\gamma_i$ are non-degenerate and
        \begin{equation*}
          \rho(\gamma_i) = \rho_{\mathrm{comb}}(\gamma),\qquad
          \mathrm{CZ}(\gamma_i) = \mathrm{CZ}_{\mathrm{comb}}(\gamma).
        \end{equation*}
  \item If $\gamma=(\Gamma_1,\dots,\Gamma_k)$ and $F_1,\dots,F_k$ are the faces containing the segments, then
        \begin{equation*}
          \sum_{j=1}^k c_{F_j} \le R.
        \end{equation*}
\end{enumerate}
\end{theorem}

\begin{corollary}[Computing $c_{\mathrm{EHZ}}$ from combinatorial orbits]\label{cor:ehz-from-comb}
Let $K\subset\R^4$ be a non-Lagrangian admissible polytope.
Then
\begin{equation*}
  c_{\mathrm{EHZ}}(K)
  = \min\left\{
            \mathcal{A}_{\mathrm{comb}}(\gamma)
          \;\middle|\;
            \gamma\text{ combinatorial orbit with }\sum_F c_F\le 2,\;
            \gamma\text{ is Type~1 with }\rho_{\mathrm{comb}}(\gamma)\le 2\text{ or Type~2}
          \right\},
\end{equation*}
where the sum is taken over the faces $F$ traversed by~$\gamma$.
\end{corollary}

\begin{proof}[Idea]
Apply Theorem~\ref{thm:smooth-to-comb} to a sequence of Reeb orbits realizing $c_{\mathrm{EHZ}}(K_\varepsilon)$ on smoothings and use the index bounds from Proposition~\ref{prop:convex-index-bounds}, together with continuity of $c_{\mathrm{EHZ}}$ under $C^0$ convergence of convex bodies.
For the reverse inequality note that every combinatorial Reeb orbit is (the limit of) a generalized closed characteristic.
\end{proof}

\section{Extending combinatorial dynamics to admissible polytopes with Lagrangian faces}
\label{sec:comb-lagrangian}

The correspondences above rely crucially on the assumption that no $2$-face of $K$ is Lagrangian.
In the rest of the thesis we will need to handle admissible polytopes that \emph{do} have Lagrangian $2$-faces.
We briefly explain how our terminology extends to this situation.

Let $K\subset\R^4$ be an arbitrary admissible polytope (possibly with Lagrangian $2$-faces).
For a Lagrangian $2$-face $F$ one checks that $R_F^+ K$ is a $2$-dimensional cone tangent to~$F$, and its boundary consists of two one-dimensional cones corresponding to the adjacent facets.
Thus the combinatorial Reeb cone is no longer uniquely directed into a single facet, and as a result the flow on~$F$ is not well-posed in the sense of Definition~\ref{def:comb-reeb-orbit}.

For our purposes we adopt the following compromise.

\begin{definition}[Generalized combinatorial closed characteristics]\label{def:gen-comb-closed-char}
Let $K\subset\R^4$ be an admissible polytope (possibly with Lagrangian $2$-faces).
A \emph{generalized combinatorial closed characteristic} on~$K$ is a loop
\begin{equation*}
  \gamma:\R/T\Z\to\partial K
\end{equation*}
in $W^{1,2}(\R/T\Z,\R^4)$ such that for almost every~$t$ there exists a face $F$ with $\gamma(t)\in F$ and
\begin{equation*}
  \dot\gamma(t) \in R_F^+ K.
\end{equation*}
\end{definition}

This definition reduces to the notion of a generalized Reeb orbit in the sense of Definition~\ref{def:generalized-closed-char} because $R_F^+ K \subset J N_F^+ K$ by construction, but it encodes more combinatorial information about which faces are used.

\begin{remark}
For non-Lagrangian admissible polytopes, generalized combinatorial closed characteristics are precisely the generalized Reeb orbits of \cite{ChaidezHutchings} whose velocity vectors always point in the unique Reeb direction inside the Reeb cone of each face.
For polytopes with Lagrangian $2$-faces we allow velocity vectors anywhere inside the $2$-dimensional Reeb cone of a Lagrangian face.
\end{remark}

We will primarily be interested in generalized combinatorial closed characteristics that arise as limits of Reeb orbits on smoothings $K_\varepsilon$ with uniformly bounded rotation number.
The generalization of Theorems~\ref{thm:comb-to-smooth} and~\ref{thm:smooth-to-comb} to that setting requires additional work and will be one of the contributions of this thesis.
For the introductory chapter it suffices to have a precise definition of the objects we aim to approximate.

\section{Haim--Kislev's combinatorial formula for \texorpdfstring{$c_{\mathrm{EHZ}}$}{cEHZ} on polytopes}
\label{sec:hk-formula}

We finally recall the explicit combinatorial formula for $c_{\mathrm{EHZ}}(K)$ for convex polytopes due to Haim--Kislev \cite{HaimKislev}, specialized to dimension~$4$.

Let $K\subset\R^{2n}$ be a convex polytope with $0\in\mathrm{int}(K)$, facets $F_1,\dots,F_{F_K}$, outer unit normals $n_1,\dots,n_{F_K}$ and oriented heights $h_i=h_K(n_i)$.
Let $S_{F_K}$ be the symmetric group on $\{1,\dots,F_K\}$ and define the polytope
\begin{equation*}
  \mathcal{M}(K) := \left\{ (\beta_i)_{i=1}^{F_K}\in[0,\infty)^{F_K}
    \,\middle|\,
    \sum_{i=1}^{F_K} \beta_i h_i = 1,\;
    \sum_{i=1}^{F_K} \beta_i n_i = 0
  \right\}.
\end{equation*}

\begin{theorem}[Haim--Kislev formula]\label{thm:hk-main}
Let $K\subset\R^{2n}$ be a convex polytope with $0\in\mathrm{int}(K)$.
Then
\begin{equation}\label{eq:hk-formula}
  c_{\mathrm{EHZ}}(K) =
  \frac12 \left[
    \max_{\sigma\in S_{F_K}} \max_{(\beta_i)\in\mathcal{M}(K)}
    \sum_{1\le j<i\le F_K} \beta_{\sigma(i)}\beta_{\sigma(j)}\,\omega_0(n_{\sigma(i)},n_{\sigma(j)})
  \right]^{-1}.
\end{equation}
\end{theorem}

\begin{remark}[Alternative formulation]\label{rem:hk-alt}
An equivalent and often more convenient form is
\begin{equation}\label{eq:hk-alt}
  c_{\mathrm{EHZ}}(K)
  = \frac12 \min_{(\beta_i,n_i)\in\mathcal{M}_2(K)} \left(\sum_{i=1}^{F_K}\beta_i h_K(n_i)\right)^2,
\end{equation}
where the minimum is taken over finite sequences $(\beta_i,n_i)$ of non-negative coefficients and outer normals to $K$ satisfying
\begin{equation*}
  \sum_i \beta_i n_i = 0,\qquad
  \sum_{1\le j<i} \beta_i \beta_j \,\omega_0(n_i,n_j) = 1.
\end{equation*}
In both formulations the key point is that one optimizes over a \emph{finite-dimensional} polytope of coefficients and a \emph{finite} permutation group, rather than over an infinite-dimensional function space.
\end{remark}

\begin{remark}[Centrally symmetric case]\label{rem:hk-central}
If $K$ is centrally symmetric, i.e.\ $K=-K$, the formula simplifies.
Label the facet normals as $\{\pm n_1,\dots,\pm n_{F_K'}\}$, so $F_K=2F_K'$.
Then
\begin{equation*}
  c_{\mathrm{EHZ}}(K)
  = \frac14 \left[
    \max_{\sigma\in S_{F_K'}}
    \max_{(\beta_i)\in\mathcal{M}_0(K)}
    \sum_{1\le j<i\le F_K'} \beta_{\sigma(i)}\beta_{\sigma(j)}\,\omega_0(n_{\sigma(i)},n_{\sigma(j)})
  \right]^{-1},
\end{equation*}
where
\begin{equation*}
  \mathcal{M}_0(K)
  := \left\{ (\beta_i)_{i=1}^{F_K'} \,\middle|\,
            \sum_{i=1}^{F_K'} |\beta_i| h_i = \tfrac12
     \right\}.
\end{equation*}
\end{remark}

\begin{remark}[Relation to polyhedral closed characteristics]\label{rem:hk-vs-polyhedral}
The proof of Theorem~\ref{thm:hk-main} proceeds via Clarke's dual action principle and the polyhedral normal form Theorem~\ref{thm:hk-polyhedral-normal-form}.
Roughly, each choice of permutation $\sigma$ and coefficients $(\beta_i)$ encodes a polyhedral closed characteristic with velocities proportional to $p_i$ from~\eqref{eq:def-pi}, and the constraint $\sum\beta_i n_i=0$ encodes the closing condition of the loop.
The quadratic form in~\eqref{eq:hk-formula} is essentially the action of this orbit.
\end{remark}

In later chapters we will use Theorem~\ref{thm:hk-main} in two complementary ways:
\begin{itemize}
  \item as an independent way to compute $c_{\mathrm{EHZ}}(K)$ and compare with the combinatorial Reeb approach of Section~\ref{sec:comb-rotation-cz};
  \item as a source of structural information about action-minimizing orbits (e.g.\ bounds on how many facets they can traverse, behavior near Lagrangian faces, and numerical stability properties).
\end{itemize}

\section{Examples}
\label{sec:examples}

We close this chapter with a few examples illustrating the definitions and highlighting some edge cases.

\subsection{The Euclidean ball}

Let $K=B(r)\subset\R^4$ be the radius-$r$ ball.
Then $\partial K$ is strictly convex and the Reeb flow of $\lambda_0$ is the Hopf flow.
Every Reeb orbit is periodic and all have the same period
\begin{equation*}
  T = r.
\end{equation*}
Thus $\mathcal{A}(\gamma)=r$ for all closed characteristics and
\begin{equation*}
  c_{\mathrm{EHZ}}(B(r)) = r,\qquad
  \mathrm{sys}(B(r)) = 1.
\end{equation*}

There are no Lagrangian $2$-faces; indeed, $\partial B(r)$ has no flat faces at all.
The tangent and normal cones at each point are half-spaces and rays, respectively, and the Reeb cone is a ray generated by the Hopf vector field.

\subsection{A polytope with Lagrangian $2$-faces: the standard simplex}

Consider the standard $4$-simplex
\begin{equation*}
  \Delta^4 := \left\{ (x_1,x_2,y_1,y_2)\in\R^4 \,\middle|\,
                         x_1,x_2,y_1,y_2\ge 0,\;
                         x_1+x_2+y_1+y_2 \le 1
                 \right\}.
\end{equation*}
After a small translation we may assume $0\in\mathrm{int}(\Delta^4)$, so $\Delta^4$ is admissible.
Many of its $2$-faces are Lagrangian: for instance the face $\{x_1=x_2=0\}\cap\Delta^4$ is a triangle in the $(y_1,y_2)$-plane on which $\omega_0$ vanishes identically.
Thus $\Delta^4$ is \emph{not} a non-Lagrangian admissible polytope.
The Reeb cone along such a Lagrangian $2$-face is two-dimensional and tangent to the face, so the combinatorial Reeb flow is not uniquely determined there.
This is a prototypical example of the situations we will need to handle in later chapters.

\subsection{A non-Lagrangian admissible polytope with systolic ratio $1$}

Examples of non-Lagrangian admissible polytopes $K\subset\R^4$ with $\mathrm{sys}(K)=1$ were found in \cite{ChaidezHutchings}.
One particularly symmetric example is a rotated $24$-cell: a regular $4$-dimensional polytope all of whose facets are octahedra.
In that case the flow graph $G(K)$ decomposes into disjoint cycles of length~$12$, and every point in the interior of a $2$-face lies on a Type~1 combinatorial Reeb orbit of the same action.
The polytope is combinatorially Zoll in the sense that an open dense subset of $\partial K$ is foliated by action-minimizing closed characteristics.

This example shows that the equality case $\mathrm{sys}(K)=1$ is not restricted to balls even within the class of polytopes; symmetry and combinatorial Zollness play an important role.

\subsection{Products and Mahler-type examples}

Given convex bodies $A\subset\R^2$ and $B\subset\R^2$, one can form the Lagrangian product $A\times B\subset\R^4$.
Such products are natural candidates for extremal behavior in Viterbo-type inequalities and in Mahler's conjecture.
For instance, the product of a triangle and a square (in suitable coordinates) gives a polytope whose systolic ratio can be computed explicitly using either Haim--Kislev's formula or combinatorial Reeb dynamics.
Some of these products have Lagrangian $2$-faces, others do not; they provide a convenient testbed for the methods developed in this thesis.

We will return to explicit numerical examples in later chapters, once the algorithmic framework is in place.

\section{Summary of notation and conventions}
\label{sec:notation-summary}

For quick reference we list the most important objects introduced in this chapter.

\begin{itemize}
  \item Ambient space: $(\R^4,\omega_0,\lambda_0,J,\langle\cdot,\cdot\rangle)$ with standard structures.
  \item Convex bodies and polytopes:
        \begin{itemize}
          \item $K\subset\R^4$: convex body with $0\in\mathrm{int}(K)$.
          \item $K$ \emph{admissible}: convex polytope with $0\in\mathrm{int}(K)$.
          \item $K$ \emph{non-Lagrangian admissible}: admissible and all $2$-faces are symplectic.
          \item Faces $F$ are always non-empty; $\dim F=3$ facets, $\dim F=2$ $2$-faces, etc.
        \end{itemize}
  \item Convex-analytic functions:
        \begin{itemize}
          \item $g_K$ gauge (Minkowski functional), $h_K$ support function, $K^\circ$ polar body.
          \item $F=\frac14 g_K^2$, $G=h_K^2$ are Fenchel conjugates, $F^*=G$, $G^*=F$.
        \end{itemize}
  \item Normal and Reeb cones:
        \begin{itemize}
          \item $N_K(x)$ outer normal cone at $x\in\partial K$.
          \item $T_x^+K$ tangent cone, $R_x^+K = T_x^+K\cap J N_K(x)$.
          \item For a face $F$, these become $N_F^+K$, $T_F^+K$, $R_F^+K$.
        \end{itemize}
  \item Periodic dynamics:
        \begin{itemize}
          \item Closed characteristics, Reeb orbits and Hamiltonian orbits are identified as in Lemma~\ref{lem:equivalence-periodic-objects}.
          \item Action of a loop: $\mathcal{A}(\gamma)=\int\gamma^*\lambda_0$; for Reeb orbits this equals the period.
          \item EHZ capacity: $c_{\mathrm{EHZ}}(K) = A_{\min}(K)$ for convex bodies.
        \end{itemize}
  \item Generalized and polyhedral closed characteristics:
        \begin{itemize}
          \item Generalized closed characteristics satisfy $\dot\gamma(t)\in J N_K(\gamma(t))$ a.e.
          \item Polyhedral closed characteristics are piecewise affine minimizers with velocities proportional to $J n_i$ on facets.
        \end{itemize}
  \item Combinatorial dynamics on non-Lagrangian admissible polytopes:
        \begin{itemize}
          \item Symplectic flow graph $G(K)$ with vertices $2$-faces and edges given by Reeb flow in facets.
          \item Combinatorial Reeb orbit: cyclic list of Reeb segments along face-wise Reeb cones.
          \item Types~1--3 depending on interaction with the $1$-skeleton.
          \item Quaternionic trivialization $\tau_F$ on each $2$-face, transition matrices $A_e$ on edges.
          \item Combinatorial rotation number $\rho_{\mathrm{comb}}$ and index $\mathrm{CZ}_{\mathrm{comb}}$ defined via the lift $\widetilde{A}_p$.
        \end{itemize}
  \item Haim--Kislev data:
        \begin{itemize}
          \item Facet normals $n_i$, oriented heights $h_i=h_K(n_i)$.
          \item Coefficient polytope $\mathcal{M}(K)$ and formula~\eqref{eq:hk-formula} for $c_{\mathrm{EHZ}}(K)$.
          \item For centrally symmetric $K$, simplification as in Remark~\ref{rem:hk-central}.
        \end{itemize}
\end{itemize}

The rest of the thesis will build on this background, extending the combinatorial--smooth correspondence to admissible polytopes with Lagrangian faces and developing efficient, numerically stable algorithms for computing $c_{\mathrm{EHZ}}(K)$ and related dynamical quantities.

