% !TeX root = ../../main.tex
\section{HK/CH combinatorial capacity formula (4D)}
Let \((n_i,h_i)_{i=1}^F\) be the outward unit normals and heights of an admissible polytope \(K\subset\R^4\). Define coefficients \(\beta_i\ge0\) with
\[
  \sum_i \beta_i h_i = 1,\qquad \sum_i \beta_i n_i = 0.
\]
For a permutation \(\sigma\in S_F\), set
\[
  Q(\sigma,\beta)=\sum_{j<i} \beta_{\sigma(i)}\,\beta_{\sigma(j)}\,\omega(n_{\sigma(i)},n_{\sigma(j)}).
\]

\begin{theorem}[HK combinatorial formula, 4D]\label{thm-hk-combinatorial}
Haim--Kislev's formula gives
\[
  \cEHZ(K) = \frac{1}{2}\Big[\max_{\sigma,\beta} Q(\sigma,\beta)\Big]^{-1}.
\]
In the centrally symmetric case the factor becomes \(\tfrac14\) with paired normals. The maximizer encodes a simple action-minimizing orbit: velocities appear in the order \(\sigma\) with normalized time weights \(\beta_{\sigma(i)}\) as in the definition of simple orbits. Chaidez--Hutchings show that for non-Lagrangian polytopes any minimizer has combinatorial rotation number \(\rho\le2\) (measured in turns, i.e. \(1 = 2\pi\)), giving a finite search set for algorithms \cite{CH2021}.
\end{theorem}
