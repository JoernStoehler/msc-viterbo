% !TeX root = ../../main.tex
\section{Viterbo conjecture (falsified)}

\begin{conjecture}[Viterbo]\label{conj-viterbo}
For any bounded convex body \(K \subset \R^4\), the systolic ratio satisfies \(\sys(K) \le 1\).
\end{conjecture}

\begin{example}[Counterexample]\label{ex-counterexample-viterbo}
\cite{HK2024} constructs a polytope \(K \subset \R^4\) with \(\sys(K) > 1\), falsifying \cref{conj-viterbo}.
\end{example}

The existing literature includes further statements:

\begin{theorem}[Simplex case]\label{thm-simplex-viterbo}
For simplices \(\sys(K) \le 3/4 < 1\). Equality holds only for the orthonormal simplex with vertices \(\{0, e_1, e_2, e_3, e_4\}\) and its symplectomorphic images (translations and linear symplectic maps).
%TODO Cite: master thesis of Haim-Kislev.
\end{theorem}

\begin{theorem}[Mahler conjecture in 2D]\label{thm-mahler-2d-viterbo}
For centrally symmetric convex polygons \(P = -P \subset \R^2\) we have \(\mathrm{area}(P)\,\mathrm{area}(P^{\polar}) \ge 8\), with equality for parallelograms. Then \(K = P \times P^{\polar} \subset \R^4\) satisfies \(\sys(K)\le1\), with equality preserved.
%TODO: Cite
\end{theorem}

Trivial families of counterexamples arise from scaling, symplectomorphisms, and small perturbations of known counterexamples, since \(\sys\) is continuous in the Hausdorff metric. We seek nontrivial counterexamples and a systematic computational search.
