% !TeX root = ../../main.tex
\section{Hamiltonian dynamics}
Now fix a polytope \(K\) with irredundant data \((n_i,h_i)_{i=1}^F\).
We assume the reader is familiar with Hamiltonian vector fields, flows, and orbits in the smooth case.

\paragraph{Standard Hamiltonian.} We set \(\Ham = \gauge_K^2\), the square of the gauge. It is 2-homogeneous, convex, and piecewise quadratic for polytopes. Derivatives fail on rays through the 2-faces of \(K\). On the polytope boundary \(\partial K\) we have \(\Ham \equiv 1\), and on the interior of a facet \(F_i\) we have
\[
  \nabla H(x) = \frac{2}{h_i} n_i
\]
so that the subdifferential in general is
\[
  \partial H(x) = \operatorname{conv} \Big\{ \frac{2}{h_i} n_i : x \in F_i \Big\}.
\]
We define facet "velocities"
\[
  p_i := \frac{2}{h_i} \Jmat n_i.
\]

\paragraph{Regular case.} For \(\Ham\in C^1(\R^4)\), the Hamiltonian vector field satisfies \(\omegaform(X_\Ham,\cdot) = - d\Ham(\cdot)\), equivalently \(X_\Ham = 2 \Jmat \nabla \Ham\). The flow \(\dot x = X_\Ham(x)\) preserves level sets of \(\Ham\).
\begin{proof}
  Quick sign check:
  \[
    \omegaform(X_H,v) = \tfrac12 \inner{\Jmat X_H}{v} = -dH(v) = \inner{-\nabla H}{v} \implies X_H = 2 \Jmat \nabla H.
  \]
\end{proof}

\paragraph{Polytope case.} For polytopes \(\Ham\) is not differentiable everywhere; we use the subdifferential \(\partial \Ham(x)\) and the inclusion
\[
  \dot x(t) \in 2 \Jmat \partial \Ham(x(t)) \quad \text{a.e.}
\]
Solutions in \(W^{1,2}(\R,\R^4)\) exist for any initial condition and all time but need not be unique. The boundary \(\partial K\) is invariant.
