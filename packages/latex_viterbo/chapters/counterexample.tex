% !TeX root = ../main.tex
This chapter records worked polytopes used as regression tests and illustrations.
We summarize the inputs (facet normals/heights), the simple minimizing orbit, and the resulting capacity/systolic ratio. Computations follow the sign conventions fixed in \cref{chap:math}: coordinates \((q_1,q_2,p_1,p_2)\) and \(J(q,p)=(-p,q)\).

\section{HK\&O 2024 counterexample}
Haim--Kislev--Ostrover construct a centrally symmetric polytope \(K\subset\R^4\) with \(\sys(K)>1\), disproving Viterbo's conjecture.
We use their data as a primary regression case.
\begin{itemize}
  \item \textbf{Facet data.} Outward unit normals \(n_i\) and heights \(h_i\) live in \path{packages/python_viterbo/data/counterexamples/hk-o-2024/} (JSON). Velocities are \(p_i = \tfrac{2}{h_i}J n_i\).
  \item \textbf{Simple orbit.} The minimizing orbit visits each facet at most once; the order and time weights \(\beta_i\) realize the maximizer in the HK/CH formula (\cref{thm-hk-combinatorial}). We normalize \(\sum_i \beta_i h_i = 1\) and \(\sum_i \beta_i n_i = 0\).
  \item \textbf{Capacity and systolic ratio.} The capacity equals \(\cEHZ(K) = \frac{1}{2\max Q(\sigma,\beta)}\); numerics from our implementation are reported in \cref{tab:hk-counterexample}. % TODO: insert table from JSON once verified
  \item \textbf{Assets.} Plots of the orbit projection and residuals are generated by Python stages and stored under \path{packages/latex_viterbo/assets/counterexamples/}.
\end{itemize}

% TODO: Insert orbit order and weights \((\sigma,\beta)\) once solver output is available.

\begin{table}[ht]
  \centering
  \caption{HK\&O 2024 pentagon product summary (data: \protect\path{packages/python_viterbo/data/counterexamples/hk-o-2024/summary.json}).}
  \label{tab:hk-counterexample}
  \begin{tabular}{lcc}
    \toprule
    Quantity & Value & Closed form \\
    \midrule
    \(\cEHZ(K)\) & $2.3776412907$ & $2\cos(\tfrac{\pi}{10})(1+\cos(\tfrac{\pi}{5}))$ \\
    \(\vol(K)\) & $2.3776412907$ & $\tfrac{5}{2}\sin(\tfrac{2\pi}{5})^2$ \\
    \(\sys(K)\) & $1.0472135955$ & $\tfrac{\sqrt{5}+3}{5}$ \\
    \bottomrule
  \end{tabular}
  \newline
  Numerical values are reproduced from the Python stage output; tolerances to be finalized once the Rust solver is in place.
\end{table}

\begin{figure}[ht]
  \centering
  \includegraphics[width=0.5\textwidth]{assets/counterexamples/hko-2024/polygons.png}
  \caption{HK\&O pentagon factors (q-plane vs. rotated p-plane).}% TODO: Replace with orbit plot once available.
  \label{fig:hko-polygons}
\end{figure}

\section{Non-Lagrangian example (rotation bound)}
To illustrate a typical non-Lagrangian polytope and the rotation-number bound \(\rho\le2\), we include a small worked example.
\begin{itemize}
  \item Facets and heights are provided in\\
    \path{packages/python_viterbo/data/examples/non-lagrangian/}.
  \item The minimizing orbit avoids the \(1\)-skeleton (Type~1) and fits in the search space used by the algorithm.
  \item This serves as a regression for the branch-and-bound/pruning logic.
\end{itemize}

\section{Sanity examples}
Simple bodies (cubes/products) are used to confirm scale/translation invariance and the \(\sys(K)\le1\) benchmark in easy cases. Data live in \path{packages/python_viterbo/data/sanity/}.

\section{What is reported here}
\begin{itemize}
  \item Tables of \(h_i\), \(n_i\), \(p_i\) for each worked polytope.
  \item Orbit order and weights \((\sigma,\beta)\) achieving the maximum in \cref{thm-hk-combinatorial}.
  \item Computed values: \(\cEHZ(K)\), \(\vol(K)\), \(\sys(K)\), with solver tolerances.
  \item Pointers to assets for figures; full reproducibility via Python stages.
\end{itemize}

\section{Open tasks}
\begin{itemize}
  \item Insert the HK\&O facet list and orbit weights once the dataset is finalized.
  \item Add a fully explicit non-Lagrangian example and narrative justification.
  \item Cross-check sign conventions against the implementation output.
\end{itemize}
