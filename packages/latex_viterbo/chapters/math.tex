% !TeX root = ../main.tex

This chapter sets notation and definitions for the rest of the thesis. 
We follow primarily~\cite{CH2021} and secondarily~\cite{HK2017,HK2024} in our choice of notation. 
We assume the reader is already familiar with basic symplectic geometry in a smooth setting (Reeb vector fields, contact forms, symplectic capacities, Hamiltonian dynamics).

\section{The Smooth Setting}
We use standard notation, and view \(\R^4\) as a symplectic vector space with the standard symplectic structure. Let
\begin{description}
  \item[Coordinates] \(z=(x_1,x_2,y_1,y_2)\).
  \item[Inner product] \(\inner{x}{y} = x^T y\).
  \item[Norm] \(\abs{x}=\sqrt{\inner{x}{x}}\).
  \item[Volume form] \(\vol = dx_1 \wedge dy_1 \wedge dx_2 \wedge dy_2\).
  \item[Symplectic form] \(\omega = dx_1 \wedge dy_1 + dx_2 \wedge dy_2\). Note: \(\omega(x,y) = \inner{Jx}{y}\).
  \item[Liouville 1-form] \(\lambda = \tfrac12 \sum_{i=1}^2 (x_i\,dy_i - y_i\,dx_i)\), so \(d\lambda = \omega\). Note: \(\lambda_x(y) = \tfrac12 \inner{Jx}{y}\).
  \item[Almost complex structure] \(J = \begin{pmatrix}0 & -I_2 \\ I_2 & 0\end{pmatrix}\)
  \item[Lagrangian subspace] A 2-dimensional linear subspace \(L\subset T_z \R^4\) is Lagrangian iff \(\omega|_L \equiv 0\).
  \item[Smooth Body] A strictly convex, bounded, closed subset \(K \subset \R^4\) that is star-shaped wrt the origin and has smooth boundary. Star-shaped means the radial vector field is transverse to the boundary.
  \item[Contact form] \(\alpha = \lambda|_{\partial K}\).
  \item[Reeb vector field] The unique vector field \(R \in \Gamma (T \partial K)\) such that \(\iota_R d\alpha = 0\) and \(\alpha(R) = 1\).
  \item[Reeb flow] The flow \(\phi^t_R\) generated by the Reeb vector field \(R\).
  \item[Reeb orbit] A periodic orbit of the Reeb flow, i.e. a map \(\gamma \in C^\infty(\R/T\Z, \partial K)\) such that \(\dot{\gamma}(t) = R(\gamma(t))\) for all \(t\).
  \item[Hamiltonian] A smooth function \(H \in C^\infty(\R^4, \R)\).
  \item[Hamiltonian vector field] The unique vector field \(X_H \in \Gamma(T\R^4)\) such that \(\iota_{X_H} \omega = -dH\).
  \item[Hamiltonian flow] The flow \(\phi^t_H\) generated by the Hamiltonian vector field \(X_H\).
  \item[Hamiltonian orbit] A periodic orbit of the Hamiltonian flow, i.e. a map \(z \in C^\infty(\R/T\Z, \R^4)\) such that \(\dot{z}(t) = X_H(z(t))\) for all \(t\).
  \item[Action of a curve] For a smooth periodic curve \(z \in C^\infty(\R/T\Z, \R^4)\), its action is defined as
    \[
       A(z) = \int_0^T \lambda(\dot{z}(t))\,dt.
    \]
\end{description}

In the smooth setting we can now state the Viterbo conjecture as follows.

\section{Viterbo Conjecture}

\begin{definition}[Ekeland-Hofer-Zehnder capacity]
  The Ekeland-Hofer-Zehnder capacity of a body \(K \subset \R^4\) is defined as
  \[
    c_{EHZ}(K) = \inf \{ A(\gamma) : \gamma \text{ is a Reeb orbit on } \partial K \}.
  \]
  It fulfills the axioms of a normalized symplectic capacity.
\end{definition}
We will later define the notation of a Reeb orbit on non-smooth bodies, so that the definition above continues to make sense in that setting.

\begin{definition}[Systolic ratio]
  For any body \(K \subset \R^4\), we define the systolic ratio as
  \[
    \sys(K) = \frac{c_{EHZ}(K)^2}{4 \vol(K)}.
  \]
\end{definition}

\begin{conjecture}[Viterbo conjecture, falsified]
  For any body \(K \subset \R^4\), the systolic ratio is \(\sys(K) \leq 1\).
\end{conjecture}
\begin{example}[Counterexample to Viterbo Conjecture]
  In \cite{HK2024}, a counterexample is explicitly given for the Viterbo conjecture. The body is a 4d polytope
  \[
    K = P_5 \times_L R(90^\circ) P_5,
  \]
  where \(P_5\) is the regular pentagon with circumradius 1, and \(R(90^\circ)\) is the rotation by 90 degrees CCW. The notation \(\times_L\) denotes the Lagrangian product, i.e. one factor lies in the \((x_1,x_2)\)-plane and the other in the \((y_1,y_2)\)-plane. For this body, it is explicitly computed that \(\sys(K) > 1\).
\end{example}

The foundational idea of this master thesis is to probe where the Viterbo conjecture fails using computational methods.

\section{Polytope Setting}

In order to algorithmically compute the Ekeland-Hofer-Zehnder capacity and volume of a body \(K \subset \R^4\), we want to switch to a simpler combinatorial setting. We use instead of smooth bodies the class of polytopes.
Polytopes can here be interpreted as limits of smooth bodies, as a dense subset in the Hausdorff metric of the space of all bodies, or we can directly do variational calculus on polytopes using nonsmooth analysis.
Regardless of which approach we take, we end up with the same canonical definitions for doing symplectic geometry on polytopes.

We will in the following talk about polytopes in \(\R^4\) only, and always mean convex, bounded, closed polytopes with finitely many faces, that are star-shaped wrt the origin.

\begin{definition}[Polytope]
  A polytope \(K \subset \R^4\) has an irredundant representation with minimal cardinality as the intersection of finitely many positive halfspaces
  \[
    K = \bigcap_{i=1}^n \{ z \in \R^4 : \inner{z}{n_i} \leq h_i \},
  \]
  where \(n_i \in \R^4\) are the outer normal vectors of the facets with \(\abs{n_i}=1\) and \(h_i > 0\) are the support values, also called heights.
  Alterantively, we can represent the polytope as the convex hull of its vertices.
\end{definition}

We now set basic notation for talking about polytopes:
\begin{description}
    \item[Half-Space] The closed positive half-space defined by a normal vector \(n_i\) and height \(h_i>0\) is \(H_i = \{ z \in \R^4 : \inner{z}{n_i} \leq h_i \}\).
    \item[Polytope] A bounded intersection of finitely many positive half-spaces \(K = \bigcap_{i=1}^n H_i\).
    \item[Face] We call a subset of the polytope boundary \(F\) a \(k\)-face if it lies in a \(k\)-dimensional affine subspace and has non-empty relative interior wrt that subspace, and is maximal with these properties. We can always represent a \(k\)-face as the intersection of the polytope with all half-space hyperplanes that contain the face.
    \item[Facet] A 3-face is called a facet. Each facet \(F_i\) corresponds to exactly one half-space \(H_i\) in the irredundant representation of the polytope. It has facet normal \(n_i\) and facet height \(h_i\).
    \item[2-Face] We avoid calling 2-faces ``ridges'' to stay unambiguous. Each 2-face \(F_{ij}\) is the intersection of exactly two facets \(F_i\) and \(F_j\).
    \item[1-Face] We avoid calling 1-faces ``edges'' to stay unambiguous. Each 1-face \(F\) is the intersection of three or more facets. Unlike with 2-faces we cannot upper-bound the number of facets intersecting in a 1-face.
    \item[0-Face] The 0-faces correspond to the vertices of the polytope, they consist of single points. Each 0-face is the intersection of four or more facets.
\end{description}

The boundary \(\partial K\) of a polytope then is the union of all its facets \(F_i\), and the disjoint union of the relative interiors of all its faces.
We can also describe the boundary of a polytope as a stratified manifold, where each stratum corresponds to the relative interior of a face.

Since smoothness, or even differentiability everywhere, is lost when moving to polytopes, we need to adapt our definitions to the lower regularity setting.

\begin{description}
    \item[Curves] We generally consider absolutely continuous periodic curves with square-integrable weak derivative, i.e. \(W^{1,2}(\R/T\Z, \R^4)\). Similar to the smooth setting, we do not normalize the period to 1, since the dynamics are more simply expressed with the natural period \(T\).
    \item[Hamiltonian] We are satisfied with an absolutely continous convex piecewise smooth Hamiltonian \(H: \R^4 \to \R\). We will later pick a single standard Hamiltonian we use anyways that we discuss in more detail.
    \item[Subdifferential] For a convex function \(H: \R^4 \to \R\), the subdifferential at a point \(z \in \R^4\) is defined as
        \[
            \partial H(z) = \{ w \in \R^4 : H(z') \geq H(z) + \inner{w}{z'-z} \text{ for all } z' \in \R^4 \}.
        \]
        It generalizes the gradient to non-differentiable functions.
        If at some point \(z\) the function \(H\) is differentiable, then the subdifferential contains only the gradient, i.e. \(\partial H(z) = \{ \nabla H(z) \}\).
        The subdifferential is always a non-empty convex set.
    \item[Hamiltonian orbit] The flow equation is now replaced by a flow inclusion relation. A curve \(\gamma\) is a (generalized) Hamiltonian orbit iff
        \[
            \dot{\gamma}(t) \in J \partial H(\gamma(t)) \text{ for almost all } t \in \R/T\Z.
        \]
        If on some open subset \(U \subset \R^4\) the Hamiltonian \(H\) is differentiable, then on any time interval where the orbit \(\gamma\) lies in \(U\), the inclusion relation reduces to the standard flow equation
        \[
            \dot{\gamma}(t) = J \nabla H(\gamma(t)).
        \]
    \item[Contact form] Not all points have a full 3-dimensional tangent space on the boundary of a polytope. So we instead look at each facet \(F\) separately, and define on the relative interior a 1-form \(\alpha_i = \lambda|_{\operatorname{relint}(F)}\). While in the smooth case we only had 3-dimensional tangent spaces, and still do on facet interiors, where \(\alpha\) is indeed a contact form, we now also have 2-,1-,0-faces. In particular, we may have lagrangian 2-faces where \(\omega|_F=0\).
    \item[Reeb orbit] On the facet interiors, we still get a Reeb vector \(R\) with \(\alpha(R)=1\), \((d\alpha)(R,\cdot)=0\). The Reeb vector is indeed constant on the facet interior and can be calculated as
    \[
        R_i = \frac{2}{h_i} J n_i.
    \]
        \begin{proof}
            \begin{align*}
                \alpha_i(R_i) &= \lambda_z(R_i) = \tfrac12 \inner{Jz}{R_i} = \tfrac12 \inner{z, \tfrac{2}{h_i} J^T J n_i} = \tfrac{1}{h_i} \inner{z, n_i} = 1, \\
                &\quad \text{since } z \in F_i \implies \inner{z}{n_i} = h_i. \\
                \omega_z(R_i, y) &= \inner{J R_i}{y} = \inner{J \tfrac{2}{h_i} J n_i}{y} = \frac{-2}{h_i} \inner{n_i}{y} = 0, \\
                &\quad \text{since } y \in T_z F_i \implies \inner{n_i}{y} = 0.
            \end{align*}
        \end{proof}
    On the relative interior of the 2-,1-,0-faces we don't have a full tangent space to define a Reeb vector field. Instead, we take just as for the subdifferential the convex hull of all incident facet Reeb vectors. So for a face \(F\) that is the intersection of facets \(F_{i_1}, \ldots, F_{i_k}\), we define the Reeb set as
    \[
        R_F = \operatorname{conv} \{ R_{i_1}, \ldots, R_{i_k} \}.
    \]
    A curve \(\gamma\) on the boundary of the polytope is then a (generalized) Reeb orbit iff
    \[
        \dot{\gamma}(t) \in \operatorname{conv} \{ R_i : \gamma(t) \in F_i \} \text{ for almost all } t \in \R/T\Z.
    \]
    \item[Action of a curve] The action of a curve is defined as in the smooth setting, for an absolutely continuous periodic curve \(\gamma \in W^{1,2}(\R/T\Z, \R^4)\) we have
    \[
        A(\gamma) = \int_0^T \lambda(\dot{\gamma}(t))\,dt.
    \]
\end{description}

The definitions of Ekeland-Hofer-Zehnder capacity and systolic ratio from the smooth setting carry over verbatim to the polytope setting, just replacing Reeb orbits by generalized Reeb orbits as defined above.

Several useful lemmas and theorems also carry over to the polytope setting, which we state and prove in the next section.
There are however also additional considerations to be made, mainly with regards to the 2-,1-,0-faces where no single Reeb vector is defined, but rather a convex set of possible Reeb flow velocities.

\section{Reeb Dynamics on Polytopes}
In this section we state and prove several useful lemmas and theorems about Reeb dynamics on polytopes.
Various of the statements are generalizations of well-known results from the smooth setting to the polytope setting.
Some discuss the behavior on lower-dimensional faces, which can be understood as the limit behavior of the smooth setting.

As usual we keep the polytope \(K \subset \R^4\) fixed.

\begin{lemma}[Action of Reeb orbits]
    Any Reeb orbit \(\gamma\) with period \(T\) on the boundary of the polytope \(K\) bounds a disk-like surface \(S \subset K\) and has action
    \[
        A(\gamma) = \int_\gamma \lambda = \int_S \omega = T.
    \]
\end{lemma}
\begin{proof}
    Smooth case proof sketch:
    The homology of \(K\) is trivial, so we get a disk. Apply Stokes' theorem. Integrate along the Reeb orbit with \(\lambda(R)=1\).
    For the polytope case the same arguments go through, we just note that \(\lambda(R)=1\) is true for all generalized Reeb vectors \(R \in \operatorname{conv}\{R_i : \gamma(t) \in F_i\}\) by linearity of \(\lambda\).
\end{proof}

\begin{theorem}[Existence of Reeb orbits on polytopes]
    There exists at least one closed Reeb orbit.
\end{theorem}
\begin{proof}
    Omitted.
\end{proof}

\begin{lemma}[Non-zero Reeb vectors]
    The Reeb set \(R_F\) of any face \(F\) of the polytope \(K\) does not contain the zero vector.
\end{lemma}
\begin{proof}
    Assume contraposition. Then some positive linear combination of the facet Reeb vectors equals zero. Since \(2/h_i\) is positive, and \(J\) is linear invertible, this implies that some positive linear combination of the facet normals equals zero.
    This contradicts that \(K\) is convex with non-empty interior.
\end{proof}

We now discuss the combinatorical structure of Reeb orbits on polytopes. This is something new that does not exist in the smooth setting, since there all points have a full 3-dimensional tangent space and a unique Reeb vector.

\begin{lemma}[Reeb flow on 3-Faces]
    Reeb orbits move along the facet interiors with constant velocity given by the facet Reeb vector.
\end{lemma}
\begin{proof}
    Follows directly from the definition of the Reeb vector on facet interiors, and the Reeb flow inclusion relation.
\end{proof}

\begin{lemma}[Reeb flow on 2-Faces]
    Reeb orbits cross non-lagrangian 2-faces instantaneously.
    Reeb orbits enter and exit lagrangian 2-faces from and into the boundary 1-faces or 0-faces. Their path inside the lagrangian 2-face is not uniquely defined, and can be any absolutely continuous \(W^{1,2}\) curve with velocities from \(\operatorname{conv}\{R_i, R_j\}\) that stays in the 2-face interior except for the endpoints.
\end{lemma}
\begin{proof}
    Let \(F_{ij} = F_i \cap F_j\) be a 2-face. Note that the Reeb vectors are orthogonal to their facet normals, i.e. lie in the tangent space of the facet.
    The two facet normals are by irredundancy linearly independent, so the tangent space \(T_z F_{ij}\) is given as the orthogonal complement of \(\operatorname{span}\{n_i, n_j\}\).
    Non-lagrangianness is equivalent to \(\omega(J n_i,J n_j) = \omega(n_i,n_j) \neq 0\).

    In the lagrangian case, the local neighborhood of a point on the 2-face's interior looks like this: We have the adjacent two facet interior, where the Reeb flow is parallel to the 2-face. So no orbit can enter or exit from the facet interiors into the 2-face interior. On the 2-face, the convex combinations of the two Reeb vectors are all tangent, so the Reeb flow can move in any convex direction inside the 2-face, as outlined in the lemma statement.

    In the non-lagrangian case, we have \(\omega(n_i,n_j) \neq 0\). No positive combination of the two Reeb vectors is orthogonal to both normals, so the Reeb flow cannot move inside the 2-face. Instead, it crosses the 2-face instantaneously. 
\end{proof}

\begin{lemma}[Reeb flow on 1-Faces]
    Reeb orbits can enter and exit 1-faces from and into the adjacent facets, 2-faces, or the bounding 0-faces. Their path inside the 1-face is not uniquely defined, and can be any absolutely continuous \(W^{1,2}\) curve with velocities from the convex hull of all incident facet Reeb vectors. All valid velocities point in the same direction.
\end{lemma}
\begin{proof}
    The set of Reeb vectors is convex and does not contain zero, so all valid velocities point in the same direction.
\end{proof} 

\begin{lemma}[Reeb flow on 0-Faces]
    Reeb orbits cross 0-faces instantaneously.
\end{lemma}
\begin{proof}
    Trivially, zero velocity is not allowed, so the Reeb flow cannot stay in the 0-face.
\end{proof}

For the 1-faces and lagrangian 2-faces we have seen that the Reeb flow has a rather high degree of freedom.
We however want to algorithmically compute with Reeb orbits, so some statement about representative orbits that are more easy to handle is needed.

\begin{definition}[Polygonal Reeb Orbit]
    A Reeb orbit \(\gamma\) on the boundary of the polytope \(K\) is called \emph{polygonal} iff
    \begin{itemize}
        \item it has piecewise constant velocity, i.e. there exists a finite partition \(0=t_0 < t_1 < \ldots < t_m = T\) of the period \(T\), such that for all \(k=0,\ldots,m-1\) the velocity \(\dot{\gamma}\) is constant on the open interval \((t_k, t_{k+1})\).
        \item in each time interval \((t_k, t_{k+1})\), the velocity \(\dot{\gamma}(t)\) is equal to some facet Reeb vector \(R_i\), instead of just a convex combination of multiple facet Reeb vectors.
    \end{itemize}
\end{definition}

Polygonal Reeb orbits are easier to handle algorithmically, e.g. using a list of breakpoints with breaking times.

\begin{theorem}[Homotopy to Polygonal Reeb Orbits]
    For any Reeb orbit \(\gamma_0\) on the boundary of the polytope \(K\), there exists a homotopy to a polygonal Reeb orbit \(\gamma_1\).
    The homotopy preserves the action, i.e. \(A(\gamma_s) = A(\gamma_0)\) for all \(s \in [0,1]\).
\end{theorem}
\begin{proof}
    The basic idea is to homotope a non-straight curve segment inside a 1-face or inside a lagrangian 2-face to a piecewise linear curve that uses only the facet Reeb vectors as velocities.
    To do this we need to find some polygonal Reeb orbit in the neighborhood of the original Reeb orbit. This is doable because locally, around some point in the face interior, we can consider the face boundary to be far away, and then just split the convex combination of Reeb vectors into distinct segments, where each segment has a pure velocity, but also only a fraction of the time.
    There also needs to be some compactness argument about how there's only finitely many times the orbit changes faces, which can come from observing that each point \(z\) has an infimum time along pure velocity Reeb flows to return to the face it is on, after leaving it. Some local consideration shows that the infimium is positive, and so the global infimum over the compact polytope surface is also positive, and so only finitely many face changes can happen in finite time.
    After finding such a polygonal Reeb orbit \(\gamma_1\) close to the original orbit \(\gamma_0\), we can do a linear homotopy between the two curves, which preserves the action since \(\lambda\) is linear in the velocity.
\end{proof}

In particular, we can restrict our search for action-minimizing Reeb orbits to polygonal Reeb orbits only.

Finally, there is a rather powerful theorem from \cite{HK2017} that shows that we can furthermore restrict our search to polygonal Reeb orbits that use each facet Reeb vector at most once.

\begin{definition}[Simple Reeb Orbit]
    A polygonal Reeb orbit \(\gamma\) on the boundary of the polytope \(K\) is called \emph{simple} iff it uses each facet Reeb vector at most once.
    I.e. for each facet \(F_i\) of the polytope, there exists at most one time interval \((t_k, t_{k+1})\) in which the velocity \(\dot{\gamma}(t)\) equals the facet Reeb vector \(R_i\).
    Note that the breaking points of the polygonal Reeb orbit need not correspond to a change in what facets are incident, so a simple polygonal Reeb orbit can cease to use a facet Reeb vector \(R_i\) yet still lie on the facet \(F_i\) for some time.
\end{definition}

\begin{theorem}[Existence of Simple Action-Minimizing Reeb Orbits]
    Any action-minimizing Reeb orbit on the boundary of the polytope \(K\) is homotopic to a simple polygonal Reeb orbit that also minimizes the action.
    The homotopy is action-preserving.
\end{theorem}

We defer the proof to \todoref{app:clarke-dual-principle}.
We need the machinery of the Clarke dual action principle in order to prove this theorem.
The original proof was given in \cite{HK2017}, with a discrete surgery on the polygonal Reeb orbit.
We wrote down the proof in more detail, and extracted a continuous homotopy instead of a discrete surgery.

\section{Trivialization of 2-Face Tangent Spaces}\label{sec:trivialization}

For algorithmic purposes, we need to track how Reeb orbits evolve as they cross non-Lagrangian 2-faces.
This requires a trivialization of the contact structure over 2-faces.
The construction below follows~\cite{CH2021}, adapted to make the computational aspects explicit.

\subsection{Quaternion Structure on \texorpdfstring{\(\R^4\)}{R4}}

The standard symplectic structure on \(\R^4\) extends to a quaternionic structure.
We define three matrices that satisfy the quaternion algebra relations \(I^2 = J^2 = K^2 = IJK = -\Id\):
\begin{align}
  \QuatI &= J = \begin{pmatrix} 0 & 0 & -1 & 0 \\ 0 & 0 & 0 & -1 \\ 1 & 0 & 0 & 0 \\ 0 & 1 & 0 & 0 \end{pmatrix}, \label{eq:quat-i} \\
  \QuatJ &= \begin{pmatrix} 0 & -1 & 0 & 0 \\ 1 & 0 & 0 & 0 \\ 0 & 0 & 0 & 1 \\ 0 & 0 & -1 & 0 \end{pmatrix}, \label{eq:quat-j} \\
  \QuatK &= \begin{pmatrix} 0 & 0 & 0 & -1 \\ 0 & 0 & 1 & 0 \\ 0 & 1 & 0 & 0 \\ -1 & 0 & 0 & 0 \end{pmatrix}. \label{eq:quat-k}
\end{align}

These satisfy \(\QuatI = \QuatJ \QuatK\), \(\QuatJ = \QuatK \QuatI\), \(\QuatK = \QuatI \QuatJ\), and all are orthogonal: \(Q^T Q = \Id\) for \(Q \in \{\QuatI, \QuatJ, \QuatK\}\).

Note that \(\QuatI = J\) coincides with the standard almost complex structure from Section~1.
We use the notation \(\QuatI, \QuatJ, \QuatK\) following~\cite{CH2021} to emphasize the quaternionic algebra structure.

\begin{lemma}[Orthonormal frame from unit vector]\label{lem:quat-frame}
  For any unit vector \(\nu \in S^3 \subset \R^4\), the vectors \(\{\nu, \QuatI\nu, \QuatJ\nu, \QuatK\nu\}\) form an orthonormal basis of \(\R^4\).
\end{lemma}
\begin{proof}
  Since \(\QuatI, \QuatJ, \QuatK\) are orthogonal matrices, we have \(\abs{\QuatI\nu} = \abs{\QuatJ\nu} = \abs{\QuatK\nu} = 1\).
  For orthogonality, observe that \(\QuatI^T = -\QuatI\) (since \(\QuatI^2 = -\Id\) and \(\QuatI\) is orthogonal).
  Therefore:
  \[
    \inner{\nu}{\QuatI\nu} = \inner{\QuatI^T\nu}{\nu} = -\inner{\QuatI\nu}{\nu} = -\inner{\nu}{\QuatI\nu},
  \]
  which implies \(\inner{\nu}{\QuatI\nu} = 0\). The same argument applies to \(\QuatJ\) and \(\QuatK\).
  For the remaining pairs, we compute \(\inner{\QuatI\nu}{\QuatJ\nu} = \inner{\nu}{\QuatI^T\QuatJ\nu} = -\inner{\nu}{\QuatI\QuatJ\nu} = -\inner{\nu}{\QuatK\nu} = 0\).
  The pairs \((\QuatJ\nu, \QuatK\nu)\) and \((\QuatK\nu, \QuatI\nu)\) follow by cyclic permutation.
\end{proof}

\subsection{The 2-Face Tangent Space}

Let \(F = F_i \cap F_j\) be a non-Lagrangian 2-face with entry facet \(F_i\) (normal \(n_{\mathrm{entry}}\)) and exit facet \(F_j\) (normal \(n_{\mathrm{exit}}\)).
The tangent space of \(F\) is the 2-dimensional subspace:
\[
  TF = \ker(n_{\mathrm{entry}}) \cap \ker(n_{\mathrm{exit}}) = \{ V \in \R^4 : \inner{V}{n_{\mathrm{entry}}} = 0 \text{ and } \inner{V}{n_{\mathrm{exit}}} = 0 \}.
\]

\begin{definition}[Trivialization of 2-face tangent space]\label{def:trivialization}
  For a unit vector \(\nu \in S^3\), define the \emph{quaternionic trivialization} \(\tau_\nu: \R^4 \to \R^2\) by
  \[
    \tau_\nu(V) = \bigl( \inner{V}{\QuatJ\nu}, \inner{V}{\QuatK\nu} \bigr).
  \]
  This is a linear map that projects onto the \(\QuatJ\nu, \QuatK\nu\) coordinates in the orthonormal frame from Lemma~\ref{lem:quat-frame}.
\end{definition}

When restricted to a 2-face tangent space \(TF\), the trivialization \(\tau_{n_{\mathrm{exit}}}\) becomes an isomorphism \(TF \xrightarrow{\sim} \R^2\), provided the 2-face is non-Lagrangian.
To compute the inverse map explicitly, we need to identify basis vectors for \(TF\) that have prescribed trivialization coordinates.

\begin{proposition}[Explicit basis for 2-face tangent space]\label{prop:explicit-basis}
  Let \(F = F_i \cap F_j\) be a non-Lagrangian 2-face with normals \(n_{\mathrm{entry}}, n_{\mathrm{exit}}\).
  Define the \(4 \times 4\) matrix
  \[
    M = \begin{pmatrix}
      n_{\mathrm{entry}}^T \\
      n_{\mathrm{exit}}^T \\
      (\QuatJ n_{\mathrm{exit}})^T \\
      (\QuatK n_{\mathrm{exit}})^T
    \end{pmatrix}.
  \]
  Then \(M\) is invertible for non-Lagrangian 2-faces, and the last two columns of \(M^{-1}\) give basis vectors \(b_1, b_2 \in TF\) such that:
  \begin{enumerate}
    \item \(b_1, b_2 \in TF\), i.e., \(\inner{b_k}{n_{\mathrm{entry}}} = \inner{b_k}{n_{\mathrm{exit}}} = 0\) for \(k=1,2\).
    \item \(\tau_{n_{\mathrm{exit}}}(b_1) = (1, 0)\) and \(\tau_{n_{\mathrm{exit}}}(b_2) = (0, 1)\).
  \end{enumerate}
  Thus for any \(V \in TF\), we have
  \[
    V = \inner{V}{\QuatJ n_{\mathrm{exit}}} \cdot b_1 + \inner{V}{\QuatK n_{\mathrm{exit}}} \cdot b_2.
  \]
\end{proposition}
\begin{proof}
  The rows of \(M\) are the four vectors \(n_{\mathrm{entry}}, n_{\mathrm{exit}}, \QuatJ n_{\mathrm{exit}}, \QuatK n_{\mathrm{exit}}\), viewed as linear functionals.
  Thus \(M\) maps a vector \(V \in \R^4\) to its coordinates
  \[
    MV = \bigl(\inner{V}{n_{\mathrm{entry}}}, \inner{V}{n_{\mathrm{exit}}}, \inner{V}{\QuatJ n_{\mathrm{exit}}}, \inner{V}{\QuatK n_{\mathrm{exit}}}\bigr)^T.
  \]
  Denote the columns of \(M^{-1}\) by \(c_1, c_2, c_3, c_4\). The defining property \(M c_k = e_k\) means that each \(c_k\) has value 1 when paired with the \(k\)-th row of \(M\) and value 0 for all other rows.

  Setting \(b_1 = c_3\): the conditions \(M c_3 = e_3\) state that \(\inner{c_3}{n_{\mathrm{entry}}} = 0\), \(\inner{c_3}{n_{\mathrm{exit}}} = 0\), \(\inner{c_3}{\QuatJ n_{\mathrm{exit}}} = 1\), and \(\inner{c_3}{\QuatK n_{\mathrm{exit}}} = 0\).
  The first two conditions place \(c_3\) in \(\ker(n_{\mathrm{entry}}) \cap \ker(n_{\mathrm{exit}}) = TF\), while the last two give \(\tau_{n_{\mathrm{exit}}}(c_3) = (1, 0)\).
  The same analysis with \(b_2 = c_4\) yields \(b_2 \in TF\) and \(\tau_{n_{\mathrm{exit}}}(b_2) = (0, 1)\).

  For invertibility, observe that \(M\) is singular iff its rows are linearly dependent.
  The last two rows \(\QuatJ n_{\mathrm{exit}}\) and \(\QuatK n_{\mathrm{exit}}\) span a 2-dimensional subspace orthogonal to both \(n_{\mathrm{exit}}\) and \(\QuatI n_{\mathrm{exit}}\) (by Lemma~\ref{lem:quat-frame}).
  Linear dependence occurs iff this 2-dimensional subspace lies in \(\mathrm{span}\{n_{\mathrm{entry}}, n_{\mathrm{exit}}\}\).
  Since the orthogonal complement of \(\mathrm{span}\{\QuatJ n_{\mathrm{exit}}, \QuatK n_{\mathrm{exit}}\}\) is \(\mathrm{span}\{n_{\mathrm{exit}}, \QuatI n_{\mathrm{exit}}\}\), containment would require \(TF = \mathrm{span}\{\QuatJ n_{\mathrm{exit}}, \QuatK n_{\mathrm{exit}}\}\), which happens exactly when \(\QuatI n_{\mathrm{exit}} \in \mathrm{span}\{n_{\mathrm{entry}}, n_{\mathrm{exit}}\}\).
  This last condition is equivalent to \(\omega(n_{\mathrm{entry}}, n_{\mathrm{exit}}) = \inner{\QuatI n_{\mathrm{entry}}}{n_{\mathrm{exit}}} = 0\), the Lagrangian case.
\end{proof}

\begin{remark}[Why explicit basis vectors are necessary]\label{rem:explicit-basis-pitfall}
  \textbf{Common pitfall:} One might naively expect the inverse trivialization to be
  \[
    \tau_{n_{\mathrm{exit}}}^{-1}(a, b) \stackrel{?}{=} a \cdot \QuatJ n_{\mathrm{exit}} + b \cdot \QuatK n_{\mathrm{exit}}.
  \]
  This formula is \textbf{incorrect}.
  The image of this map lies in \(\mathrm{span}\{\QuatJ n_{\mathrm{exit}}, \QuatK n_{\mathrm{exit}}\}\), which is the orthogonal complement of \(\mathrm{span}\{n_{\mathrm{exit}}, \QuatI n_{\mathrm{exit}}\}\).
  Meanwhile, the 2-face tangent space is \(TF = \ker(n_{\mathrm{entry}}) \cap \ker(n_{\mathrm{exit}})\), the orthogonal complement of \(\mathrm{span}\{n_{\mathrm{entry}}, n_{\mathrm{exit}}\}\).
  These two 2-dimensional subspaces coincide only in the Lagrangian case \(\omega(n_{\mathrm{entry}}, n_{\mathrm{exit}}) = 0\), which is precisely when the trivialization is ill-defined.

  The explicit basis construction in Proposition~\ref{prop:explicit-basis} resolves this by computing vectors \(b_1, b_2\) that lie in \(TF\) while having the desired trivialization coordinates \((1,0)\) and \((0,1)\).
\end{remark}

\subsection{Transition Matrices}\label{sec:transition-matrix}

When a Reeb orbit crosses a non-Lagrangian 2-face \(F = F_i \cap F_j\), the trivialization coordinates transform by a \emph{transition matrix}.

\begin{definition}[Transition matrix]
  For a non-Lagrangian 2-face \(F\) with entry normal \(n_{\mathrm{entry}}\) and exit normal \(n_{\mathrm{exit}}\), the transition matrix \(\psi_F \in \mathrm{GL}(2, \R)\) is defined by
  \[
    \psi_F = \tau_{n_{\mathrm{exit}}} \circ \tau_{n_{\mathrm{entry}}}^{-1}: \R^2 \to \R^2.
  \]
  That is, \(\psi_F\) converts coordinates in the entry trivialization to coordinates in the exit trivialization.
\end{definition}

\begin{theorem}[Properties of transition matrices]\label{thm:transition-matrix}
  For a non-Lagrangian 2-face \(F\), the transition matrix \(\psi_F\) satisfies:
  \begin{enumerate}
    \item \(\psi_F \in \mathrm{Sp}(2) = \mathrm{SL}(2, \R)\), i.e., \(\det(\psi_F) = 1\).
    \item \(\mathrm{tr}(\psi_F) = 2\inner{n_{\mathrm{entry}}}{n_{\mathrm{exit}}}\).
    \item For adjacent facets of a convex polytope, \(\abs{\mathrm{tr}(\psi_F)} < 2\).
  \end{enumerate}
\end{theorem}
\begin{proof}
  \textbf{(1)} The transition matrix preserves the symplectic form on \(TF\) induced by restriction of \(\omega\).
  Since \(\omega\) is non-degenerate on \(TF\) (by non-Lagrangianness), \(\psi_F\) preserves a non-degenerate 2-form, hence \(\det(\psi_F) = 1\).

  \textbf{(2)} We use the explicit formula from~\cite{CH2021} (Definition~2.15 and Lemma~2.17).
  Let \(a_1 = \inner{n_{\mathrm{entry}}}{n_{\mathrm{exit}}}\), \(a_2 = \inner{n_{\mathrm{entry}}}{\QuatI n_{\mathrm{exit}}}\), \(a_3 = \inner{n_{\mathrm{entry}}}{\QuatJ n_{\mathrm{exit}}}\), \(a_4 = \inner{n_{\mathrm{entry}}}{\QuatK n_{\mathrm{exit}}}\).
  Since \(\{n_{\mathrm{exit}}, \QuatI n_{\mathrm{exit}}, \QuatJ n_{\mathrm{exit}}, \QuatK n_{\mathrm{exit}}\}\) is an orthonormal basis, we have \(a_1^2 + a_2^2 + a_3^2 + a_4^2 = 1\).
  The transition matrix is:
  \[
    \psi_F = \frac{1}{a_2} \begin{pmatrix}
      a_1 a_2 - a_3 a_4 & -(a_2^2 + a_4^2) \\
      a_2^2 + a_3^2 & a_1 a_2 + a_3 a_4
    \end{pmatrix}.
  \]
  Computing the trace: \(\mathrm{tr}(\psi_F) = \frac{1}{a_2}(a_1 a_2 - a_3 a_4 + a_1 a_2 + a_3 a_4) = 2a_1 = 2\inner{n_{\mathrm{entry}}}{n_{\mathrm{exit}}}\).

  \textbf{(3)} For adjacent facets of a convex polytope, the normals point ``outward,'' so \(\inner{n_{\mathrm{entry}}}{n_{\mathrm{exit}}} \in (-1, 1)\).
  Equality \(\inner{n_{\mathrm{entry}}}{n_{\mathrm{exit}}} = \pm 1\) would require the normals to be parallel or anti-parallel, but then the facets would be parallel and couldn't share a 2-face.
  Thus \(\abs{\mathrm{tr}(\psi_F)} = 2\abs{\inner{n_{\mathrm{entry}}}{n_{\mathrm{exit}}}} < 2\).
\end{proof}

\subsection{Rotation Numbers}\label{sec:rotation-number}

The trace condition \(\abs{\mathrm{tr}(\psi_F)} < 2\) means \(\psi_F\) is an elliptic element of \(\mathrm{SL}(2,\R)\), i.e., conjugate to a rotation.

\begin{definition}[Rotation number of a 2-face]
  For a non-Lagrangian 2-face \(F\) with transition matrix \(\psi_F\), define the \emph{rotation number} (in turns, i.e., \(1 = 2\pi\)) as
  \[
    \rho(F) = \frac{1}{2\pi} \arccos\Bigl(\frac{\mathrm{tr}(\psi_F)}{2}\Bigr) = \frac{1}{2\pi} \arccos\bigl(\inner{n_{\mathrm{entry}}}{n_{\mathrm{exit}}}\bigr).
  \]
\end{definition}

\begin{corollary}[Rotation number range]\label{cor:rotation-range}
  For any non-Lagrangian 2-face of a convex polytope, the rotation number satisfies \(\rho(F) \in (0, \tfrac{1}{2})\).
\end{corollary}
\begin{proof}
  By Theorem~\ref{thm:transition-matrix}(3), \(\mathrm{tr}(\psi_F)/2 = \inner{n_{\mathrm{entry}}}{n_{\mathrm{exit}}} \in (-1, 1)\).
  Therefore \(\arccos(\mathrm{tr}(\psi_F)/2) \in (0, \pi)\), giving \(\rho(F) \in (0, \tfrac{1}{2})\).
\end{proof}

\begin{remark}[Geometric interpretation]
  The rotation number measures the angle (in turns) between the contact planes at entry and exit.
  More precisely, when the Reeb orbit enters the 2-face and immediately exits, the contact structure ``twists'' by angle \(2\pi\rho(F)\).
  For closed orbits, the total rotation \(\sum_F \rho(F)\) over all crossed 2-faces must satisfy constraints from Conley--Zehnder index theory; see~\cite{CH2021}, Proposition~1.10.
\end{remark}

\section{Polytopes as the Limit of Smooth Bodies}

With the preliminary discussion of Reeb dynamics on polytopes done, we now move to show that it indeed arises as the limit of smooth bodies.
We largely follow the treatment in \cite{CH2021}.

As a result of this section, we will see a few more useful lemmas about action-minimizing generalized Reeb orbits on polytopes, which we later employ in our algorithms.

Fix, as usual, a polytope \(K \subset \R^4\).

\begin{definition}[\(\varepsilon\)-smoothing]
    For \(\varepsilon>0\) define the \(\varepsilon\)-smoothing of the polytope \(K\) as
    \[
        K_\varepsilon = \{ z \in \R^4 : \exists z' \in K \text{ with } \abs{z-z'} \leq \varepsilon \}.
    \]
\end{definition}

We quickly see a few basic properties of the \(\varepsilon\)-smoothing:

\begin{lemma}[Properties of \(\varepsilon\)-smoothing]
    The \(\varepsilon\)-smoothing \(K_\varepsilon\) of the polytope \(K\) has the following properties:
    \begin{itemize}
        \item It is the maximum body with Hausdorff distance at most \(\varepsilon\) from \(K\), and contains all other such bodies.
        \item It is monotone wrt inclusion: \(K \subset K_\varepsilon \subset K_{\varepsilon'}\) for \(\varepsilon < \varepsilon'\).
        \item It converges to the polytope in the Hausdorff metric: \(\lim_{\varepsilon \to 0} d_H(K, K_\varepsilon) = 0\).
        \item It is convex, bounded, closed, and star-shaped wrt the origin.
        \item It has a \(C^{1}\) boundary.
    \end{itemize}
\end{lemma}


% TODO: replace this placeholder with the full HK/CH combinatorial formula exposition.
\begin{theorem}[HK combinatorial formula, 4D]\label{thm-hk-combinatorial}
  For a centrally symmetric polytope \(K\subset \R^4\) with facet normals \(n_i\) and heights \(h_i\), the discrete maximization problem from Haim--Kislev/Colesanti--Hug yields the EHZ capacity via their combinatorial formula. A full statement and proof will be inserted once the derivation is finalized.
\end{theorem}
