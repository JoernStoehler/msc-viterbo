% !TeX root = ../main.tex
This chapter sets notation and definitions for the rest of the thesis. We follow primarily \cite{CH2021} and secondarily \cite{HK2017,HK2024} in our choice of notation. We assume the reader is already familiar with basic symplectic geometry in a smooth setting (Reeb vector fields, contact forms, symplectic capacities, Hamiltonian dynamics).

\section{Standard symplectic \texorpdfstring{$\R^4$}{R4}}
We work in the standard \(\R^4\) setting. Let
\begin{description}
  \item[Coordinates] \(z=(q_1,q_2,p_1,p_2)\).
  \item[Inner product] \(\inner{x}{y} = x^T y\).
  \item[Norm] \(\abs{x}=\sqrt{\inner{x}{x}}\).
  \item[Volume form] \(\vol = \prod_{i=1}^2 dq_i \wedge dp_i\).
  \item[Almost complex structure] \(\Jmat = \begin{pmatrix}0 & -I_2 \\ I_2 & 0\end{pmatrix}\) so that \(\Jmat^2 = -I_4\) and \(J z = (-p_1,-p_2,q_1,q_2)\).
  \item[Symplectic form] \(\omegaform = \sum_{i=1}^2 dq_i \wedge dp_i\), equivalently \(\omegaform(x,y)=\tfrac12 \inner{\Jmat x}{y}\).
  \item[Liouville 1-form] \(\lambdaform = \tfrac12 \sum_{i=1}^2 (q_i\,dp_i - p_i\,dq_i)\), so \(\lambdaform_z(\dot z) = \tfrac12 \inner{\Jmat z}{\dot z}\) and \(d\lambdaform = \omegaform\).
  \item[Lagrangian 2-plane] An affine 2-plane \(L\subset\R^4\) is Lagrangian iff \(\omegaform|_L \equiv 0\).
\end{description}

\section{Convex bodies and polytopes}
For an outward unit normal \(n\in\R^4\), \(\abs{n}=1\), and height \(h\in\R\), the half-space \(\{x: \inner{x}{n} \le h\}\) has boundary hyperplane \(\{x: \inner{x}{n}=h\}\). If \(h>0\) the half-space contains the origin; we then call it \/positive\/. A bounded, convex, star-shaped set \(K\subset\R^4\) that contains the origin in its interior is called \emph{admissible}. Any bounded convex body with non-empty interior can be translated to an admissible one.

\paragraph{Irredundant H-representation.} Any admissible polytope \(K\) has a unique representation as the intersection of finitely many positive half-spaces with minimal cardinality. Writing the outward unit normals and heights as \((n_i,h_i)_{i=1}^F\), \(\abs{n_i}=1\), \(h_i>0\), we have
\[
  K = \bigcap_{i=1}^F \{x: \inner{x}{n_i} \le h_i\}.
\]
Any such representation with bounded \(K\) defines an admissible polytope.

\paragraph{Faces.} We write \(\partial K\) for the boundary. Facets (3-faces) are \(F_i = K \cap \{x: \inner{x}{n_i}=h_i\}\). Where multiple hyperplanes meet we have 2-, 1-, and 0-faces. By irredundancy, every 2-face is the intersection of two unique facets. The \(n_i, h_i\) are the facet normals and heights.

\paragraph{Support and gauge.} For an admissible body \(K\subset\R^4\), the support function is \(\support_K(v) = \max_{x\in K} \inner{x}{v}\). For polytopes, \(\support_K(n_i) = h_i\). The gauge is \(\gauge_K(v) = \min\{r>0: v \in rK\}\), so \(\gauge_K(x)=1\) on \(\partial K\).

\paragraph{Polar body.} The polar body is \(K^{\polar} = \{y: \support_K(y) \le 1\}\). For admissible polytopes, the polar is admissible with vertices at the points \(n_i/h_i\).

\section{Smooth vs.~combinatorial viewpoints}
Admissible polytopes approximate any bounded convex star-shaped body arbitrarily well in the Hausdorff metric. Statements about polytope data should match the limits of their smoothings.

\paragraph{\(\varepsilon\)-smoothings.} For an admissible polytope \(K\) and \(\varepsilon>0\), set \(K_\varepsilon := \{x \in \R^4 : \mathrm{dist}(x,K) \le \varepsilon\}\). Then \(K_\varepsilon\) is admissible with \(C^{1,1}\) boundary. Any sequence of admissible bodies converging to \(K\) in the Hausdorff metric lies in some sequence of \(K_{\varepsilon_j}\) with \(\varepsilon_j\to0\).

\section{Hamiltonian dynamics}
Fix an admissible polytope \(K\) with irredundant data \((n_i,h_i)_{i=1}^F\).

\paragraph{Standard Hamiltonian.} We set \(\Ham = \gauge_K^2\), the square of the gauge. It is 2-homogeneous, convex, and piecewise quadratic for polytopes. Derivatives fail on rays through the 2-faces of \(K\).

\paragraph{Regular case.} For \(\Ham\in C^1(\R^4)\), the Hamiltonian vector field satisfies \(\omegaform(X_\Ham,\cdot) = d\Ham(\cdot)\), equivalently \(X_\Ham = \Jmat \nabla \Ham\). The flow \(\dot x = X_\Ham(x)\) preserves level sets of \(\Ham\).

\paragraph{Polytope case.} For polytopes \(\Ham\) is not differentiable everywhere; we use the subdifferential \(\partial \Ham(x)\) and the inclusion
\[
  \dot x(t) \in \Jmat \partial \Ham(x(t)) \quad \text{a.e.}
\]
Solutions in \(W^{1,2}(\R,\R^4)\) exist for any initial condition and all time but need not be unique. The boundary \(\partial K\) is invariant.

\paragraph{Facet velocities.} At any \(x\in\partial K\), the subgradients come from incident facets. Thus
\[
  \partial \Ham(x) = \operatorname{conv}\{ (2/h_i)\, n_i : x \in F_i \},\qquad p_i := (2/h_i)\, \Jmat n_i.
\]
On the interior of \(F_i\) the velocity is the constant \(p_i\).

\section{Reeb dynamics and closed characteristics}
Let \(\partial K\) be a contact-type hypersurface with contact form \(\alpha = \lambdaform|_{\partial K}\).

\paragraph{Regular case.} At \(x\in\partial K\) with outward unit normal \(n_x\), any \(v\in T_x\partial K\) satisfies \(\inner{\Jmat v}{\Jmat n_x}=\inner{v}{n_x}=0\). Normalizing to \(\alpha(R)=1\) gives
\[
  \Reeb(x) = \frac{2}{\inner{x}{n_x}}\, \Jmat n_x.
\]
The Reeb flow \(\dot x = \Reeb(x)\) preserves \(\alpha\) and \(d\alpha\). Periodic solutions are Reeb orbits. Closed characteristics are loops \(\gamma\) with \(\dot\gamma(t)\) parallel and positively oriented to \(\Reeb\).

\paragraph{Polytope case.} On facet interiors the Reeb vector field matches the Hamiltonian velocity. Generalized closed characteristics are loops \(\gamma \in W^{1,2}(\T,\partial K)\) with
\[
  \dot\gamma(t) \in \mathrm{cone}\{p_i : \gamma(t) \in F_i\} \quad \text{a.e.}
\]
Thus generalized Reeb orbits, closed characteristics, and Hamiltonian orbits coincide up to orientation-preserving reparametrization. In the interior of \(F_i\) the orbit has a linear segment with velocity \(p_i\).

\begin{fact}\label{fact:closed-characteristic}
Any closed characteristic \(\gamma\) is parametrized uniquely as \(\gamma\in W^{1,2}(\T,\partial K)\) with period \(T>0\) such that \(\dot\gamma(t) \in \operatorname{conv}\{p_i : \gamma(t) \in F_i\}\) almost everywhere.
\end{fact}

\begin{lemma}[Flow on facets]\label{lem-facet-flow}
If an orbit \(\gamma\) meets the interior of a facet \(F_i\), it does so along a closed linear segment with velocity \(p_i\), entering and exiting at the boundary of \(F_i\) after finite time.
\end{lemma}
\begin{proof}
Immediate from the definition of \(p_i\) on facet interiors.
\end{proof}

At lower-dimensional faces the behavior depends on geometry.

\begin{lemma}[Flow on Lagrangian 2-faces]\label{lem-lagrangian-2face}
If a 2-face \(F_{ij}=F_i\cap F_j\) is Lagrangian, then \(p_i\) and \(p_j\) lie in its tangent plane. Locally the orbit may slide along \(F_{ij}\) with any \(W^{1,2}\) velocity in \(\operatorname{conv}\{p_i,p_j\}\). It enters and exits \(F_{ij}\) after finite time through its boundary (a 1- or 0-face).
\end{lemma}
\begin{proof}
Since \(p_i,p_j\) are tangent, the orbit cannot enter from facet interiors; it passes through the boundary. Also \(0\notin \operatorname{conv}\{p_i,p_j\}\), so \(\alpha\) integrates to a potential on \(F_{ij}\) and the orbit cannot stay forever.
\end{proof}

\begin{lemma}[Flow through non-Lagrangian 2-faces]\label{lem-nonlagrangian-2face}
If \(F_{ij}\) is non-Lagrangian, then the orbit crosses \(F_{ij}\) from one facet to the other at isolated times. The direction is determined by \(\omegaform(n_i,n_j)\): if \(\omegaform(n_i,n_j) > 0\) the orbit crosses from \(F_i\) to \(F_j\); if \(\omegaform(n_i,n_j)<0\) it crosses from \(F_j\) to \(F_i\).
\end{lemma}
\begin{proof}
We use \(\inner{\Jmat n_i}{n_j} = - \inner{\Jmat n_j}{n_i} = \omegaform(n_i,n_j)\). Its sign determines whether \(p_i\) points into or out of the half-space defined by \(F_j\). Locally the orbit consists of linear segments with velocities \(p_i\) and \(p_j\); touching times are isolated and the crossing direction follows the sign.
\end{proof}

\begin{lemma}[Flow on 1-faces]\label{lem-1face-flow}
If \(\gamma\) meets the interior of a 1-face \(F = \bigcap_{k=1}^m F_{i_k}\) (with \(m\ge3\)), there is a unique direction of flow along \(F\). Velocities may vary within the convex cone of incident \(p_{i_k}\). The orbit may enter or exit \(F\) through adjacent 0-, 2-, or 3-faces; the touching time is finite.
\end{lemma}
\begin{proof}
Convexity shows the normals do not convexly combine to zero; neither do the \(p_{i_k}\). Their convex cone lies on a unique half-line, giving the direction. Examples show trajectories can enter/exit as stated.
\end{proof}

\begin{lemma}[Flow on 0-faces]\label{lem-0face-flow}
No general statement beyond finiteness of touching time.
\end{lemma}

\subsection{Generic behavior}
We ask which degeneracies are generic.

\begin{lemma}[Non-Lagrangianness is generic]\label{lem-nonlagrangian-generic}
For a fixed number of facets (or vertices), polytopes with no Lagrangian 2-faces form an open dense set in the parameter space of facet normals/heights (respectively vertices).
\end{lemma}
\begin{proof}
Lagrangianness of \(F_{ij}\) is the single equation \(\omegaform(n_i,n_j)=0\), a codimension-one condition. A finite union of such subsets has complement open dense.
\end{proof}

\begin{conjecture}\label{conj-generic-0faces}
For a generic admissible polytope, no closed characteristic passes through a 0-face.
\end{conjecture}
\begin{remark}
This conjecture appears in \cite{CH2021} (Conjecture~1.26): “We expect that Type 2 combinatorial Reeb orbits do not exist for generic polytopes.”
\end{remark}

\subsection{Degeneracies of the action}
\begin{definition}[Polygonal orbit]\label{def-piecewise-constant-velocity}
A Hamiltonian/Reeb orbit is \emph{polygonal} if time can be partitioned into finitely many open intervals such that during each interval the incident facet set is constant and the velocity is a single incident \(p_i\). Breakpoints need not coincide with face changes.
\end{definition}

\begin{theorem}[Homotopy to polygonal orbit]\label{thm-homotopy-pl}
Any Hamiltonian/Reeb orbit \(\gamma\) is homotopic through Hamiltonian/Reeb orbits of the same action to a polygonal orbit \(\gamma'\).
\end{theorem}
\begin{proof}
The face incidence changes only at isolated times, yielding a finite partition. On 3-faces the velocity is constant. On Lagrangian 2-faces, write \(\dot\gamma = a(t)p_i + b(t)p_j\); keeping the integrals of \(a,b\) fixed preserves the action because \(\inner{\gamma(t)}{\Jmat p_i}\) and \(\inner{\gamma(t)}{\Jmat p_j}\) are constant along the 2-face. Homotope to step functions, giving a polygonal segment. An analogous argument with the convex cone works on 1-faces. Glue the homotopies over intervals.
\end{proof}

\begin{definition}[Simple orbit]\label{def-simple-orbit-2}
A polygonal Hamiltonian/Reeb orbit is \emph{simple} if each facet velocity \(p_i\) is used at most once.
\end{definition}

\begin{theorem}[Homotopy to simple orbit]\label{thm-min-action-simple}
Any Hamiltonian/Reeb orbit is homotopic through Hamiltonian/Reeb orbits to a simple orbit whose action is non-increasing along the homotopy.
\end{theorem}
\begin{proof}
See the argument in \cite{HK2017} using Clarke's dual principle; the homotopy can be arranged without increasing action.
\end{proof}

\begin{corollary}[Simple minimum-action orbit]\label{cor-simple-min-action}
There exists a minimum-action Hamiltonian/Reeb orbit that is simple.
\end{corollary}

\begin{example}[Viterbo counterexample degeneracy]\label{ex-viterbo-degeneracy}
The counterexample from \cite{HK2024} has multiple distinct minimum-action closed characteristics that are all homotopic: in the two-bounce orbits one bounce point moves along a pentagon edge while the other stays at a vertex, and vice versa; three-bounce orbits can be homotoped into a two-bounce orbit by moving neighboring bounce points to a common vertex.
\end{example}

\section{Action, EHZ capacity, and systolic ratio}
For a curve \(\gamma\) in \(\R^4\), the action is
\[
  A(\gamma) = \int_\gamma \lambdaform = \int_0^T \lambdaform_{\gamma(t)}(\dot\gamma(t))\,dt = \tfrac12 \int_0^T \inner{\Jmat\gamma(t)}{\dot\gamma(t)}\, dt.
\]
It is invariant under orientation-preserving reparametrization and changes sign if orientation is reversed.

For admissible bodies with enough regularity there is a closed characteristic minimizing the action; the minimum equals the Ekeland--Hofer--Zehnder capacity
\[
  \cEHZ(K) = \min\{A(\gamma) : \gamma \text{ closed characteristic on } \partial K\}.
\]
By continuity of symplectic capacities in the Hausdorff metric this definition extends to polytopes using generalized closed characteristics.

The systolic ratio is
\[
  \sys(K) = \frac{\cEHZ(K)^2}{2\, \vol(K)}.
\]
It is scale and translation invariant; for balls and cylinders of radius \(r\), \(\sys(B(r)) = \sys(Z(r)) = 1\).

\section{Viterbo conjecture (falsified)}
\begin{conjecture}[Viterbo]\label{conj-viterbo}
For any admissible body \(K \subset \R^4\), the systolic ratio satisfies \(\sys(K) \le 1\).
\end{conjecture}

\begin{example}[Counterexample]\label{ex-counterexample-viterbo}
\cite{HK2024} constructs an admissible polytope \(K \subset \R^4\) with \(\sys(K) > 1\), falsifying \cref{conj-viterbo}.
\end{example}

The existing literature includes further statements:

\begin{theorem}[Simplex case]\label{thm-simplex-viterbo}
For simplices (which, if star-shaped, are automatically admissible polytopes), \(\sys(K) \le 3/4 < 1\). Equality holds only for the orthonormal simplex with vertices \(\{0, e_1, e_2, e_3, e_4\}\) and its symplectomorphic images (translations and linear symplectic maps).
\end{theorem}

\begin{theorem}[Mahler conjecture in 2D]\label{thm-mahler-2d-viterbo}
For centrally symmetric convex polygons \(P = -P \subset \R^2\) we have \(\mathrm{area}(P)\,\mathrm{area}(P^{\polar}) \ge 8\), with equality for parallelograms. Then \(K = P \times P^{\polar} \subset \R^4\) satisfies \(\sys(K)\le1\), with equality preserved.
\end{theorem}

Trivial families of counterexamples arise from scaling, symplectomorphisms, and small perturbations of known counterexamples, since \(\sys\) is continuous in the Hausdorff metric. We seek nontrivial counterexamples and a systematic computational search.

\section{Clarke's dual action principle}
Fenchel duality yields
\[
  \gauge_K(x)^2 = \sup_y \big( \inner{x}{y} - \tfrac14 \support_K(y)^2 \big),\qquad
  \tfrac14 \support_K(y)^2 = \sup_x \big( \inner{x}{y} - \gauge_K(x)^2 \big),
\]
with equality iff \(y\in\partial \gauge_K^2(x)\) iff \(x\in\partial(\tfrac14 \support_K^2)(y)\), identifying supporting facets and contact points by Legendre duality.

Clarke's dual action principle (specialized to convex bodies) considers
\[
  E = \Big\{ z\in W^{1,2}([0,1],\R^4): \int_0^1 \dot z = 0,\ \int_0^1 \inner{-\Jmat\dot z}{z}\,dt = 1 \Big\},\qquad
  I_K(z)=\tfrac14\int_0^1 \support_K^2(-\Jmat\dot z(t))\,dt.
\]
Critical points of \(I_K\) correspond to generalized characteristics and
\[
  \cEHZ(K) = \inf_{z\in E} I_K(z).
\]
For polytopes \(\support_K\) is piecewise linear, so \(I_K\) is piecewise quadratic in \(\dot z\); minimizers can be taken piecewise affine, leading to the combinatorial model below \cite{HK2017}.

\section{HK/CH combinatorial capacity formula (4D)}
Let \((n_i,h_i)_{i=1}^F\) be the outward unit normals and heights of an admissible polytope \(K\subset\R^4\). Define coefficients \(\beta_i\ge0\) with
\[
  \sum_i \beta_i h_i = 1,\qquad \sum_i \beta_i n_i = 0.
\]
For a permutation \(\sigma\in S_F\), set
\[
  Q(\sigma,\beta)=\sum_{j<i} \beta_{\sigma(i)}\,\beta_{\sigma(j)}\,\omegaform(n_{\sigma(i)},n_{\sigma(j)}).
\]

\begin{theorem}[HK combinatorial formula, 4D]\label{thm-hk-combinatorial}
Haim--Kislev's formula gives
\[
  \cEHZ(K) = \frac{1}{2}\Big[\max_{\sigma,\beta} Q(\sigma,\beta)\Big]^{-1}.
\]
In the centrally symmetric case the factor becomes \(\tfrac14\) with paired normals. The maximizer encodes a simple action-minimizing orbit: velocities appear in the order \(\sigma\) with normalized time weights \(\beta_{\sigma(i)}\) as in the definition of simple orbits. Chaidez--Hutchings show that for non-Lagrangian polytopes any minimizer has combinatorial rotation number \(\rho\le2\), giving a finite search set for algorithms \cite{CH2021}.
\end{theorem}

\section{Forward use}
This chapter fixes conventions and the variational/combinatorial tools used later. \cref{chap:counterexample} will work out the HK2024 counterexample and other explicit polytopes using these conventions. Algorithms and Lean formalizations will cite the facet dynamics, duality, and the HK/CH formula established here.
