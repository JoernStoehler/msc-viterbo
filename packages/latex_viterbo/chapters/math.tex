% !TeX root = ../main.tex
This chapter sets notation and definitions for the rest of the thesis. We follow primarily \cite{CH2021} and secondarily \cite{HK2017,HK2024} in our choice of notation. We assume the reader is already familiar with basic symplectic geometry in a smooth setting (Reeb vector fields, contact forms, symplectic capacities, Hamiltonian dynamics).

\section{Standard symplectic \texorpdfstring{$\R^4$}{R4}}
We work in the standard \(\R^4\) setting. Let
\begin{description}
  \item[Coordinates] \(z=(q_1,q_2,p_1,p_2)\).
  \item[Inner product] \(\inner{x}{y} = x^T y\).
  \item[Norm] \(\abs{x}=\sqrt{\inner{x}{x}}\).
  \item[Volume form] \(\vol = \prod_{i=1}^2 dq_i \wedge dp_i\).
  \item[Almost complex structure] \(\Jmat = \begin{pmatrix}0 & -I_2 \\ I_2 & 0\end{pmatrix}\) so that \(\Jmat^2 = -I_4\) and \(J z = (-p_1,-p_2,q_1,q_2)\).
  \item[Symplectic form] \(\omegaform = \sum_{i=1}^2 dq_i \wedge dp_i\), equivalently \(\omegaform(x,y)=\tfrac12 \inner{\Jmat x}{y}\).
  \item[Liouville 1-form] \(\lambdaform = \tfrac12 \sum_{i=1}^2 (q_i\,dp_i - p_i\,dq_i)\), so \(\lambdaform_z(\dot z) = \tfrac12 \inner{\Jmat z}{\dot z}\) and \(d\lambdaform = \omegaform\).
  \item[Lagrangian 2-plane] An affine 2-plane \(L\subset\R^4\) is Lagrangian iff \(\omegaform|_L \equiv 0\).
\end{description}

\section{Convex bodies and polytopes}

We are generally interested in bounded, convex, star-shaped bodies \(K\subset\R^4\) containing the origin in their interior. In particular, for us polytopes are always bounded, convex, and, unless mentioned otherwise, star-shaped with respect to the origin. We will in the following drop the explicit mention of these properties and simply call such sets \emph{bodies} or \emph{polytopes}.

For an outward unit normal \(n\in\R^4\), \(\abs{n}=1\), and height \(h\in\R\), the half-space \(\{x: \inner{x}{n} \le h\}\) has boundary hyperplane \(\{x: \inner{x}{n}=h\}\). If \(h>0\) the half-space contains the origin; we then call it \emph{positive}.

\paragraph{Irredundant H-representation.} Any polytope \(K\) has a unique irredundant representation with minimal cardinality as the intersection of finitely many positive half-spaces. Writing the outward unit normals and heights as \((n_i,h_i)_{i=1}^F\), \(\abs{n_i}=1\), \(h_i>0\), we have
\[
  K = \bigcap_{i=1}^F \{x: \inner{x}{n_i} \le h_i\}.
\]
Any such representation with bounded \(K\) defines a polytope.

\paragraph{Faces.} We write \(\partial K\) for the boundary. Facets (3-faces) are \(F_i = K \cap \{x: \inner{x}{n_i}=h_i\}\). Where multiple hyperplanes meet we have 2-, 1-, and 0-faces. By irredundancy, every 2-face is the intersection of two unique facets. We don't have an upper bound on how many facets may meet at a 1- or 0-face. The half-space normal and height also are called the facet normal and height.

\paragraph{Support and gauge.} For a body \(K\subset\R^4\), the support function is \(\support_K(v) = \max_{x\in K} \inner{x}{v}\). For polytopes, \(\support_K(n_i) = h_i\). The gauge is \(\gauge_K(v) = \min\{r>0: v \in rK\}\), so \(\gauge_K(x)=1\) on \(\partial K\).

\paragraph{Polar body.} The polar body is \(K^{\polar} = \{y: \support_K(y) \le 1\}\). For polytopes, the polar is a polytope with vertices at the points \(n_i/h_i\).

\section{Smooth vs.~combinatorial viewpoints}
Polytopes approximate any bounded convex star-shaped body arbitrarily well in the Hausdorff metric, and are simpler to compute with algorithmically.
However, polytopes have nonsmooth boundaries, which requires us to define Hamiltonian and Reeb dynamics in a generalized sense, replacing equations with inclusion relations on lower-dimensional faces.
We want our definitions to be compatible with the smooth case, in the sense that if smooth bodies converge wrt the Hausdorff metric to a polytope, the dynamics should converge as well.
Often, it's sufficient to consider sequences \(K \subset B_n \subset K_{\varepsilon_n}\) with \(\varepsilon_n \to 0\) where \(B_n\) are smooth bodies approximating the polytope \(K\) from the outside, and \(K_{\varepsilon}\) is the largest body within Hausdorff distance \(\varepsilon\) of \(K\), called an \(\varepsilon\)-smoothing.
Often, it's then sufficient to prove statements for the limit of smoothings, and other sequences follow by continuity arguments.

\paragraph{\(\varepsilon\)-smoothings.} For a polytope \(K\) and \(\varepsilon>0\), set \(K_\varepsilon := \{x \in \R^4 : \mathrm{dist}(x,K) \le \varepsilon\}\). Then \(K_\varepsilon\) is a bounded convex star-shaped body with \(C^{1,1}\) boundary.

\section{Hamiltonian dynamics}
Now fix a polytope \(K\) with irredundant data \((n_i,h_i)_{i=1}^F\).
We assume the reader is familiar with Hamiltonian vector fields, flows, and orbits in the smooth case.

\paragraph{Standard Hamiltonian.} We set \(\Ham = \gauge_K^2\), the square of the gauge. It is 2-homogeneous, convex, and piecewise quadratic for polytopes. Derivatives fail on rays through the 2-faces of \(K\). On the polytope boundary \(\partial K\) we have \(\Ham \equiv 1\), and on the interior of a facet \(F_i\) we have
\[
  \nable H(x) = \frac{2}{h_i} n_i
\]
so that the subdifferential in general is
\[
  \partial H(x) = \operatorname{conv} \Big\{ \frac{2}{h_i} n_i : x \in F_i \Big\}.
\]
We define facet "velocities"
\[
  p_i := \frac{2}{h_i} \Jmat n_i.
\]

\paragraph{Regular case.} For \(\Ham\in C^1(\R^4)\), the Hamiltonian vector field satisfies \(\omegaform(X_\Ham,\cdot) = - d\Ham(\cdot)\), equivalently \(X_\Ham = 2 \Jmat \nabla \Ham\). The flow \(\dot x = X_\Ham(x)\) preserves level sets of \(\Ham\).
\begin{proof}
  Quick sign check:
  \[
    \omegaform(X_H,v) = \tfrac12 \inner{\Jmat X_H}{v} = -dH(v) = \inner{-\nabla H}{v} \implies X_H = 2 \Jmat \nabla H.
  \]
\end{proof}

\paragraph{Polytope case.} For polytopes \(\Ham\) is not differentiable everywhere; we use the subdifferential \(\partial \Ham(x)\) and the inclusion
\[
  \dot x(t) \in 2 \Jmat \partial \Ham(x(t)) \quad \text{a.e.}
\]
Solutions in \(W^{1,2}(\R,\R^4)\) exist for any initial condition and all time but need not be unique. The boundary \(\partial K\) is invariant.

\section{Reeb dynamics and closed characteristics}
We next consider the Reeb dynamics in the polytope case. Again we assume familiarity with contact topology in the smooth case. The 1-form we are interested in is \(\alpha = \lambdaform|_{\partial K}\). We for now do not assume that \(\partial K\) is of contact type, i.e. we allow that \(\alpha|_{F_i \cap F_j} = 0\) on some 2-faces \(F_i \cap F_j\) (Lagrangian 2-faces). We are slightly more general here than \cite{CH2021}, who assume no Lagrangian 2-faces.

\paragraph{Regular case.} At \(x\in\partial K\) with outward unit normal \(n_x\), any \(v\in T_x\partial K\) satisfies \(\inner{\Jmat v}{\Jmat n_x}=\inner{v}{n_x}=0\). Normalizing to \(\alpha(R)=1\) gives
\[
  \Reeb(x) = \frac{2}{\inner{x}{n_x}}\, \Jmat n_x.
\]
The Reeb flow \(\dot x = \Reeb(x)\) preserves \(\alpha\) and \(d\alpha\). Periodic solutions are Reeb orbits. Closed characteristics are loops \(\gamma\) with \(\dot\gamma(t) \in \R_+ \Reeb(\gamma(t))\) for all \(t\).

\paragraph{Polytope case.} On facet interiors the Reeb vector field matches the facet velocity. Generalized closed characteristics are loops \(\gamma \in W^{1,2}(\T,\partial K)\) with
\[
  \dot\gamma(t) \in \R_+ \mathrm{conv}\{p_i : \gamma(t) \in F_i\} \quad \text{a.e.}
\]
Thus the concepts of generalized Reeb orbits, closed characteristics, and Hamiltonian orbits coincide up to time reparametrization that preserve orientation. We will in the following often talk about "orbits", and mean "Reeb orbits".

\begin{fact}\label{fact:closed-characteristic}
Any closed characteristic \(\gamma\) is parametrized uniquely as \(\gamma\in W^{1,2}(\T,\partial K)\) with period \(T>0\) such that \(\dot\gamma(t) \in \operatorname{conv}\{p_i : \gamma(t) \in F_i\}\) almost everywhere.
\end{fact}

\begin{lemma}[Flow on facets]\label{lem-facet-flow}
If an orbit \(\gamma\) meets the interior of a facet \(F_i\), it does so along a closed linear segment with velocity \(p_i\), entering and exiting at the boundary of \(F_i\) after finite time.
\end{lemma}
\begin{proof}
Immediate from the definition of \(p_i\) on facet interiors and the Reeb/Hamiltonian inclusion.
\end{proof}

At lower-dimensional faces the behavior depends on geometry.

\begin{lemma}[Flow on Lagrangian 2-faces]\label{lem-lagrangian-2face}
If a 2-face \(F_{ij}=F_i\cap F_j\) is Lagrangian, then \(p_i\) and \(p_j\) lie in its tangent plane. Locally the orbit may slide along \(F_{ij}\) with any \(W^{1,2}\) velocity in \(\operatorname{conv}\{p_i,p_j\}\). It enters and exits \(F_{ij}\) after finite time through its boundary (a 1- or 0-face).
\end{lemma}
\begin{proof}
Since \(p_i,p_j\) are tangent, the orbit cannot enter from facet interiors; it passes through the boundary. Also \(0\notin \operatorname{conv}\{p_i,p_j\}\), so \(\alpha\) integrates to a potential on \(F_{ij}\) and the orbit cannot stay forever.
\end{proof}

\begin{lemma}[Flow through non-Lagrangian 2-faces]\label{lem-nonlagrangian-2face}
If \(F_{ij}\) is non-Lagrangian, then the orbit crosses \(F_{ij}\) from one facet to the other at isolated times. The direction is determined by \(\omegaform(n_i,n_j)\): if \(\omegaform(n_i,n_j) > 0\) the orbit crosses from \(F_i\) to \(F_j\); if \(\omegaform(n_i,n_j)<0\) it crosses from \(F_j\) to \(F_i\).
\end{lemma}
\begin{proof}
We use \(\inner{\Jmat n_i}{n_j} = - \inner{\Jmat n_j}{n_i} = \omegaform(n_i,n_j)\). Its sign determines whether \(p_i\) points into or out of the half-space defined by \(F_j\). Locally the orbit consists of linear segments with velocities \(p_i\) and \(p_j\); touching times are isolated and the crossing direction follows the sign.
\end{proof}

\begin{lemma}[Flow on 1-faces]\label{lem-1face-flow}
If \(\gamma\) meets the interior of a 1-face \(F = \bigcap_{k=1}^m F_{i_k}\) (with \(m\ge3\)), there is a unique direction of flow along \(F\). Velocities may vary within the convex cone of incident \(p_{i_k}\). The orbit may enter or exit \(F\) through adjacent 0-, 2-, or 3-faces; the touching time is finite.
\end{lemma}
\begin{proof}
Convexity shows the normals do not convexly combine to zero; neither do the \(p_{i_k}\). Their convex cone lies on a unique half-line, giving the direction. Examples show trajectories can enter/exit as stated.
\end{proof}

\begin{lemma}[Flow on 0-faces]\label{lem-0face-flow}
If \(\gamma\) meets a 0-face where facets \(F_{i_1},\ldots,F_{i_m}\) meet, the orbit crosses through instantaneously. The orbit may enter from or exit to any of the incident facets, 2-faces, or 1-faces.
\end{lemma}
\begin{proof}
  Again we can see from convexity of \(K\) that the \(p_{i_k}\) do not convexly combine to zero, so the orbit cannot stay at the 0-face. Examples show all stated transitions can occur.
\end{proof}

\subsection{Generic behavior}
Some of the above behavior is known to be or conjectured to be non-generic. We briefly state our knowledge here, but won't use it in the rest of the thesis. It justifies why \cite{CH2021} focused on the non-Lagrangian generic case.
We will use the term "generic" formally, i.e. dense and open subset of configuration space.

\begin{lemma}[Non-Lagrangianness is generic]\label{lem-nonlagrangian-generic}
For a fixed number of facets (or vertices), polytopes with no Lagrangian 2-faces form an open dense set in the parameter space of facet normals/heights (respectively vertices).
\end{lemma}
\begin{proof}
Lagrangianness of \(F_{ij}\) is the single equation \(\omegaform(n_i,n_j)=0\), a codimension-one condition. A finite union of such subsets has complement open dense.
\end{proof}

\begin{conjecture}\label{conj-generic-0faces}
For a generic polytope, no closed characteristic passes through a 0-face.
\end{conjecture}
\begin{remark}
This conjecture appears in \cite{CH2021} (Conjecture~1.26): “We expect that Type 2 combinatorial Reeb orbits do not exist for generic polytopes.”
\end{remark}

\subsection{Homotopies of orbits}
Finally, we want to discuss how we can go from \(W^{1,2}\) orbits to simpler combinatorial representatives, using homotopies through orbits. For our algorithm we want to use such combinatorial representatives, instead of working with general \(W^{1,2}\) orbits. Intuitively that makes sense, since \(W^{1,2}\) behavior on flat polytope faces is not that interesting.
We will use the "action" of orbits, which is defined only one section later, all that matters is that the action is preserved along the homotopies we find.

\begin{definition}[Polygonal orbit]\label{def-piecewise-constant-velocity}
A Hamiltonian/Reeb orbit is \emph{polygonal} if time can be partitioned into finitely many open intervals such that during each interval the incident facet set is constant and the velocity is a single incident \(p_i\). Breakpoints need not coincide with face changes.
\end{definition}

\begin{theorem}[Homotopy to polygonal orbit]\label{thm-homotopy-pl}
Any Hamiltonian/Reeb orbit \(\gamma\) is homotopic through Hamiltonian/Reeb orbits of the same action to a polygonal orbit \(\gamma'\).
\end{theorem}
\begin{proof}
The face incidence changes only at isolated times, yielding a finite partition. On 3-faces the velocity is constant. On Lagrangian 2-faces, and on 1-faces, the velocity may vary as an arbitrary \(W^{1,2}\) path with velocities in \(\operatorname{conv}\{p_i: F_i \text{ incident}\}\). The start and end point on the face define a convex set of possible paths. The codomain of the paths is a convex subset of the face. Thus, polygonal paths lie dense in the set of possible paths, and so at least one lies in the set, which we can homotope to. We cannot directly define a decomposition into segments with one segment for each velocity here since the path may leave the face in that case; but we can find at least some finite polygonal approximation that lies within the face, and that we can linearly homotope to. The action of the path depends only on the endpoints since the codomain is flat, so the action is preserved along the homotopy.
\end{proof}

\begin{definition}[Simple orbit]\label{def-simple-orbit-2}
A polygonal Hamiltonian/Reeb orbit is \emph{simple} if each facet velocity \(p_i\) is used at most once.
\end{definition}

\begin{theorem}[Homotopy to simple orbit]\label{thm-min-action-simple}
Any Hamiltonian/Reeb orbit is homotopic through Hamiltonian/Reeb orbits to a simple orbit whose action is non-increasing along the homotopy.
\end{theorem}
\begin{proof}
The argument in \cite{HK2017} uses Clarke's dual principle, so we defer to the next chapter for our recounting of the proof.
\end{proof}

\begin{corollary}[Simple minimum-action orbit]\label{cor-simple-min-action}
There exists a minimum-action Hamiltonian/Reeb orbit that is simple.
\end{corollary}

\begin{example}[Viterbo counterexample degeneracy]\label{ex-viterbo-degeneracy}
The counterexample from \cite{HK2024} has multiple distinct minimum-action closed characteristics that are all homotopic.
\end{example}

\section{Action, EHZ capacity, and systolic ratio}
For a curve \(\gamma\) in \(\R^4\), the action is
\[
  A(\gamma) = \int_\gamma \lambdaform = \int_0^T \lambdaform_{\gamma(t)}(\dot\gamma(t))\,dt = \tfrac12 \int_0^T \inner{\Jmat\gamma(t)}{\dot\gamma(t)}\, dt.
\]
It is invariant under orientation-preserving reparametrization and changes sign if orientation is reversed.

For bodies with enough regularity there is a closed characteristic minimizing the action; the minimum equals the Ekeland--Hofer--Zehnder capacity
\[
  \cEHZ(K) = \min\{A(\gamma) : \gamma \text{ closed characteristic on } \partial K\}.
\]
By continuity of symplectic capacities in the Hausdorff metric this definition extends to polytopes using generalized closed characteristics.

The systolic ratio is
\[
  \sys(K) = \frac{\cEHZ(K)^2}{2\, \vol(K)}.
\]
It is scale and translation invariant; for balls and cylinders of radius \(r\), \(\sys(B(r)) = \sys(Z(r)) = 1\).

\section{Viterbo conjecture (falsified)}

\begin{conjecture}[Viterbo]\label{conj-viterbo}
For any bounded convex body \(K \subset \R^4\), the systolic ratio satisfies \(\sys(K) \le 1\).
\end{conjecture}

\begin{example}[Counterexample]\label{ex-counterexample-viterbo}
\cite{HK2024} constructs a polytope \(K \subset \R^4\) with \(\sys(K) > 1\), falsifying \cref{conj-viterbo}.
\end{example}

The existing literature includes further statements:

\begin{theorem}[Simplex case]\label{thm-simplex-viterbo}
For simplices \(\sys(K) \le 3/4 < 1\). Equality holds only for the orthonormal simplex with vertices \(\{0, e_1, e_2, e_3, e_4\}\) and its symplectomorphic images (translations and linear symplectic maps).
%TODO Cite: master thesis of Haim-Kislev.
\end{theorem}

\begin{theorem}[Mahler conjecture in 2D]\label{thm-mahler-2d-viterbo}
For centrally symmetric convex polygons \(P = -P \subset \R^2\) we have \(\mathrm{area}(P)\,\mathrm{area}(P^{\polar}) \ge 8\), with equality for parallelograms. Then \(K = P \times P^{\polar} \subset \R^4\) satisfies \(\sys(K)\le1\), with equality preserved.
%TODO: Cite
\end{theorem}

Trivial families of counterexamples arise from scaling, symplectomorphisms, and small perturbations of known counterexamples, since \(\sys\) is continuous in the Hausdorff metric. We seek nontrivial counterexamples and a systematic computational search.

\section{Clarke's dual action principle}
We will below always consider the cases of a bounded strictly convex body with smooth boundary, or a polytope. We often skip explicitly tracking the inclusion relations for polytope dynamics, and instead focus on equality on strictly convex smooth bodies; the polytope case then works algebraically analogously.

\theorem{Fenchel duality for gauge and support functions.}
Recall the gauge and support functions:
\[
  \gauge_K(x) = \min\{r>0: x \in rK\},\qquad
  \support_K(y) = \max_{x\in K} \inner{x}{y}.
\]
They are dual norms on \(\R^4\):
\[
  \gauge_K^2(x) = \sup_{y \in \R^4} \big( \inner{x}{y} - \tfrac14 \support_K(y)^2 \big),\qquad
  \tfrac14 \support_K^2(y) = \sup_{x \in \R^4} \big( \inner{x}{y} - \gauge_K(x)^2 \big).
\]
Thus
\[
  \gauge_K^2(x) + \tfrac14 \support_K^2(y) \ge \inner{x}{y},
\]
with equality iff \(y\in\partial \gauge_K^2(x)\) iff \(x\in\partial(\tfrac14 \support_K^2)(y)\).
In the polytope case this identifies supporting facets and contact points by Legendre duality:
\[
  \partial \gauge_K^2(x \in F) = \operatorname{conv}\Big\{ \frac{2}{h_i} n_i : x \in F_i \Big\},
\]
\[
  \partial (\tfrac14 \support_K^2)(y \in \dots) = F
\]
\end{theorem}
\begin{proof}
  Calculation.
\end{proof}

\paragraph{Clarke's dual action principle.}

To set up Clarke's dual action principle, we work in the Sobolev space \(W^{1,2}([0,T],\R^4)\) of absolutely continuous curves with square-integrable weak derivative and period \(T>0\).
We consider the following functionals
\begin{align*}
  A(z) &= \tfrac12 \int_0^T \inner{-\Jmat\dot z(t)}{z(t)}\,dt, \\
  I_g(z) &= \frac{1}{T} \int_0^T \gauge_K^2(-\Jmat\dot z(t))\,dt, \\
  I_h(z) &= T \int_0^T \tfrac14 \support_K^2(-\Jmat\dot z(t))\,dt, \\
\end{align*}
The functionals are all quadratic in spatial scaling, constant in time scaling. The functionals \(A,I_h\) are constant under spatial translation.

An interesting observation is that the Hamiltonian orbits on \(\R^4\) now correspond to the critical points of \(A - T I_g\). This can also be understood as \(T\) serving as a Lagrange multiplier, yielding critical points of \(A\) constrained to level sets of \(\gauge_K\). Intuitively, this again tells us that Hamiltonian orbits and Reeb orbits correspond.

One problem we have with this formulation is that \(I_g\) does not give us much freedom to homotope orbits. On the other hand \(I_h\) only depends on how much time is spent on each velocity, so even full rearrangement of velocities is possible without changing \(I_h\). In this sense, \(I_h\) is a useful framework to find new homotopies of orbits, that \(I_g\) does not show us.

\begin{theorem}[Clarke's dual action principle]\label{thm-clarke-dual}
Critical points of \(A - T I_g\) correspond to Hamiltonian orbits in \(\R^4\). The lagrange multiplier \(T\) can be used to fix the level set. Critical points of \(A - T I_h\) correspond to Hamiltonian orbits as well, but with a potential time and spatial scaling. The lagrange multiplier \(T\) can be used to fix the level set.
\end{theorem}


  
Clarke's dual action principle (specialized to convex bodies) considers
\[
  E = \Big\{ z\in W^{1,2}([0,1],\R^4): \int_0^1 \dot z = 0,\ \int_0^1 \inner{-\Jmat\dot z}{z}\,dt = 1 \Big\},\qquad
  I_K(z)=\tfrac14\int_0^1 \support_K^2(-\Jmat\dot z(t))\,dt.
\]
Critical points of \(I_K\) correspond to generalized characteristics and
\[
  \cEHZ(K) = \inf_{z\in E} I_K(z).
\]
For polytopes \(\support_K\) is piecewise linear, so \(I_K\) is piecewise quadratic in \(\dot z\); minimizers can be taken piecewise affine, leading to the combinatorial model below \cite{HK2017}.

\section{HK/CH combinatorial capacity formula (4D)}
Let \((n_i,h_i)_{i=1}^F\) be the outward unit normals and heights of an admissible polytope \(K\subset\R^4\). Define coefficients \(\beta_i\ge0\) with
\[
  \sum_i \beta_i h_i = 1,\qquad \sum_i \beta_i n_i = 0.
\]
For a permutation \(\sigma\in S_F\), set
\[
  Q(\sigma,\beta)=\sum_{j<i} \beta_{\sigma(i)}\,\beta_{\sigma(j)}\,\omegaform(n_{\sigma(i)},n_{\sigma(j)}).
\]

\begin{theorem}[HK combinatorial formula, 4D]\label{thm-hk-combinatorial}
Haim--Kislev's formula gives
\[
  \cEHZ(K) = \frac{1}{2}\Big[\max_{\sigma,\beta} Q(\sigma,\beta)\Big]^{-1}.
\]
In the centrally symmetric case the factor becomes \(\tfrac14\) with paired normals. The maximizer encodes a simple action-minimizing orbit: velocities appear in the order \(\sigma\) with normalized time weights \(\beta_{\sigma(i)}\) as in the definition of simple orbits. Chaidez--Hutchings show that for non-Lagrangian polytopes any minimizer has combinatorial rotation number \(\rho\le2\), giving a finite search set for algorithms \cite{CH2021}.
\end{theorem}

\section{Forward use}
This chapter fixes conventions and the variational/combinatorial tools used later. \cref{chap:counterexample} will work out the HK2024 counterexample and other explicit polytopes using these conventions. Algorithms and Lean formalizations will cite the facet dynamics, duality, and the HK/CH formula established here.
