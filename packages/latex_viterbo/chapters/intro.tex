We recall a standard inequality that will be useful.

\begin{definition}
A function $f:\mathbb{R}^n \to \mathbb{R}$ is \emph{Lipschitz} with constant $L$ if
\[
  \lvert f(x) - f(y) \rvert \leq L \|x-y\| \quad\text{for all } x,y\in\mathbb{R}^n.
\]
\end{definition}

\begin{theorem}[Triangle inequality]
For any $x,y\in\mathbb{R}^n$, $\|x+y\| \le \|x\| + \|y\|$.
\end{theorem}
\begin{proof}
This follows from Cauchy--Schwarz; see, e.g., \cite{rudin1987real}.
\end{proof}

We also keep a small code listing to exercise the `listings` package:

\begin{lstlisting}[language=Python,caption={Gradient step snippet}]
import numpy as np

def step(x, grad, lr=1e-2):
    # simple gradient descent step
    return x - lr * grad

x = np.array([1.0, 2.0])
grad = np.array([0.5, -0.1])
print(step(x, grad))
\end{lstlisting}
